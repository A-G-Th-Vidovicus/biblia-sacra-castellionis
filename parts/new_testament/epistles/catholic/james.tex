% Copyright (C) 2024 A. G. Th. Vidovicus.

% This program is free software: you can redistribute it and/or modify it
% under the terms of the GNU General Public License as published
% by the Free Software Foundation, either version 3 of the License,
% or (at your option) any later version.

% This program is distributed in the hope that it will be useful,
% but WITHOUT ANY WARRANTY; without even the implied warranty
% of MERCHANTABILITY or FITNESS FOR A PARTICULAR PURPOSE.
% See the GNU General Public License for more details.

% You should have received a copy of the GNU General Public License along
% with this program. If not, see <https://www.gnu.org/licenses/>.

% ┌──────────┐
% │01.07.2024│
% └──────────┘

\startcomponent james
\startbook[
  bookmark={Epistola Iacobi},
   marking={Epistola Iacobi},
     title={Jācōbī Apostolī Epistola},
]

\startintroduction[title={Argūmentum Athanāsiī in Epistolam Jācōbī}]
  Ab auctōribus suīs et ipsa nōmina sortīta sunt epistolae catholicae.
  Hanc enim Jācōbus ad dispersās duodecim tribūs in Chrīstum crēdentēs scrīpsit.
  Scrīpsit autem eam mōre et genere docendī,
  docēns eās dē tantatiōnum discrīmine,
  quaenam ā \DEŌ, quae vērō ex propriō cujus‧que mortālium corde sint;
  et quod nōn verbīs tantum, sed et ipsō opere ostendenda sit fidēs,
  quod‧que nōn audītōrēs lēgis, sed factōrēs justificentur.

  Praecipit etiam dē dīvitibus, nē illī pauperibus in \ECCLĒSIĪS praeferantur,
  sed magis increpentur, tamquam superbī.
  Tandem ubi adflīctōs cōnsōlātus,
  et ad tolerantiam ūsque ad jūdicis adventum servandum adhortātus est,
  dē‧que patientiā, ex iīs quae Jōb adcidērunt, admonuit.
  Praecipit, utī seniōrēs vocentur ad aegrōtōs,
  studeātur‧que, quōmodo errantēs ad vēritātem convertantur,
  eō quod mercēs hujus apud \DOMINUM sit remissiō peccātōrum;
  atque ita terminat epistolam.
\stopintroduction

\startbookchapter % Chapter 1
\startabstract
  Disserit dē patientiā, dē fide, dē animī modestiā ad dīvitēs.
  Deinde ostendit, tentātiōnēs ad mālum nōn ex Deō esse,
  quum omnis bonī auctor sit.
  Quāliter recipiendus sit sermō vītae.
\stopabstract

\vs%{Iac-1-1}
Jācōbus, \DEĪ et \DOMINĪ Jēsū Chrīstī servus,
duodecim tribubus ultrō citrō‧que dispersīs salūtem.

\vs%{Iac-1-2}
Summopere gaudendum exīstimāre, frātrēs meī,
quum in variās incīdistis perīclitātiōnēs,
\vs%{Iac-1-3}
intellegentēs fideī vestrae explōrātiōne patientiam gignī;
\vs%{Iac-1-4}
patientia autem perfectam āctiōnem habeat,
utī perfectī integrī‧que nūllā rē careātis.

\vs%{Iac-1-5}
Quodsī quis vestrum sapientiā caret, postulet ā \DEŌ,
quī omnibus plānē dōnat nec exprobrat, et eī dōnābitur.
\vs%{Iac-1-6}
Postulet autem cum fīdūciā, nihil dubitāns:
quī enim dubitat, similis est flūctuī maris, ventīs āctō et excitō.
\vs%{Iac-1-7}
Neque vērō sē putet is homō quidquam ā \DOMINŌ impetrātūrum,
\vs%{Iac-1-8}
homō duplicis animī, in omnibus suīs āctiōnibus incōnstāns.

\vs%{Iac-1-9}
Glōriētur autem frāter humilis in sublīmitāte suā,
\vs%{Iac-1-10}
et dīves in humilitāte suā: nam ut herbae flōs, sīc perībit.
\vs%{Iac-1-11}
Ut enim exortus cum ārdōre sōl herbam ārefacit,
ejus‧que et flōs dēcidit et decōra speciēs interit;
ita dīves cum suīs cōpiīs marcēscet.
\vs%{Iac-1-12}
Fēlīx homō, quī tentātiōnem sustinet:
adprobātus enim vītae corōnam accipiet,
quam prōmīsit \DOMINUS suīs amātōribus.

\vs%{Iac-1-13}
Nēmō dum tentātur, dīcat \indirectquote{ā \DEŌ tentārī sē}:
\DEUS enim nec malīs umquam tentātur nec tentat ipse quemquam.
\vs%{Iac-1-14}
Sed ā suā quisque cupiditāte adlectus ac inescātus tentātur;
\vs%{Iac-1-15}
deinde cupiditās conceptum parit peccātum,
peccātum porrō perpetrātum gignit mortem.

\vs%{Iac-1-16}
Nōlīte errāre, meī frātrēs cārissimī.
\vs%{Iac-1-17}
Omne bonum dōnum omne‧que perfectum mūnus superne est,
dēscendēns ā \PATRE lūminum,
apud quem nūlla est mūtātiō aut ūlla conversiōnis umbra.
\vs%{Iac-1-18}
Is cōnsultō nōs genuit vēritātis ōrātiōne,
ut essēmus quaedam quasi prīmitiae suōrum operum.

\vs%{Iac-1-19}
Proinde, meī frātrēs cārissimī, ēsto omnis homō
ad audiendum celer, ad loquendum tardus, ad īram tardus:
\vs%{Iac-1-20}
nam hominis īra dīvīnae jūstitiae nōn pāret.
\vs%{Iac-1-21}
Quamobrem, frātrēs, omnibus vitiōrum sordibus excrēmentīs‧que dēpositīs,
cōmiter accipite satīvum sermōnem, quī vestrās animās servāre potest.
\vs%{Iac-1-22}
Este autem iī, quī sermōnī pāreātis, nōn eum dumtaxat audiātis:
aliōquīn vōs ipsōs dēcipitis.
\vs%{Iac-1-23}
Nam sī quis sermōnem audit nec exsequitur,
is perinde est, acsī quis corporis suī faciem contemplētur in speculō;
\vs%{Iac-1-24}
deinde sēipsum speculātus abeat ac prōtinus, quālis sit, oblīvīscātur.
\vs%{Iac-1-25}
Quī vērō in perfectam lībertātis lēgem penitus intrōspexerit et in eā mānserit,
is quī nōn sit oblīviōsus audītor sed operis ecfector, suō factō beātus erit.

\vs%{Iac-1-26}
Sī quis sibi vidētur esse religiōsus apud vōs,
neque tamen linguam suam frēnat sed animum suum fallit, hujus vāna est religiō.
\vs%{Iac-1-27}
Religiō pūrā et impollūtā apud \DEUM et \PATREM haec est:
cūram gerere pūpillōrum et viduārum in eōrum calamitāte,
et intāminātum sē cūstōdīre ā mundō.

\stopbookchapter

\startbookchapter % Chapter 2
\startabstract
  Persōnārum adceptiōnēs discrepāre dīcit ā Chrīstī fidē,
  quam nōn satis est verbīs profitērī,
  nisi et operibus misericordiae et cāritātis praestēmus eamdem.
  Abrāhae exemplō.
\stopabstract

\vs%{Iac-2-1}
Frātrēs meī,
nōlīte ita fidem habēre in glōriōsō \DOMINŌ nostrō Jēsū Chrīstō conlocātam,
utī persōnārum habeātis ratiōnem.
\vs%{Iac-2-2}
Nam sī quis aureōs gestāns annulōs in veste splendidā,
in vestrum conventum ingrediātur,
eōdem‧que intret etiam pauper in obsolētā veste,
\vs%{Iac-2-3}
et vōs in eum, quī vestem gestat splendidam, intueāminī eī‧que dicātis:
\quote{Tū sēde hīc pulcrē}, pauperī autem dicātis:
\quote{Tū istīc stā aut hīc sub meō subselliō sēde},
\vs%{Iac-2-4}
nōnne discrīmen apud vōs fēcistis et male cōgitantēs jūdicēs fuistis?

\vs%{Iac-2-5}
Audīte, meī frātrēs cārissimī:
nōnne \DEUS ēlēgit hujus mundī pauperēs, fidē dīvitēs, hērēdēs‧que rēgnī,
quod suīs prōmīsit amātōribus?
\vs%{Iac-2-6}
Et vōs pauperem dēspicātuī habētis.
Nōnne dīvitēs vōbīs imperant et iīdem vōs in jūs trahunt?
\vs%{Iac-2-7}
Nōnne iīdem pulcrō nōminī maledīcunt, ā quō vōs nuncupāminī?
\vs%{Iac-2-8}
Sī̆quidem legī pārētis rēgiae, quae sīc litterīs est prōdita:
\biblequote{Alterum utī tēipsum dīligitō}, rēctē facitis;
\vs%{Iac-2-9}
sīn persōnīs indulgētis, peccātum committitis, et ā lēge,
utī quī contrā eam faciātis, arguiminī.
\vs%{Iac-2-10}
Quī enim, quamvīs aliōquīn tōtam lēgem observet,
tamen in ūnō dēlinquit, is omnibus tenētur.
\vs%{Iac-2-11}
Nam quī dīxit: \biblequote{Nē adulterātō},
īdem dīxit: \biblequote{Nē occīditō};
quodsī nōn adulterās, sed occīdis, contrā lēgem committis.
\vs%{Iac-2-12}
Sīc loquiminī et sīc agite, utī lībertātis lēge jūdicandī:
\vs%{Iac-2-13}
nam jūdicium erit inclēmēns in eum, quī pietātem nōn exercuerit,
at pietās jūdiciō īnsultat.

\vs%{Iac-2-14}
Quae est ūtilitās, meī frātrēs,
sī quis sē fidem habēre dīcat, et opera nōn habeat?
Num potest eum servāre fidēs?
\vs%{Iac-2-15}
Sī frāter aut soror nūdī sint cottīdiānō‧que alimentō dēstitūtī,
\vs%{Iac-2-16}
et eōs vestrum aliquis
\indirectquote{bona cum pāce abīre calefierī‧que et satiārī} jubeat,
nec iīs dētis, quae corpus requīrit, quid prōdest?
\vs%{Iac-2-17}
Ita fidēs ejusmodī per sē est, utī, sī opera nōn habeat, mortua sit.
\vs%{Iac-2-18}
Itaque dīcet aliquis: \quote{Tū fidem habēs, ego opera habeō}:
ostende mihi tuam fidem sine tuīs operibus,
ego tibi ex meīs operibus fidem meam ostendam.
\vs%{Iac-2-19}
Tū \DEUM ūnum esse crēdis? Rēctē facis; et daemonēs crēdunt atque horrent!
\vs%{Iac-2-20}
Vīs autem intellegere, ō homō vāne, fidem sine operibus mortuam esse?
\vs%{Iac-2-21}
Abrāhāmus, parēns ille noster, nōnne operibus factus est jūstus,
Isaācō fīliō suō in ārā lībandō?
\vs%{Iac-2-22}
Vidēs, utī fidēs ejus opera adjuvāret utī‧que operibus fidēs perfecta sit.
\vs%{Iac-2-23}
Atque ita comprobātum est scrīptum illud, quod sīc sē habet:
\biblequote{Crēdidit Abrāhāmus \DEŌ id‧que eī jūstitiae ductum est},
et \DEĪ amīcus est appellātus.
\vs%{Iac-2-24}
Vidētis ergō, operibus jūstum reddī hominem, nōn sōlum fidē.
\vs%{Iac-2-25}
Itidem Rachaba meretrīx, nōnne operibus justificāta est,
admissīs nūntiīs et aliā viā ēmissīs?
\vs%{Iac-2-26}
Ut enim corpus absque spīritū mortuum est, ita fidēs sine operibus mortua est.
\stopbookchapter

\startbookchapter
\startabstract
  Utī doceat linguam hominī Chrīstiānō esse coercendam
  fideī et cāritātis sānctae repāgulīs,
  conmoda et inconmoda, quae ab eā prōveniunt, ostendit,
  et quid hūmāna sapientia ā caelestī distet.
\stopabstract

\vs%{Iac-3-1}
Nōlīte multī esse magistrī, frātrēs meī,
scientēs nōs esse graviōrēs poenās datūrōs:
\vs%{Iac-3-2}
multa enim dēlinquimus omnēs.
Sī quis in verbīs nōn dēlinquit, is perfectus homō est,
quī possit etiam tōtum frēnāre corpus.
\vs%{Iac-3-3}
Frēnōs quidem equōrum ōribus adhibēmus,
utī nōbīs parcant, eōrum‧que tōtum corpus agimus.
\vs%{Iac-3-4}
Item‧que tantae nāvēs, tam vehementibus impulsae ventīs,
tantulō gubernāculō, quōcumque vult gubernātōris impetus, aguntur;
\vs%{Iac-3-5}
sīc et lingua, quae parvum membrum est, magnās vir|īs habet.
Ēn quantulus ignis quantam māteriam incendit!
\vs%{Iac-3-6}
Et lingua ignis est, scelerum mundus.
Sīc sē habet in nostrīs membrīs lingua, tōtum corpus contāmināns
et aevī cursum īnflammāns et ipsa ā gehennā īnflammanda.
\vs%{Iac-3-7}
Omnis enim et ferārum nātūra et volucrum serpentum‧que et aquātilium
hūmānō ingeniō domātur domita‧que est;
\vs%{Iac-3-8}
at linguam nēmō potest hominum domāre, ecfrēnātum malum,
mortiferō refertam venēnō.
\vs%{Iac-3-9}
Eā \DEUM et \PATREM conlaudāmus,
eādem hominēs exsecrāmur ad \DEĪ similitūdinem factōs,
\vs%{Iac-3-10}
prōdeunte ex eōdem ōre laudātiōne et exsecrātiōne.
Nōn decet, meī frātrēs, haec ita fierī.
\vs%{Iac-3-11}
Num fōns ex eōdem forāmine dulcī aquā mānat et amārā?
\vs%{Iac-3-12}
Potest‧ne, meī frātrēs, aut fīcus oleās aut vīt|īs fīcus edere?
Sīc nūllus fōns et falsam et dulcem aquam ēmittere.

\vs%{Iac-3-13}
Quī in vōbīs sapiēns est et intellegēns,
ostendat bonīs mōribus sua opera cum sapientiae mānsuētūdine.
\vs%{Iac-3-14}
Quodsī amāram invidiam et simultātem habētis in animō,
nōlīte contrā vērum jactāre vōs atque mentīrī.
\vs%{Iac-3-15}
Nōn ea est ista sapientia, quae superne proficīscātur,
sed terrestris, hūmānā, daemoniacā:
\vs%{Iac-3-16}
ubī̆ enim invidia est et simultās, ibī̆ est incōnstantia omnis‧que vitiōsa rēs.
\vs%{Iac-3-17}
At superne oriēns sapientia, prīmum casta est,
deinde pācifica, aequa, obsequiōsa, pietāte bonīs‧que referta frūctibus,
sevēra, et minimē simulātrīx;
\vs%{Iac-3-18}
frūctus vērō jūstitiae in pāce seritur colentibus pācem.
\stopbookchapter

\startbookchapter
\startabstract
  Ēnumerat inconmoda, quae ex carnīs operibus prōveniunt.
  Animī submissiōnem subjungit et \unclassical{repūrgātiōnem}
  ā superbiā, dētractiōne, et ab oblīviōne propriae īnfirmitātis.
\stopabstract

\vs%{Iac-4-1}
Unde apud vōs duella et pugnae?
Nōnne hinc, ex vestrīs libīdinibus, quae mīlitant in vestrīs membrīs?
\vs%{Iac-4-2}
Concupīscitis nec obtinētis; invidētis et aemulāminī nec adsequī potestis;
pugnātis et duelligerātis. Nōn obtinētis, quoniam nōn postulātis;
\vs%{Iac-4-3}
postulātis et nōn adipīsciminī, quia male postulātis,
ut in vestrās impendātis libīdinēs.
\vs%{Iac-4-4}
Ō dēgenerēs et virī et fēminae! Nescītis mundī amīcitiās \DEĪ esse inimīcitiās?

Quī vult amīcus esse mundī, inimīcus ecficitur \DEĪ.
\vs%{Iac-4-5}
An putātis illud ināniter esse litterīs prōditum, scīlicet:
\quote{Ad invidiam prōpēnsus est is, quī in vōbīs habitat, \SPĪRITUS}?
\vs%{Iac-4-6}
At majus cōnfert beneficium;
itaque
\quote{%
  \bibleallusion{\DEUS} --- inquit ille ---
  \bibleallusion{et superbīs adversātur et modestōrum fautor est}}.

\vs%{Iac-4-7}
Quamobrem pārēte \DEŌ, resistite \DIABOLŌ, et is vōs fugiet.
\vs%{Iac-4-8}
Adcēdite \DEUM, et adcēdet vōs.
Pūrgāte manus, ō improbī, et animōs lūstrāte!
\vs%{Iac-4-9}
\obsolete{Quī estis animō duplicī,} adflīctāte vōs, et lūgēte ac plōrāte;
rīsus vester in lūctum convertātur, et gaudium in maerōrem.
\vs%{Iac-4-10}
Submittite vōs \DOMINŌ, et vōs ēvehet.

\vs%{Iac-4-11}
Frātrēs, nōlīte alius aliī dētrahere;
quī frātrī dētrahit frātrem‧que suum damnat, legī dētrahit lēgem‧que damnat.
Quodsī lēgem damnās, nōn lēgis exsecūtor es, sed condemnātor.
\vs%{Iac-4-12}
Ūnus est lēgislātor atque jūdex, quī et servāre potest et perdere.
Tū vērō quis es, quī alterum damnās?

\vs%{Iac-4-13}
Age jam, quī dīcitis:
\quote{%
  Hodiē aut crās proficīscēmur in illam urbem,
  et illīc ūnum annum agēmus et mercābimur et lucrābimur},
\vs%{Iac-4-14}
quum nesciātis, quid sit crās futūrum. Quālīs enim est vīta vestra!
Vapor est, quī paulīsper exstat, deinde ēvānēscit;
\vs%{Iac-4-15}
quum dīcere dēbeātis:
\quote{Sī volet \DOMINUS et sī vīvēmus, hoc aut illud faciēmus}.
\vs%{Iac-4-16}
Nunc in vestrīs īnsolentiīs vōs jactātis; omnis ejusmodī jactantia prāva est.
\vs%{Iac-4-17}
Ergō quī rēctē facere nōvit neque facit, is in vitiō est!
\stopbookchapter

\startbookchapter
\startabstract
  Dīvīnī jūdiciī sevēritātem dīvitibus dēnūntiat, ēnumerātā eōrum superbiā,
  utī pauperēs, audientēs dīvitum īnfēlīcem exitum, aequō animō adflīctiōnēs,
  utī Jōb, etiam quum īnfirmantur, ferant.
\stopabstract

\vs%{Iac-5-1}
Jam vērō, ō dīvitēs, plōrāte ejulantēs ob vestrās miseriās vōbīs ēventūrās.
\vs%{Iac-5-2}
% putrēre is anteclassical
Dīvitiae vestrae putruērunt, vestīmenta vestra ā tineīs ērōsa sunt,
\vs%{Iac-5-3}
aurum et argentum aerūginōsum, eōrum‧que aerūgō contrā vōs testimōnium dīcet,
vestrās‧que carnēs ignis in mōrem cōnsūmet:
% the following line is utterly incomprehensible for me
L. poenam G. vōbīs congessistis in ultimum tempus.
\vs%{Iac-5-4}
Mercēs ipsa mercēnāriōrum, quī vestrōs messuērunt agrōs,
ā vōbīs retenta quirītātur,
messōrum‧que querēlae in \DOMINĪ armipotentis aur|īs penetrārunt.
\vs%{Iac-5-5}
Luxuriāstis ac lascīvīstis in terrīs,
geniō indulsistis, tamquam ad laniēnae diem.
\vs%{Iac-5-6}
Condemnāstis, occīdistis īnsont||īs, vōbīs nōn resistentēs.

\vs%{Iac-5-7}
Quamobrem dūrāte, frātrēs, ūsque ad \DOMINĪ adventum.
Ipse agricola exspectat eximium terrae frūctum, et quidem in eō dūrat,
dum \bibleallusion{prīmōrem sērōtinam‧que pluviam} accipiat.
\vs%{Iac-5-8}
Dūrāte vōs quoque, rōborāte vestrōs animōs: nam \DOMINĪ adventus īnstat.
\vs%{Iac-5-9}
Nōlīte alius adversus alium gemere, frātrēs, nē damnēminī:
ēn jūdex prō foribus adest.
\vs%{Iac-5-10}
Exemplum capite, frātrēs meī, tolerantiae ac patientiae dē vātibus,
quī \DOMINĪ nōmine locūtī sunt.
\vs%{Iac-5-11}
Equidem beātōs jūdicāmus eōs, quī patiuntur;
Jōbī patientiam audīvistis, et quem eī fīnem Dominus
--- quippe \bibleallusion{misericors et clēmēns} --- dederit, vīdistis.

\vs%{Iac-5-12}
In prīmīs autem, meī frātrēs,
nē jūrātōte neque caelum neque terram nec aliud ūllum jūsjūrandum,
sed estō ejusmodī sermō vester, ut \quote{Etiam} sit etiam,
et \quote{Nōn} nōn, nē in simulātiōnem cadātis.

\vs%{Iac-5-13}
Adflīgitur vestrum aliquis? Subplicet. Laetus est quis? Psāllat.
\vs%{Iac-5-14}
Aegrōtat vestrum aliquis? Advocet \ECCLĒSIAE seniōrēs,
quī eum in \DOMINĪ nōmine ungant oleō prō eō‧que subplicent;
\vs%{Iac-5-15}
et fideī subplicātiō labōrantem servābit eum‧que adlēnābit \DOMINUS et,
sī peccāta commīserit, ignōscentur eī.
\vs%{Iac-5-16}
Cōnfitēminī invicem dēlicta et subplicāte alius prō aliō, utī sānēminī.
Multum valet ecficit‧que virī bonī precātiō.
\vs%{Iac-5-17}
Ēliās homō erat iīsdem, quibus nōs, obnoxius.
Et tamen, quum precibus petiisset, nē plueret,
nōn pluit in terram per annōs tr|īs et sex mēnsēs;
\vs%{Iac-5-18}
eōdem rūrsum precātō, et caelum dedit imbrem et terrā suum frūctum germināvit.
\vs%{Iac-5-19}
Frātrēs, sī quis vestrum ā vērō aberrāverit et eum quispiam revocāverit,
\vs%{Iac-5-20}
intellegat eum, quī sontem ab errōris suī via revocāverit,
animam ex morte vindicāsse multitūdinem‧que tēxisse peccātōrum.
\stopbookchapter
\stopbook
\stopcomponent

%%% Local Variables:
%%% coding: utf-8
%%% mode: context
%%% TeX-master: t
%%% End:
