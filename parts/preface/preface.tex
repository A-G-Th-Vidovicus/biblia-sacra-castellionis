% Copyright (C) 2024 A. G. Th. Vidovicus.

% This program is free software: you can redistribute it and/or modify it
% under the terms of the GNU General Public License as published
% by the Free Software Foundation, either version 3 of the License,
% or (at your option) any later version.

% This program is distributed in the hope that it will be useful,
% but WITHOUT ANY WARRANTY; without even the implied warranty
% of MERCHANTABILITY or FITNESS FOR A PARTICULAR PURPOSE.
% See the GNU General Public License for more details.

% You should have received a copy of the GNU General Public License along
% with this program. If not, see <https://www.gnu.org/licenses/>.

\startcomponent preface
\starttitle[
  bookmark={Praefatiō},
  marking={Praefatiō},
  title={%
    Sebastiānus Castelliō\\
    Eduardō Sextō,\\
    Angliae Rēgī Clārissimō,\\
    salūtem},
]

Quum sacrārum litterārum librōs, ā mē in Latīnum sermōnem conversōs,
in lūcem ēmitterem, serēnissime Rēx Angliae, suāsērunt amīcī,
ut eōs tuae Majestātī dēdicārem;
idque ut facere vellem, tribus ratiōnibus persuāsērunt.
Prīmum, quod dīcerent, nūllī convenientius sacrās litterās offerrī posse,
quam eī, cujus rēgnum asȳlum esset iīs,
quī propter sacrārum litterārum studium atque dēfensiōnem vexārentur.
Deinde, quod tū nūper hanc eamdem trānsferendī librōs sacrōs prōvinciam
hominibus doctīs mandāvissēs, sed ūnī̆us obitū, impedītus fuissēs.
Postrēmō, quod praeter cēterās disciplīnās atque linguās, etiam Latīnitātis,
vel in prīmīs (cujus nōs hīc nōn nūllam ratiōnem habuimus),
studiōsus essēs et ad eam rem magistrum ērudītum habērēs.

Hīs causīs adductus sum, utī tibi, Rēgī inlūstrissimō,
ego, īnfimae conditiōnis homō, vigiliās meās offerre audērem spērāns fore,
ut quantō in altiōre dignitātis fastīgiō tē Deī benignitās conlocāsset,
tantō clēmentius tenuēs admitterēs, amōre Chrīstī, quī, quum Deō foret aequālis,
nōn dubitāvit ad infimōs sē dēmittere.
Quod sī placet, et sī hanc trānslātiōnem, Rēx, legere nōn recūsās,
expōnam tibi jam quod fuerit in hōc negōtiō meum īnstitūtum.

Ego operam dedī, utī fidēlis et Latīna et perspicua esset haec trānslātiō,
quoad ejus fierī potes; nē quem deinceps ōrātiōnis obscūritās aut horriditās
aut etiam interpretātiōnis īnfidēlitās ab hōrum librōrum lēctiōne revocāret.
Sed perspicuitātis et fidēlitātis potissimam ratiōnem dūximus:
nam quod ad Latīnitātem adtinet,
est ōrātiō nihil aliud quam reī quaedam quasi vestis, et nōs sartōrēs sumus.
Rēs quidem manet eadem, nec ōrātiōnis ēlegantia fit mēliō,
nec vīlitāte dēterior; neque rem vērē amat is,
quem ab eā cognōscendā retrahit inculta ōrātiō.
Quīn etiam vidēmus {\EVANGELII} arcāna nōbīs trādita esse verbīs impolītīs
et ē mediā indoctōrum plēbe dēsūmptīs,
nē quid inde hominum ēloquentiae tribuerētur.

Cēterum quoniam in hōc studiō versantem, necesse erat nōn sōlum verba, sed etiam
rēs ipsās --- sine quibus verba saepe intellegī nōn possunt --- perpendere:
sī quid interim animadvertī obscūrius,
nōn sōlum quod ad verba, sed etiam quod ad rem pertinēret,
id cōnātus sum paucīs inlūstrāre;
atque in verbīs quidem dēclārandīs, fuī aliquantō cōnfīdentior.
Vērum rēs ut parcius adtingerem,
in causā fuit partim praesēns trānsferendī īnstitūtum,
quod circā verba versātur,
partim, et quidem multō magis, ignōrantia mea.
Quum sint enim hī librī dē rēbus dīvīnīs scrīptī, necesse est nōs,
quantō magis hominēs sumus, tantō minōrem hōrum intellegentiam habēre.

Et profectō sī vērum fatērī volumus, est adhūc nostrum
saeculum in profundīs ignōrantiae tenebrīs dēmersum,
cujus reī certissimum testimōnium sunt
tam gravēs, tam pertinācēs, tam perniciōsae dissēnsiōnēs;
tam multī et iīdem inritī conventūs dē hīsce contrōversiīs,
tantusque numerus cottīdiē nāscentium librōrum,
et eōrum inter sēsē tōtō caelō diffīdentium.

Sī enim ūnus Deī spīritus et ūna vēritās est, necesse est,
in quibus īdem spīritus, eadem vēritās īnsit, eōs ūnum esse,
idemque sentīre spīrituālibus in rēbus;
et sī nōbīs clārissimae vēritātis orta diēs esset,
numquam tot obscūrās librōrum adcenderēmus lucernās.
Atque equidem mihi hujus ignōrantiae causam quaerentī,
vidēbātur inventū difficilis; % supine in ablative
Neque enim tribuī haec litterārum et artium ignōrātiōnī potest,
quum nostrō saeculō nihil fierī possit ērudītius.
Et tamen tantīs studiīs, tantā linguārum cognitiōne, tot artibus per tot annōs,
tantum abest, ut multum prōfectum sit, ut in diēs in dēterius abeant rēs.
Itaque quum adtentius hanc rem cōnsīderārem,
vīsus sum mihi hujus ignōrantiae ūnam vērissimamque causam invēnisse:
vitiōsitātem et inpietātem.

Itaque dicit Danieli angelus:
\quote{Inpie agent inpii, nec intellegent ulli inpii}.
Et David contra:
\quote{Jovae metuentibus patefit ejus arcanum}.
Et Esaias:
\quote{Preme oraculum, obsigna disciplinam apud discipulos meos}.

Quod si Jovae arcana in his libris obcultata nosse volumus, metuendum nobis et
colendus Jova est, eique oboediendum, ut decet ejus discipulos. Haec vera ad
divinarum rerum cognitionem via, haec una ad hoc divinorum arcanorum sigillum
aperiendum clavis est: quam qui habebit, nullam aliam requiret: qui non habebit,
frustra ceteras admovebit huic serae. Quod si esset in nobis verus Dei amor
atque metus, et si tantum studii et operae in eo poneremus, quantum alii in
comparanda pecunia, alii in litteris, alii in disciplinis, alii in honore, alii
alia in re ponunt, non solum abesset crassa haec --- liceat mihi dicere quae
sentio --- crassa haec, inquam, quae tenet saeculus, ignorantia: verum etiam tantae
esset Jovae cognitionis, hoc est, verae pietatis plena terra, quanta nobis, si
sapimus, promissa est. Atque ita nasceretur illus vere aureum saeculum, quo
essent omnes divinitus docti: fieretque, quod armipotens Jova pollicitus est his
verbis:
\quote{Cudent ex ensibus suis vomeres, et ex spiculis falces: nec gentes aliae
  aliis
arma inferent, nec amplius duella discent degentque sub suis quisque vitibus ac
ficubus, exterrente nullo, quoniam Jovae armipotentis os loquitur}.

\stoptitle
\stopcomponent

%%% Local Variables:
%%% coding: utf-8
%%% mode: context
%%% TeX-master: t
%%% End:
