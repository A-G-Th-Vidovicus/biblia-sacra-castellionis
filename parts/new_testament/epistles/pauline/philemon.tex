% Copyright (C) 2024 A. G. Th. Vidovicus.

% This program is free software: you can redistribute it and/or modify it
% under the terms of the GNU General Public License as published
% by the Free Software Foundation, either version 3 of the License,
% or (at your option) any later version.

% This program is distributed in the hope that it will be useful,
% but WITHOUT ANY WARRANTY; without even the implied warranty
% of MERCHANTABILITY or FITNESS FOR A PARTICULAR PURPOSE.
% See the GNU General Public License for more details.

% You should have received a copy of the GNU General Public License along
% with this program. If not, see <https://www.gnu.org/licenses/>.

% ┌──────────┐
% │01.07.2024│
% └──────────┘

\startcomponent philemon
\startbook[
  bookmark={Epistola ad Philemonem},
   marking={Epistola ad Philemonem},
     title={Paulī Apostolī Epistola ad Philēmonem},
]

\startintroduction[title={Argūmentum Athanāsiī in Epistolam ad Philēmonem}]
  Hanc mīsit ē Rōmā. Argūmentum ejus hoc est:
  Onēsimus, servus Philēmonos, aufūgerat et ad Paulum vēnerat,
  atque ab illō in fidē īnstrūctus jam,
  et ūsuī illī̆us ministrandō commodus erat.
  Dē hōc itaque scrībit Philēmoni illīquē commendat Onēsimum,
  utī sincērē ergā eum adficiātur,
  nec amplius eum habeat utī servum, sed utī frātrem.
  Hortātur illum etiam, ut hospitium sibi pāret, ut ubi vēnerit,
  manendī locum inveniat; et sīc fīnit epistolam.
\stopintroduction

\startbookchapter
\startabstract
  Paulus argūmentum humile hīc tractāns
  suō tamen mōre sublīmis ad Deum ēvehitur.
  Fugitīvum servum et fūrem Philēmoni remittēns, prō illō dēprecātur veniam,
  et graviter dē aequitāte Chrīstiānā dēprecātur.
\stopabstract

\vs%{Phl-1-1}
Paulus, Jēsū Chrīstī vīnctus et Tīmotheus frāter
Philēmoni cārō amīcō adjūtōrīque nostrō,
\vs%{Phl-1-2}
et Apphiae cārissimae et Archippō commīlitōnī nostrō et \ECCLĒSIAE domūs tuae:
\vs%{Phl-1-3}
grātiam et pācem ā Deō \PATRE nostrō et \DOMINŌ Jēsū Chrīstō.

\vs%{Phl-1-4}
Agō Deō meō grātiās, semper tuī mentiōnem faciēns in meīs subplicātiōnibus,
\vs%{Phl-1-5}
audiēns, quā cāritāte et fidē praeditus sīs
tum in \DOMINUM Jēsum, tum ergā sānctōs omn|īs
\vs%{Phl-1-6}
Quō istā līberālitāte tuae in Chrīstō Jēsū conlocātae fideī fīat,
ut intellegātur, quantum sit in vōbīs bonī;
\vs%{Phl-1-7}
magnā quidem laetitiā et cōnsōlātiōne adficimur ex cāritāte tuā,
quod sānctōrum animī per tē, frāter, recreātī sint.

\vs%{Phl-1-8}
Proinde quum magnam in Chrīstō auctōritātem habeam imperandī tibi obficium,
\vs%{Phl-1-9}
propter cāritātem potius rogō,
tālis ego, hoc est, Paulus senex et nunc etiam Jēsū Chrīstī vīnctus:
\vs%{Phl-1-10}
tē, inquam, rogō dē meō fīliō, quem in meīs vinculīs genuī, Onēsimō:
\vs%{Phl-1-11}
eō, quī tibi aliquandō fuit inūtilis, nunc et tibi et mihi ūtilis, quem dīmīsī.
\vs%{Phl-1-12}
Tū vērō eum, hoc est, meī intimum, recipe:
\vs%{Phl-1-13}
quem ego volēbam apud mē retinēre,
utī mihi tuā vice ministrāret in \ĒVANGELIĪ vinculīs.

\vs%{Phl-1-14}
Sed injussū tuō nihil voluī facere, utī nōn coāctum,
sed voluntārium esset tuum beneficium.
\vs%{Phl-1-15}
Fortassis enim ideō ad tempus abfuit, ut eum reciperēs aeternum:
\vs%{Phl-1-16}
nōn jam utī servum, sed plūs quam servum, frātrem etiam mihi cārissimum,
nēdum tibi et carnis et \DOMINICAE religiōnis nōmine.

\vs%{Phl-1-17}
Quārē sī mē socium habēs, accipe eum utī mē.
\vs%{Phl-1-18}
Quodsī quid tē laesit aut tibi dēbet, id mihi imputātō.
\vs%{Phl-1-19}
Ego Paulus scrīpsī meā manū, ego dissolvam;
ut interim illud tibi nōn dīcam, quod tū mihi etiam tēipsum dēbēs.
\vs%{Phl-1-20}
Age frāter, grātificāre mihi per \DOMINUM, recreā meum animum per \DOMINUM!

\vs%{Phl-1-21}
Frētus oboedientiā tuā scrīpsī tibi, sciēns tē plūs etiam, quam petō, factūrum.
\vs%{Phl-1-22}
Praetereā parā mihi hospitium: spērō enim fore,
utī vōbīs per vestrās precēs condōner.

\vs%{Phl-1-23}
Salūtant tē Epaphrās, meae in Chrīstō Jēsū captīvitātis comes,
\vs%{Phl-1-24}
et Mārcus, Aristarchus, Dēmās, Lūcās, conlēgae meī.

\vs%{Phl-1-25}
Dominī nostrī Jēsū Chrīstī favor adsit animīs vestrīs. \obsolete{Āmēn.}
\stopbookchapter
\stopbook
\stopcomponent

%%% Local Variables:
%%% coding: utf-8
%%% mode: context
%%% TeX-master: t
%%% End:
