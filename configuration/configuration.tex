% configuration.tex

% Copyright (C) 2024 A. G. Th. Vidovicus.

% This program is free software: you can redistribute it and/or modify it
% under the terms of the GNU General Public License as published
% by the Free Software Foundation, either version 3 of the License,
% or (at your option) any later version.

% This program is distributed in the hope that it will be useful,
% but WITHOUT ANY WARRANTY; without even the implied warranty
% of MERCHANTABILITY or FITNESS FOR A PARTICULAR PURPOSE.
% See the GNU General Public License for more details.

% You should have received a copy of the GNU General Public License along
% with this program. If not, see <https://www.gnu.org/licenses/>.

% ┏━━━━━━━━━━━━━━━━━━━━━━━━━━━━━━━━━━━━━━━━━━━━━━━━━━━━━━━━━━┓
% ┃ Biblia Sacra ex Sebastiani Castellionis interpretatione: ┃
% ┃                   Editio Contemporanea                   ┃
% ┃                        01.07.2024                        ┃
% ┗━━━━━━━━━━━━━━━━━━━━━━━━━━━━━━━━━━━━━━━━━━━━━━━━━━━━━━━━━━┛

\startenvironment configuration

% Module for raw input translation
\usemodule[translate]

% Loading Lua tables with input translations and macros
\startluacode
  require("input_translations")
  require("macros")
  require("font_features")
\stopluacode

\mainlanguage[la]

% Bibliography in xml file
\usebtxdataset[bibliography.xml]

\definefontfeature[classical][default]
                  [no_macrons=yes,fimacron_to_fi=yes,no_j=yes]
\definefontfeature[no_macrons][default]
                  [no_macrons=yes,fimacron_to_fi=yes]
\definefontfeature[no_j][default]
                  [no_j=yes]
\definefontfeature[oxford][default]
                  [no_macrons=yes,fimacron_to_fi=yes,no_j=yes,no_v=yes]
\definefontfeature[y_tilde_to_y_macron][default]
                  [y_tilde_to_y_macron=yes]

% Latin Modern, but with proper ȳ
\definefontfamily[lmrm][rm][Latin Modern Roman]
                 [features=y_tilde_to_y_macron,sc=file:lmromancaps10-regular]

% Latin Modern without j and macrons
\definefontfamily[lmrm_classical][rm][Latin Modern Roman]
                 [features=classical,sc=file:lmromancaps10-regular]

% Latin Modern without macrons
\definefontfamily[lmrm_no_macrons][rm][Latin Modern Roman]
                 [features=no_macrons,sc=file:lmromancaps10-regular]

% Latin Modern without j
\definefontfamily[lmrm_no_j][rm][Latin Modern Roman]
                 [features=no_j,y_tilde_to_y_macron,sc=file:lmromancaps10-regular]

% Latin Modern without j, v, and macrons
\definefontfamily[lmrm_oxford][rm][Latin Modern Roman]
                 [features=oxford,sc=file:lmromancaps10-regular]

% Latin Modern Small Caps without j and macrons
\definefontfamily[lmsc][rm][Latin Modern Roman]
                 [features=classical,sc=file:lmromancaps10-regular]

% Standard body font
\setupbodyfont[lmrm]
\translateinput[ȳ][ỹ]
%\translateinput[Ȳ][Ỹ]

% ┌────────┐
% │Metadata│
% └────────┘

\setupdocument[
    metadata:author={Sebastianus Castellio},
     metadata:title={Biblia Sacra ex Sebastiani Castellionis interpretatione},
   metadata:subject={Biblia Sacra ex Sebastiani Castellionis interpretatione},
  metadata:keywords={%
                      acta,%
                      actus,%
                      apocalypsis,%
                      apostoli,%
                      apostolus,%
                      biblia,%
                      castellio,%
                      christus,%
                      epistola,%
                      epistolae,%
                      evangelia,%
                      evangelium,%
                      foedus,%
                      jacobus,%
                      jesus,%
                      johannes,%
                      juda,%
                      lucas,%
                      marcus,%
                      matthaeus,%
                      paulus,%
                      petrus,%
                      sacra,%
                      sacrum,%
                      sebastianus,%
                      scriptura,%
                      testamentum,%
                      translatio,%
                    },
]

% ┌───────────────┐
% │Quotation marks│
% └───────────────┘

% Following the Swiss convention for quotation marks,
% using «…» for quotes and ‹…› for subquotes
\setuplanguage[la][
   leftquotation={\leftguillemot\nobreak\quotespace},   % «
  rightquotation={\quotespace\nobreak\rightguillemot},  % »
       leftquote={\guilsingleleft\nobreak\quotespace},  % ‹
      rightquote={\quotespace\nobreak\guilsingleright}, % ›
]

% ┌──────┐
% │Macros│
% └──────┘
\ctxlua{write_out_macros()}

% Color Definitions
\definecolor[normal_text_color][black]
\definecolor[accent_color][darkred]
\definecolor[verse_number_color][darkred]
\definecolor[christ_quote_color][darkred]
\definecolor[obsolete_words_color][darkgray]

% Space definitions
\define\quotespace{\hairspace}
\define\versenumberspace{\hairspace}

% ┌────────────────┐
% │Paragraphs setup│
% └────────────────┘

% No indentation for paragraphs,
% instead they are separated by a big space
\setupwhitespace[big]

% Big tolerance for horizontal whitespace
\tolerance=8000

% ┌───────────┐
% │Fonts setup│
% └───────────┘

% For now I use Latin Modern font,
% since it is æsthetic and not too wide;
% GFS Bodoni is also a nice font,
% but looks good only in Greek
% \definefontfamily[bodoni][rm][GFS Bodoni]

% ┌───────────────────────────────────────────┐
% │Unconditional letter and word substitutions│
% └───────────────────────────────────────────┘

% A dirty hack, since there is no ‘ȳ’ in Latin Modern font
% Caution: must be above modeset block!
%\translateinput[ȳ][$\overline{\mathrm{y}}$]
%\translateinput[Ȳ][$\overline{\mathrm{Y}}$]

\ctxlua{write_out_input_translations("default")}

% ┌──────────────────────┐
% │Default B5 page layout│
% └──────────────────────┘

\setuppapersize[B5] % 17.6cm : 25cm

\setuplayout[
  backspace=3cm,
  leftedge=0.2cm,
  leftedgedistance=0.1cm,
  leftmargin=2.5cm,
  leftmargindistance=0.2cm,
  width=13.1cm,
  rightmargindistance=0.2cm,
  rightmargin=1cm,
  rightedgedistance=0.1cm,
  rightedge=0.2cm,
  %
  topspace=1.5cm,
  top=1cm,
  topdistance=0.5cm,
  header=0.5cm,
  headerdistance=0.5cm,
  height=middle,
  footerdistance=0.5cm,
  footer=0.5cm,
  bottomdistance=0.5cm,
  bottom=1cm,
  %
]
\setupheadertexts[margin][\bf\pagenumber][][][\bf\pagenumber]
\setupheadertexts[text][][\bf\getmarking[book]][][]

\startmodeset
  % ┌───────────┐
  % │Paper setup│
  % └───────────┘

  % Good dimensions for my 4:3 e-ink tablet
  [tablet_4_3]{
    \definepapersize[tablet_4_3][
      width=12cm,
      height=16cm,
    ]

    \setuppapersize[tablet_4_3]

    \setuplayout[
      backspace=1cm,
      width=middle,
      margin=0.9cm,
      margindistance=0.1cm,
      topspace=0.25cm,
      header=0.5cm,
      headerdistance=0.25cm,
      height=middle,
      footerdistance=0.25cm,
      footer=0.5cm,
      bottomdistance=0cm,
      bottom=0cm,
    ]

    % Empty header
    \setupheadertexts[margin][][][][]
    \setupheadertexts[text][][][][]
    % Page numbers in the middle of the footer
    \setupfootertexts[\bf\pagenumber]
  }

  % Good dimensions for my 8:5 tablet
  [tablet_8_5]{
    \definepapersize[tablet_8_5][
      width=12cm,
      height=19.2cm,
    ]

    \setuppapersize[tablet_8_5]

    \setuplayout[
      backspace=1cm,
      width=middle,
      margin=0.9cm,
      margindistance=0.1cm,
      topspace=0.25cm,
      header=0.5cm,
      headerdistance=0.25cm,
      height=middle,
      footerdistance=0.25cm,
      footer=0.5cm,
      bottomdistance=0cm,
      bottom=0cm,
    ]

    % Empty header
    \setupheadertexts[margin][][][][]
    \setupheadertexts[text][][][][]
    % Page numbers in the middle of the footer
    \setupfootertexts[\bf\pagenumber]
  }

  % Good dimensions for any 16:9 tablet
  [tablet_16_9]{
    \definepapersize[tablet_16_9][
      width=12cm,
      height=21.333cm,
    ]

    \setuppapersize[tablet_16_9]

    \setuplayout[
      backspace=1cm,
      width=middle,
      margin=0.9cm,
      margindistance=0.1cm,
      topspace=0.25cm,
      header=0.5cm,
      headerdistance=0.25cm,
      height=middle,
      footerdistance=0.25cm,
      footer=0.5cm,
      bottomdistance=0cm,
      bottom=0cm,
    ]

    % Empty header
    \setupheadertexts[margin][][][][]
    \setupheadertexts[text][][][][]
    % Page numbers in the middle of the footer
    \setupfootertexts[\bf\pagenumber]
  }

% ┌─────────────────────────────┐
% │Letter and word substitutions│
% └─────────────────────────────┘

  [ae_oe_ligatures]{
    % Mediæval convention: ‘(ae, oe, ë) → (æ, œ, e)’;
    % when ‘(ae, oe)’ and ‘(æ, œ)’ are distinguished,
    % there is no need for diæresis ‘ë’
    \ctxlua{
      write_out_input_translations("ae_oe_ligatures")
      write_out_input_translations("no_diaeresis")
    }
  }

  [no_assimilation]{
    % No assimilation of letters, e.g. ‘immortalis → inmortalis’
  }

  [c_to_ch]{
    % Changing ‘c’ to ‘ch’ in some words,
    % e.g. ‘pulcer → pulcher’
    \ctxlua{write_out_input_translations("c_to_ch")}
  }

  [classical]{
    % Classical Latin conventions, which cover
    % etymological hyphenation (patterns at ./lua/lang-classla.lua),
    % no macrons, no ‘j’ letter, ‘ae, oe’ spelling, etc.
    \setupbodyfont[lmrm_classical]
    \translateinput[ji][i]
    \translateinput[ȳ][y]
    \translateinput[Ȳ][Y]
    \ctxlua{
      write_out_input_translations("i_to_e_in_acc_plur")
      write_out_input_translations("no_quum")
    }
  }

  [ecclesiastical]{
    % Ecclesiastical Latin conventions, which cover
    % Italian hyphenation (provided as default by ConTeXt),
    % no macrons, ‘j’ is used, ‘æ, œ’ spelling, etc.
    \setuphyphenation[method=traditional]

    % Default ConTeXt Latin hyphenation does not cover vowels with macrons,
    % so we explicitly project them to vowels without macrons
    \definehyphenationfeatures[ignore_macrons_when_hyphenating][characters={āēīōūȳ}]
    \sethyphenationfeatures[ignore_macrons_when_hyphenating]

    \setuplanguage[la][patterns=la]
    \setupbodyfont[lmrm_no_macrons]
    \translateinput[ȳ][y]
    \translateinput[Ȳ][Y]
    \ctxlua{
      write_out_input_translations("ae_oe_ligatures")
      write_out_input_translations("c_to_ch")
      write_out_input_translations("gn_to_n")
      write_out_input_translations("i_to_e_in_acc_plur")
      write_out_input_translations("m_to_n")
      write_out_input_translations("s_to_r")
    }
  }

  [emdash_to_endash]{
  % Changing ‘---’ to ‘--’
    \ctxlua{write_out_input_translations("emdash_to_endash")}
  }

  [gn_to_n]{
  % Changing old-fashioned forms, e.g. ‘gnātus → nātus’
    \ctxlua{write_out_input_translations("gn_to_n")}
  }

  [no_h]{
  % Spelling many words without ‘h’
    \ctxlua{write_out_input_translations("no_h")}
  }

  [no_uti]{
    % Original Castellion’s translation does not use ‘utī’, only ‘ut’,
    % while I decided to use ‘ut’ before vowels and ‘h’, and ‘utī’ otherwise
    \ctxlua{write_out_input_translations("no_uti")}
  }

  [m_to_n]{
    % Changing ‘m’ to ‘n’is a few words,
    % e.g. ‘quemdam → quendam’
    \ctxlua{write_out_input_translations("m_to_n")}
  }

  [s_to_r]{
    % Changing ‘(colōr, honōs) → (color, honor)’
    \ctxlua{write_out_input_translations("s_to_r")}
  }

  [u_to_i]{
    % ‘monumentum → monimentum’
    \ctxlua{write_out_input_translations("u_to_i")}
  }

  [i_to_e]{
    % Changing i-forms,
    % e.g. ‘(clāvim, clāvī, clāvīs) → (clāvem, clāve, clāvēs)’
    \ctxlua{write_out_input_translations("i_to_e")}
  }

  [i_to_e_in_acc_plur]{
    % Changing i-forms in accusative plural, e.g. clāvīs → clāvēs
    \ctxlua{write_out_input_translations("i_to_e_in_acc_plur")}
  }

  [division]{
    % Changing words written as one word to separate ones,
    % e.g. ‘quamobrem → quam ob rem’
    \ctxlua{write_out_input_translations("division")}
  }

  [no_macrons]{
    % Quantity of vowels is usually not indicated,
    % but I like to keep short and long vowels distinct
    \translateinput[ȳ][y]
    \translateinput[Ȳ][Y]
    \setupbodyfont[lmrm_no_macrons]
  }

  [no_diaeresis]{
    % Changing ‘ë → e’
    \ctxlua{write_out_input_translations("no_diaeresis")}
  }

  [no_hyphenation]{
    % Disabling the hyphenation; not recommended!
    \setuphyphenation[method=none]
  }

  [no_j]{
    % In Classical and Contemporary Latin, letter ‘j’ is not used
    \setupbodyfont[lmrm_no_j]
  }

  [no_quum]{
    % In Classical Latin, both the conjunction ‘quum’
    % and the præposition ‘cum’ are spelled ‘cum’,
    % but I prefer the archaic distinction
    % EXCEPTIONS:
    % Words ‘antīquus’ and ‘equus’ are usually spelled as such,
    % not as ‘antīcus’ and ‘ecus’
    \ctxlua{write_out_input_translations("no_quum")}
  }

  [oxford]{
    % Ancient convention followed by the Oxford University Press:
    % it is similar to classical convention, but letters ‘v/U’ are not used
    \setupbodyfont[lmrm_oxford]
    \translateinput[ji][i]
    \translateinput[ȳ][y]
    \translateinput[Ȳ][Y]
    \ctxlua{
      write_out_input_translations("i_to_e_in_acc_plur")
      write_out_input_translations("no_quum")
    }
  }

  [default]{
  % Configuration used when no modes are specified;
  % currently empty
  }
\stopmodeset

\startnotmode[no_assimilation]
  % In Classical Latin when two consonants are next to each other,
  % the first one very often gets assimilated by the second one;
  % e.g. ‘(inmortalis, obferre) → (immortalis, offerre)’
  \ctxlua{write_out_input_translations("assimilation")}
\stopnotmode

\startnotmode[ecclesiastical]
  % Using etymological hyphenation in every case
  % besides ecclesiastical option
  \setuplanguage[la][patterns=classla]
\stopnotmode

% ┌──────────────────────────┐
% │Hyperlinks and interaction│
% └──────────────────────────┘

\setupinteraction[
  state=start,
  color=accent_color,
]


\placebookmarks[book,bookchapter][book][force=yes]

% Turning on table of contents in pdf viewers
\setupinteractionscreen[option=bookmark]

\definesectionblock[cover][cover]
\definesectionblock[innertitle][innertitle]
\definesectionblock[backcover][backcover]

% ┌──────────────┐
% │Page numbering│
% └──────────────┘

\setupsectionblock[cover][
  before={
    \setcounter[userpage][0]
    \setuppagenumbering[
      alternative=singlesided,
         location=none,
    ]
    \setupuserpagenumber[viewerprefix=Cover]
  },
]

\setupsectionblock[innertitle][
  before={
    \resetcounter[userpage]
    \setuppagenumbering[
      alternative=doublesided,
         location=none,
    ]
  },
]

\setupsectionblock[frontpart][
  before={
    \resetcounter[userpage]
    \setuppagenumbering[
      alternative=doublesided,
         location=none,
    ]
  },
]

\setupsectionblock[bodypart][
  before={
    \resetcounter[userpage]
    \setuppagenumbering[
      alternative=doublesided,
         location=none,
    ]
  },
]

\setupsectionblock[backpart][
  before={
    \setuppagenumbering[
      alternative=doublesided,
         location=none,
    ]
  },
]

\setupsectionblock[backcover][
  before={
    \setcounter[userpage][0]
    \setuppagenumbering[
      alternative=singlesided,
         location=none,
    ]
    \setupuserpagenumber[viewerprefix=BackCover]
  },
]

\definestructureconversionset[cover:pagenumber][][characters]
\definestructureconversionset[backcover:pagenumber][][characters]
\definestructureconversionset[innertitle:pagenumber][][characters]
\definestructureconversionset[frontpart:pagenumber][][romannumerals]
%\definestructureconversionset[bodypart:pagenumber][][arabicnumerals]

\setuplabeltext[section=Caput~]

\setuplines[
  indenting={yes,small},
  %option=packed,
  after={\nobreak},
]

% ┌───────────────┐
% │Verse numbering│
% └───────────────┘

\definecounter[VerseCounter][
  prefix=no,
     way=bysection,
]

% Verse numbering
\define\vs{
  \incrementcounter[VerseCounter]%
  \high{%
    \tfxx\color[verse_number_color]{%
      \rawcountervalue[VerseCounter]%
    }%
  }\nobreak\hairspace%
}

% ┌───────────────────────────────┐
% │Parts of Bible and their styles│
% └───────────────────────────────┘

\definehead[book][chapter]
\definehead[bookchapter][section]
\definehead[introduction][subject]
\definehead[adnotation][subject]

\setuphead[introduction][
  beforesection={\startabstract},
   aftersection={\stopabstract},
]

\setuphead[adnotation][
  beforesection={\startabstract},
   aftersection={\stopabstract},
]

\setuphead[title][
   align=middle,
  number=no,
    page=yes,
   style={\lmsc\scd},
]

\setuphead[book][
   align=middle,
  header=empty,
  number=no,
    page=yes,
   style={\lmsc\scd},
]

\setuphead[bookchapter][
            align=middle,
            color=normal_text_color,
             page=no,
  sectionsegments=section,
            style={\lmsc\scd},
]

\setuphead[introduction][
  align=middle,
   page=no,
  style={\lmsc\scb},
]

\setuphead[adnotation][
  align=middle,
   page=no,
  style={\lmsc\sca},
]

\definestartstop[abstract][
  color=normal_text_color,
  style={\itx},
]

% ┌────────────────────────┐
% │Various quotation styles│
% └────────────────────────┘

\definedelimitedtext[quote][
   left={\symbol[leftquotation]},
  right={\symbol[rightquotation]},
]
\definedelimitedtext[Lquote][
   left={\symbol[leftquotation]},
  right={},
]
\definedelimitedtext[Rquote][
   left={},
  right={\symbol[rightquotation]},
]

\definedelimitedtext[christquote][
   left={\symbol[leftquotation]},
  right={\symbol[rightquotation]},
  color=christ_quote_color,
]
% Rather inelegant, but quotations do not play nicely with lines
\definedelimitedtext[Lchristquote][
   left={\symbol[leftquotation]},
  right={},
  color=christ_quote_color,
]
\definedelimitedtext[Rchristquote][
   left={},
  right={\symbol[rightquotation]},
  color=christ_quote_color,
]

\definedelimitedtext[godquote][
   left={\symbol[leftquotation]},
  right={\symbol[rightquotation]},
]
\definedelimitedtext[biblequote][
   left={\symbol[leftquotation]},
  right={\symbol[rightquotation]},
  style=italic,
]

\definedelimitedtext[subquote][
   left={\symbol[leftquote]},
  right={\symbol[rightquote]},
]
\definedelimitedtext[subchristquote][
   left={\symbol[leftquote]},
  right={\symbol[rightquote]},
  color=christ_quote_color,
]
\definedelimitedtext[subbiblequote][
   left={\symbol[leftquote]},
  right={\symbol[rightquote]},
  style=italic,
]

\definehighlight[bibleallusion][
  style=italic,
]

\definehighlight[indirectquote][
  %style=slanted,
]

\definehighlight[indirectbiblequote][
  %style=slanted,
]

\definehighlight[indirectchristquote][
  %style=slanted,color=darkred,
]

\definehighlight[indirectsubquote][
  %style=slanted,
]

\definehighlight[num][
  %style=slanted,
]

\definehighlight[name][
  style=slanted,
]

% For words and sentences which are in Castellio’s translation,
% but are not in modern Bible editions
\definehighlight[obsolete][
  color=obsolete_words_color,
]

\definehighlight[unclassical][]

% For words added by me
\definedelimitedtext[added][
   left={\lbrack},
  right={\rbrack},
]

% Title written over Jesus’ cross
\definehighlight[tit][
  style=bold,
]

\definehighlight[explication][
  %style=italic,
]
\definehighlight[implication][
  %style=italic,
]

% For Latin word forms which I deem to be dubious
\definehighlight[dubious][
  %style=italic,
]

% ┌───────────────────────┐
% │Table of contents style│
% └───────────────────────┘

% Including books of Bible in the table of contents
\setupcombinedlist[content][list={book}]

\setuplist[
  alternative=c, % produces lots of dots
  width=1.75em,
]

% ┌──────────────────┐
% │Bibliography style│
% └──────────────────┘

\setupbtxlist[
  alternative=c,
  stopper={.},
  width=1.75em,
]

\setupbtx[
  stopper:initials={.\space}]

\setupbtxlabeltext[la][
  default:and=\btxcomma\space\nospace,
  default:Editor={editor},
  default:Editors={editores},]
\stopenvironment

%%% Local Variables:
%%% coding: utf-8
%%% mode: context
%%% TeX-master: t
%%% End:
