% Copyright (C) 2024 A. G. Th. Vidovicus.

% This program is free software: you can redistribute it and/or modify it
% under the terms of the GNU General Public License as published
% by the Free Software Foundation, either version 3 of the License,
% or (at your option) any later version.

% This program is distributed in the hope that it will be useful,
% but WITHOUT ANY WARRANTY; without even the implied warranty
% of MERCHANTABILITY or FITNESS FOR A PARTICULAR PURPOSE.
% See the GNU General Public License for more details.

% You should have received a copy of the GNU General Public License along
% with this program. If not, see <https://www.gnu.org/licenses/>.

\startcomponent 3_john
\startbook[
  bookmark={Epistola tertia Iohannis},
   marking={Epistola tertia Iohannis},
     title={Jōhannis Apostolī Epistola Tertia},
]

\startintroduction[title={Argūmentum Athanāsiī in Tertiam Jōhannis Epistolam}]
  Ista quoque Jōhannis est, ut et inscrīptiō habet, missa est autem ad Gājum.
  Et prīmum quidem laudat eum propter testimōnium hospitālitātis,
  quod habēbat ab omnibus, hortātur‧que eum, ut in prōpositō persistat,
  ac frātrēs recipiat ac dēdūcat.
  % acc. Διοτρεφή ?
  Adcūsat autem Diotrephē, quod is nōn sōlum nihil praebēbat pauperibus,
  sed etiam aliīs prohibēbat ac multa nūgābātur.
  Tāl|īs aliēnōs esse dīcit ā vēritāte neque cognōscere \DEUM;
  Dēmētrium vērō laudat, optimum illī testimōnium inpendēns.
\stopintroduction

\startbookchapter
\startabstract % gen. Διοτρεφοῦς ?
  Hospitālitātem Gāiī laudat et, Diotrephūs petulantiam adcūsāns, Gājum monet,
  utī benefaciendō persevēret. Postrēmō Dēmētrium conmendat.
\stopabstract

\vs%{3Io-1-1}
% Γάϊος, G1050
Senior Gājō cārissimō, quem ego vērē amō.

\vs%{3Io-1-2}
Cārissime, cupiō tē ante omnia fēlīciter agere ac valēre,
quemadmodum animō fēlīciter agis.
\vs%{3Io-1-3}
Laetātus sum enim vehementer,
quum vēnissent frātrēs et tuam vēritātem testātī essent,
ut ex vēritāte vīvās.
\vs%{3Io-1-4}
Majōrem laetitiam nūllam habeō, quam ut audiam, meōs gnātōs ex vēritāte vīvere.

\vs%{3Io-1-5}
Cārissime, fēlīciter agis, quidquid ergā frātrēs et hospitēs facis,
\vs%{3Io-1-6}
quī tuam cāritātem in \ECCLĒSIAE cōnspectū testātī sunt.
Quōs, rēctē faciēs, sī, utī \DEŌ dignum est, dēdūxeris.
\vs%{3Io-1-7}
Nam prō ejus nōmine profectī sunt: nihil‧que ab extrāneīs adcēpērunt.
\vs%{3Io-1-8}
Nōs vērō tālēs adcipere dēbēmus, utī vēritātis sīmus adjūtōrēs.

\vs%{3Io-1-9}
% Διοτρεφής, G1361
Scrīpsī \ECCLĒSIAE; sed quī vult eōrum esse prīmārius, Diotrephēs,
nōn admittit nōs.
\vs%{3Io-1-10}
Itaque, sī vēnerō, conmemorābō, quālia patret facinora,
quī in nōs verbīs inprobīs garriat;
nec hīs contentus, nec ipse frātrēs admittat,
et volentēs prohibeat ex‧que \ECCLĒSIĀ ējiciat.

\vs%{3Io-1-11}
Cārissime, nē imitāre malum, sed bonum.
Quī bene facit, ā \DEŌ est; quī vērō male facit, \DEUM nōn nōvit.

\vs%{3Io-1-12}
% Δημήτριος, G1216
Dēmētrius et omnium et ipsī̆us vēritātis testimōniō conmendātus est;
et nōs idem testāmur, et vōs, vērum esse testimōnium nostrum, scītis.

\vs%{3Io-1-13}
Multa habērem scrībenda tibi, sed nōlō tibi chartā et ātrāmentō scrībere;
\vs%{3Io-1-14}
brevī tē, utī spērō, vidēbō, et praesentēs conloquēmur.

\vs%{3Io-1-15}
Valē. Salūtant tē amīcī. Salūtā amīcōs nōminātim.
\stopbookchapter
\stopbook
\stopcomponent

%%% Local Variables:
%%% coding: utf-8
%%% mode: context
%%% TeX-master: t
%%% End:
