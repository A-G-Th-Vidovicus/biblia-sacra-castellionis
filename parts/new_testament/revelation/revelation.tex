% Copyright (C) 2024 A. G. Th. Vidovicus.

% This program is free software: you can redistribute it and/or modify it
% under the terms of the GNU General Public Licēnse as published
% by the Free Software Foundation, either version 3 of the Licēnse,
% or (at your option) any later version.

% This program is distributed in the hope that it will be useful,
% but WITHOUT ANY WARRANTY; without even the implied warranty
% of MERCHANTABILITY or FITNESS FOR A PARTICULAR PURPOSE.
% See the GNU General Public Licēnse for more details.

% You should have received a copy of the GNU General Public Licēnse along
% with this program. If not, see <https://www.gnu.org/licēnses/>.

% ┌──────────┐
% │01.07.2024│
% └──────────┘

\startcomponent revelation
\startbook[
  bookmark={Apocalypsis Iohannis},
   marking={Apocalypsis Iohannis},
     title={Apocalypsis Jōhannis Theologī},
]
\startadnotation[title={Adnotātiō interpretis}]
  Dē hujus auctōre librī dubitātur, titulus Jōhannem Theologum nōminat.
  Quis fuerit aut quandō fuerit, nōn magis sum sollicitus,
  quam dē vīnī dōliō aut tempore, dum vīnum sit bonum.
  Fuisse quidem hujus librī auctōrem vērum vātem {\DEĪ}‧que discipulum,
  persuāsum habeō; nec dē eō magis dubitō, quam dē Jōhannis \ĒVANGELIŌ.
  Et tamen hujus librī vix mīllēsimam partem intellegō.
\stopadnotation

\startintroduction[title={Argūmentum Athanāsiī in Apocalypsin Jōhannis}]
  Hōc nōmine vocātur hic liber, proptereā quod hanc revēlātiōnem ipse Jōhannēs
  Ēvangelista et Theologus in Patmō īnsulā dictā, \DOMINICŌ diē vīdit,
  jussūs‧que cōnscrīpsit, ut ad septem \ECCLĒSIĀS mitteret, vidēlicet istās:
  Ephesum, Smyrnam, Pergamon, Thyatīra, Sardīs, Philadelphīam, et Lāodicēam;
  quae vērō in hāc vīsiōne vīdit, multa sunt ac differentia.
  Circā fīnem perditiōnem etiam Antichristī cum \DIABOLŌ vīdit.

  Jubētur autem prīmum scrībere singulīs praedictārum \ECCLĒSIĀRUM angelīs,
  secundum cujusque āctiōnēs. Multās itaque et admīrābilīs vīsiōnēs vīdit,
  utpote septem candēlābra aurea et in mediō illōrum similem \FĪLIŌ hominis;
  et interpretātiōnem adcēpit: candēlābra illa esse septem \ECCLĒSIĀS,
  et eum quī in mediō erat esse \DOMINUM.

  Deinde jānuam vīdit apertam in caelō et \DOMINUM in thronō sedentem
  et vīgintī quattuor seniōrēs,
  sedent|īs super thronōs et adōrant|īs \DOMINUM.
  Vīdit etiam septem sigilla aperīrī,
  et ūnōquōque apertō, visiō quaedam adpārēbat.
  Vīdit item septem angelōs habent|īs septem tubās,
  et ūnōquōque illōrum clangente, fīēbat signum;
  septimō vērō clangente, audīvit dīcent|īs:
  \quote{Rēgnum mundī, \DOMINĪ factum est}.
  Et \ARCAM\ \TESTĀMENTĪ vīdit in caelō.

  Deinde mulierem parturientem vīdit, et \DRACŌNEM ignītum,
  quī illam persequēbātur.
  Mulier quidem servāta est in sōlitūdine, \DRACŌ vērō abjectus in ign|im.
  Post ista, \BĒSTIAM vīdit habentem cornua decem et capita septem,
  et diadēma ejus blasphēmiā plēnum erat.
  Nōmen vērō ejus nōn patefēcit, sed numerum nōminis ejus,
  nempe sēscentī sexāgintā sex. %666

  Et virginēs audīvit canent|īs et \AGNUM sequentēs.
  Vīdit etiam angelum per medium caelī volantem, et post hunc alterum,
  post alterum tertium quoque.
  Deinde et nūbem candidam vīdit, et īnsidentem illī similem \FĪLIŌ hominis,
  in capite corōnam auream habentem, et in manū suā falcem auream.
  Dēnique septem angelōs vīdit, habentem septem plāgās novissimās,
  et phialās aureās plēnās īrā \DEĪ.

  Posthāc ablātus est \SPĪRITŪ in sōlitūdinem,
  et vīdit mulierem sedentem super \BĒSTIAM coccinam, habentem capita septem,
  et indicātum est eī ex hāc futūrum esse Antichristum.
  Vīdit etiam \BĒSTIAM ac rēgēs terrae duellum facere colent|īs cum eō,
  quī sedēbat super equum album;
  et adprehēnsa est \BĒSTIA, et cum ipsō pseudoprophēta, quī est Antichristus,
  et vīdit eōs abjicī in stāgnum ignis.
  Post ista vīdit angelum dēscendentem dē caelō, et abjicientem Sātānān cum hīs
  quī charactērem ipsī̆us habēbant in abyssum, et claudentem superne,
  nē posthāc amplius sēdūcat gent|īs, dōnec cōnsūmptī sint mīlle annī.
  Martyrēs deinde vīdit cum glōriā in Chrīstō rēgnant|īs,
  et quod Sātānās post mīlle annōs solvendus sit,
  et abjiciendus in stāgnum ignis cum Antichristō in saecula saeculōrum.

  Posthāc vīdit caelum novum, terram novam, ac \JEROSOLYMAM novam,
  et fluvium aquae vīvae, ac servōs \DEĪ contemplant|īs faciem ejus,
  et habent|īs nōmen ejus in frontibus suīs.
  Quum haec vīdisset Jōhannēs, prōcidit volēns adōrāre angelum,
  quī sibi hāc omnia ostenderat; at angelus prohibuit illī̆us cōnātum dīcēns:
  \quote{Nōn mē, sed \DEUM adōrā! Ego enim cōnservus tuus sum}.

  Postquam autem ista vīdit, audīvit ab ipsō nostrō \DOMINŌ Jēsū,
  quod existēns sit, et īdem ipse sit \VERBUM\ \DEĪ,
  quod posteriōribus temporibus propter nōs cārō factum est.
  Homō, inquam, perfectus est, et \FĪLIUS hominis vocātus est.

  Sunt et aliī quidem librī variī,
  praeter praedictōs utrī̆usque tum Veteris tum Novī Īnstrūmentī,
  quōrum aliīs contrādīcitur, aliī apocryphī vocantur.
\stopintroduction

\startbookchapter %Chapter 1
\startabstract
  Indicat doctrīnae genus, quod hīc tractātur: nempe illī̆us,
  quī est \PRĪNCIPIUM et \FĪNIS.
  Deinde septem candēlābrōrum et stēllārum mystērium explicat.
\stopabstract

\vs%{Apc-1-1}
Jēsū Chrīstī patefactiō, quam dedit eī \DEUS, utī servīs suīs ostenderet,
quae brevī futūra sunt, et missō angelō suō,
per eum significāvit servō suō Jōhannī,
\vs%{Apc-1-2}
quī \DEĪ sermōnem et Jēsū Chrīstī testimōnium ac quaecumque vīdit, testātus est.
\vs%{Apc-1-3}
Fēlīx, quī legit quī‧que ōrāculī dicta audiunt,
et quae in eō scrīpta sunt, exsequuntur: nam tempus īnstat.

\vs%{Apc-1-4}
Jōhannēs septem Asiānīs \ECCLĒSIĪS favōrem et pācem,
ab eō, quī est, et quī fuit, et quī ventūrus est,
et ā septem spīritibus, quī sunt ante ejus solium,
\vs%{Apc-1-5}
et ā Jēsū Chrīstō, teste illō fidēlī, prīmōgenitō ex mortuīs,
et rēgum terrae prīncipe.

% Originally, the following verse was part of Apc 1,6.
Eī, quī nōs amāvit suō‧que sanguine ā peccātīs abluit,
\vs%{Apc-1-6}
et rēgēs ac sacerdōtēs \DEŌ et \PATRĪ suō fēcit,
eī glōria et inperium in perpetua saecula. Āmēn.

\vs%{Apc-1-7}
\bibleallusion{Ventūrus est cum nūbibus}, \bibleallusion{eum}‧que
\bibleallusion{cernet} omnis oculus, et quī eum
\bibleallusion{compūnxērunt, dē‧que eō plangent omnēs terrārum nātiōnēs}.
Etiam, profectō.

\vs%{Apc-1-8}
\christquote{Ego sum \ALPHA et \ŌMEGA%
                           \obsolete{, \PRĪNCIPIUM et \FĪNIS}},
inquit \DOMINUS\ \added{\DEUS},
quī est, quī fuit, et quī ventūrus est, omnipotēns.

\vs%{Apc-1-9}
Ego Jōhannēs, quī et frāter sum vester et socius in calamitāte
in‧que rēgnō et patientiā Jēsū Chrīstī,
quum in īnsulā nōmine Patmō,
propter \DEĪ sermōnem propter‧que Jēsū Chrīstī testimōnium, essem,
\vs%{Apc-1-10}
dīvīnitus adflātus diē \DOMINICŌ,
audīvī post mē magnam quasi tubae vōcem dīcentis:
\vs%{Apc-1-11}
\christquote{%
  \obsolete{Ego sum \ALPHA et \ŌMEGA, \PRĪMUS et \ULTIMUS.}
  Tū, quod cernis, scrībe in librō
  et mitte \added{septem} \obsolete{Asiānīs} \ECCLĒSIĪS:
  Ephesum et Smyrnam et Pergamum et Thyatīra
  et Sardēs et Philadelphīam et Lāodicēam}.
\vs%{Apc-1-12}
Tum ego mē convertī, utī vidērem, quae vōx esset adlocūta mē;
conversus‧que vīdī septem aurea candēlābra
\vs%{Apc-1-13}
et in mediō septem candēlābrōrum \bibleallusion{similem hominis \FĪLIŌ,
indūtum veste tālārī}, et ad mammās \bibleallusion{cīnctum aureō} cingulō;
\vs%{Apc-1-14}
\bibleallusion{ejus} autem \bibleallusion{caput et crīnēs erant albī, ut}
alba \bibleallusion{lāna aut nix, oculī ut ignis} flamma;
\vs%{Apc-1-15}
\bibleallusion{pedēs chalcolibanō similēs}, tamquam in fornāce ignītī;
\bibleallusion{vōx ejus quālis est aquārum multārum vōx}.
\vs%{Apc-1-16}
Habēbat in suā manū dexterā septem stēllās,
ex‧que ejus ōre gladius anceps acūtus prōdībat;
ejus adspectus erat quālis micat cum suā virtūte sōl.

\vs%{Apc-1-17}
Eum cōnspicātus ego, ad ejus pedēs paene mortuus cecidī;
at ille, dexterā suā manū mihi inpositā, dīxit:
\christquote{%
  Nē metue! Ego sum \PRĪMUS et \ULTIMUS, quī vīvō
\vs%{Apc-1-18}
  et fuī mortuus et jam sum vīvus in perpetua saecula.
  Ita est: et habeō \ORCĪ clāv|īs atque mortis.
\vs%{Apc-1-19}
  Scrībe, quae vīderīs quae‧que sunt et quae sunt posthāc futūra.
\vs%{Apc-1-20}
  Arcānum septem stēllārum, quās in meā dexterā vīdistī,
  et septem aureōrum candēlābrōrum est hoc:
  septem stēllae angelī sunt septem \ECCLĒSIĀRUM;
  septem autem candēlābra, quae vīdistī, septem \ECCLĒSIAE sunt.
\stopbookchapter

\startbookchapter %Chapter 2
\startabstract
  Jubētur Jōhannēs scrībere, quae \DOMINUS necessāria nōverat \ECCLĒSIĪS:
  Ephesiōrum, Smyrnēnsium, Pergamēnōrum, Thyatīrēnsium;
  utī, quae ab \APOSTOLĪS adcēpērunt, teneant.
\stopabstract

\vs%{Apc-2-1}
  Ephesīnae \ECCLĒSIAE angelō sīc scrībitō:

  Haec dīcit, quī septem stēllās suā dexterā tenet,
  quī in mediō septem aureōrum candēlābrōrum ambulat:
\vs%{Apc-2-2}
  Novī tua facta, labōrem, et patientiam, utī‧que malōs ferre nequeās,
  et eōs explōrāverīs, quī sē profitentur apostolōs, nec sunt,
  quōs tū falsōs dēprehendistī;
\vs%{Apc-2-3}
  et tolerāstī patientiā‧que ūteris et propter meum nōmen labōrās inde fessus.
\vs%{Apc-2-4}
  Sed habeō dē tē, quod expostulem, quod cāritātem tuam prīstinam omīsistī.
\vs%{Apc-2-5}
  Quamobrem mementō, unde excīderīs,
  et ad sānitātem redī ac priōra opera facitō;
  aliōquīn adoriar tē brevī tuum‧que candēlābrum ex suō locō trānsmovēbō,
  nisi ad sānitātem redieris.
\vs%{Apc-2-6}
  Quamquam id habēs, quod Nicolaitārum facta ōdistī, quae ego quoque ōdī.

\vs%{Apc-2-7}
  Quī habet aurem, audiat quid \SPĪRITUS dīcat \ECCLĒSIĪS:
  Victōrī dabō vēscī arbore vītālī, quae in mediō est dīvīnī pōmāriī.

\vs%{Apc-2-8}
  Item angelō \ECCLĒSIAE Smyrnaeōrum scrībitō:

  Haec dīcit \PRĪMUS et \ULTIMUS, quī fuit mortuus, et revīxit:
\vs%{Apc-2-9}
  Novī tua facta et calamitātem et pauperiem --- quum tamen dīves sīs ---
  eōrum‧que inpium dictum, quī sē Jūdaeōs esse dīcunt nec sunt,
  sed Sātānae caterva.
\vs%{Apc-2-10}
  Nōlī metuere, quae passūrus es.
  Datūrus est quidem \DIABOLUS vestrum nōnnūllōs in cūstōdiam, ut explōrēminī;
  ferētis‧que decem diērum adflīctiōnem.
  Estō fidēlis ad mortem ū̆sque, et tibi dabō vītae corōnam.
\vs%{Apc-2-11}
  Quī habet aurem, audiat quid \SPĪRITUS dīcat \ECCLĒSIĪS:
  Quī vīcerit, nōn laedētur ā secundā morte.

\vs%{Apc-2-12}
  Item Pergamēnsae \ECCLĒSIAE angelō scrībitō:
  % orig. Pergamēnsis

  Haec dīcit, quī gladium habet ancipitem acūtum:
\vs%{Apc-2-13}
  Novī tua facta, et ubī̆ habitēs, ubī̆ Sātānae solium est;
  tenēs‧que meum nōmen nec meam fidem negāstī,
  etiam iīs temporibus, quibus Antipās, fīdus ille testis meus,
  apud vōs interfectus est, ubī̆ Sātānās habitat.
\vs%{Apc-2-14}
  Sed habeō contrā tē pauca, quod habēs istīc, quī Balaamī doctrīnam teneant,
  quī propter Bālācum docēbat Isrāēlītīs dētrīmentum adferre,
  quō inmolātā \unclassical{deastrīs} vēscerentur ac scortārentur;
\vs%{Apc-2-15}
  item habēs tū idem, quī Nicolaitārum doctrīnam tēneant, id quod ego ōdī.
\vs%{Apc-2-16}
  Conrige tē;
  aliōquīn adgrediar tē brevī et cum istīs ōris meī gladiō dēpugnābō.
\vs%{Apc-2-17}
  Quī habet aurem, audiat quid \SPĪRITUS dīcat \ECCLĒSIĪS:
  Victōrī dabō obcultō mannā vēscī eī‧que dabō calculum album
  et in calculō novum nōmen scrīptum, quod nūllus nōvit, nisi quī adcipit.

\vs%{Apc-2-18}
  Item angelō Thyatīrēnsis \ECCLĒSIAE scrībitō:

  Haec dīcit \DEĪ\ \FĪLIUS, quī oculōs habet igneae flammae simil|īs
  cujus‧que pedēs similēs sunt chalcolibanō:
\vs%{Apc-2-19}
  Novī tua facta et cāritātem et administrātiōnem et fidem
  et patientiam et postrēma plūra priōribus.
\vs%{Apc-2-20}
  Sed habeō contrā tē pauca, quod mulierem Jezabelem, quae sē vātem esse dīcit,
  docēre sinis servōs‧que meōs dēcipī et scortārī
  et \unclassical{deastrīs} inmolātā vēscī.
\vs%{Apc-2-21}
  Cui quum ego tempus dederim, quō ab inpudīcitiā suā recēderet,
  nōn vult recēdere.
\vs%{Apc-2-22}
  Ego vērō eam in lectum
  et cum eā adulterant|īs in magnam calamitātem prōsternam,
  nisi ā suīs operibus dēstiterint.
\vs%{Apc-2-23}
  Ejus‧que līberōs morte multābō et ecficiam,
  ut intellegant omnēs \ECCLĒSIAE eum esse mē quī rēnēs et corda scrūter,
  et vōs prō suīs quemque factīs remūnerābor.
\vs%{Apc-2-24}
  Vōbīs autem dīcō reliquīs‧que Thyatīrēnsibus,
  quotquot hanc disciplīnam nōn habent,
  et quī Sātānae calliditātēs nōn nōrunt, ut ajunt:
  nōn inpōnam vōbīs aliud onus;
\vs%{Apc-2-25}
  ac quod habētis, tenēte, dōnec veniam.
\vs%{Apc-2-26}
  Quī vērō vīcerit opera‧que mea ad extrēmum retinuerit,
  \bibleallusion{huic ego dabō} in \bibleallusion{gent|īs} eam potestātem,
\vs%{Apc-2-27}
 \bibleallusion{ut eōs virgā ferreā pāscat,
  quōmodo fictilia vāsa conteruntur},
  ut ego ā \PATRE meō adcēpī;
\vs%{Apc-2-28}
  eī‧que stēllam dabō mātūtīnam.

\vs%{Apc-2-29}
  Quī habet aurem, audiat quid \SPĪRITUS dīcat \ECCLĒSIĪS.
\stopbookchapter

\startbookchapter %Chapter 3
\startabstract
  Sequitur epistola quīnta ad pāstōrēs \ECCLĒSIĪS
  Sardēnsium, Philadelphīēnsium, et Lāodicēnsium,
  nē tepidī sint, sed glōriae \DEĪ prōmovendae operam dent.
\stopabstract

\vs%{Apc-3-1}
  Item angelō Sardēnsis \ECCLĒSIAE scrībitō:

  Haec dīcit, quī septem \DEĪ spīritūs et septem stēllās habet:
  Nōvī tua facta, quī, quum vīventis nōmen habeās, mortuus es.
\vs%{Apc-3-2}
  Praestā tē vigilantem et cētera moritūra cōnfirmā:
  nōn enim invēnī tua opera ergā \DEUM obficiōsa.
\vs%{Apc-3-3}
  Reminīscere igitur, quōmodo adcēperīs audīverīs‧que,
  et servā ad frūgem‧que redī;
  quod nisi vigilāveris, adoriar tē utī fūr neque sciēs, quā tē adoriar hōrā.
\vs%{Apc-3-4}
  Sed habēs pauca nōmina etiam Sardīs, quae sua vestīmenta nōn contāminārunt:
  iī mēcum in albīs vestibus ambulābunt, utpote dignī.
\vs%{Apc-3-5}
  Quī vīcerit, is albīs induētur vestibus nec ejus nōmen ex librō vītae dēlēbō
  et ejus nōmen cōram meō \PATRE cōram‧que ejus angelīs cōnfitēbor.
\vs%{Apc-3-6}
  Quī habet aurem, audiat quid \SPĪRITUS dīcat \ECCLĒSIĪS.

\vs%{Apc-3-7}
  Item angelō Philadelphiēnsis \ECCLĒSIAE scrībitō:

  Haec dīcit \SĀNCTUS, \VĒRUS, quī Dāvīdis \bibleallusion{clāv|im habet,
  quō aperiente nūllus claudit, claudente nūllus aperit}:
\vs%{Apc-3-8}
  Novī tua facta. Ego tibi apertam prōposuī portam, quam claudere nūllus queat
  Quoniam exiguās vīrīs habēns, meīs dictīs pāruistī
  neque meum nōmen negāstī,
\vs%{Apc-3-9}
  ego cōgam nōnnūllōs ex Sātānae catervā,
  --- quī sē Jūdaeōs esse dīcunt nec sunt, sed mentiuntur ---
  eōs, inquam, eō adigam,
  % supine in accusative
  utī tē ante pedēs tuōs venerātum veniant tē‧que mihi cārum esse intellegant.
\vs%{Apc-3-10}
  Quoniam meum patientiae sermōnem cōnservāstī,
  ego quoque tē adversus tentātiōnis hōram cōnservābō ēventūram tōtī orbī,
  quō tententur incolae terrārum.
\vs%{Apc-3-11}
  Ego brevī ventūrus sum; tenē quod habēs, nē quis tibi corōnam praeripiat.
\vs%{Apc-3-12}
  Quī vīcerit, eum ego columnam faciam in templō \DEĪ meī,
  quae deinceps inde nōn exībit;
  eī‧que \DEĪ meī nōmen īnscrībam et nōmen urbis \DEĪ meī, novae \JEROSOLYMAE,
  dē caelō ā \DEŌ meō dēscendentis, item‧que meum nōmen novum.

\vs%{Apc-3-13}
  Quī habet aurem, audiat quid \SPĪRITUS dīcat \ECCLĒSIĪS.

\vs%{Apc-3-14}
  Item angelō Lāodicēnsis \ECCLĒSIAE scrībitō:

  Haec dīcit \ĀMĒN, testis fīdus atque vērus, opificiī \DEĪ caput:
\vs%{Apc-3-15}
  Novī tua facta, quod neque frīgidus sīs neque fervidus.
  Utinam aut frīgidus essēs aut fervidus!
\vs%{Apc-3-16}
  Ergō quia tepidus es nequē frīgidus neque fervidus, ēvomam tē ex ōre meō.
\vs%{Apc-3-17}
  Quoniam
  \indirectquote{tē dīvitem dītātum‧que esse} dīcis
  \indirectquote{nec ūllā rē indigēre},
  neque scīs, tē esse aerumnōsum et miserum et pauperem et caecum et nūdum,
\vs%{Apc-3-18}
  suādeō tibi, ut ā mē aurum ign|ī candēns emās, quō dītēscās;
  et vestīmenta candida, quō induāre, nē adpāreat tuae nūditātis obscēnitās;
  et collȳriō inline tuōs oculōs, utī cernās.
\vs%{Apc-3-19}
  Ego, quōs amō, arguō et castīgō: quamobrem vehementior estō tē‧que conrige.
\vs%{Apc-3-20}
  Equidem prō foribus adstō atque pulsō.
  Sī quis, audītā vōce meā, jānuam aperuerit,
  ingrediar ad eum et cum eō cēnābō, et ipse mēcum.
\vs%{Apc-3-21}
  Quī vīcerit, eī dabō mēcum in meō sedēre soliō,
  ut ego quoque victor cōnsēdī cum meō \PATRE in ejus soliō.

\vs%{Apc-3-22}
  Quī habet aurem, audiat quid \SPĪRITUS dīcat \ECCLĒSIĪS}.
\stopbookchapter

\startbookchapter %Chapter 4
\startabstract
  Alia narrātur vīsiō, dīvīnae majestātis glōriam continēns,
  quae ā quattuor animālibus et vīgintī quattuor seniōribus celebrātur.
\stopabstract

\vs%{Apc-4-1}
Posteā animadvertī apertam in caelō portam, vōx‧que illa prīma
--- quam perinde acsī mēcum tuba loquerētur, audīveram --- dīxit:
\quote{Adscende hūc, et tibi ostendam, quae posthāc futūra sunt}.
\vs%{Apc-4-2}
Hīc ego continuō dīvīnitus adflātus,
videō positum in caelō solium et in soliō sedentem quemdam.
\vs%{Apc-4-3}
Quī sedēns adspectū similis erat jaspidī gemmae et sardiō;
solium circumdabat īrīs, adspectū smaragdum referēns.
\vs%{Apc-4-4}
Idem‧que solium circumstābant vīgintī quattuor solia,
in quibus soliīs vīdī sedent|īs vīgintī quattuor senātōrēs,
vestibus indūtōs candidīs, quī in capitibus corōnās habēbant aureās.
\vs%{Apc-4-5}
Ex soliō prōdībant fulgura et tonitrua et fragōrēs;
et septem facēs igneae, et ante solium ārdentēs, quae sunt septem \DEĪ spīritūs.
\vs%{Apc-4-6}
Ante solium erat mare vitreum, crystallī simile;
et in mediō soliī circum‧que solium quattuor animālia,
oculīs et ā fronte et ā tergō referta:
\vs%{Apc-4-7}
\bibleallusion{prīmum} animal erat \bibleallusion{leōnī} simile,
\bibleallusion{alterum} animal erat \bibleallusion{vitulō} simile,
\bibleallusion{tertium} animal
\bibleallusion{faciem} habēbat \bibleallusion{hominis},
\bibleallusion{quārtum} animal simile erat \bibleallusion{aquilae} volantī.
\vs%{Apc-4-8}
Haec quattuor animālia habēbant \bibleallusion{singula sēnās ālās} per circuitum,
et intus erant oculīs referta. Ea sine ūllā intermissiōne diēs et noctēs dīcunt:
\startlines\quote{%
  \bibleallusion{Sānctus, \SĀNCTUS, \SĀNCTUS\ \DOMINUS, \DEUS armipotēns},
  quī fuit, quī est, et quī ventūrus est!}.
\stoplines

\vs%{Apc-4-9}
Quum autem tribuunt animālia glōriam et honōrem et grātiārum āctiōnem
sedentī in soliō, vīventī in perennia saecula,
\vs%{Apc-4-10}
prōcumbunt vīgintī quattuor senātōrēs ante sedentem in soliō et adōrant vīventem
in perennia saecula abjiciunt‧que suās corōnās ante solium, dīcentēs:
\vs%{Apc-4-11}
\startlines\quote{%
  Dignus es, \DOMINE\ \added{et \DEUS noster},
  quī glōriam et honōrem et potestātem adcipiās:
  quoniam tū omnia condidistī,
  ea‧que tuā voluntāte et sunt et condita sunt}.
\stoplines
\stopbookchapter

\startbookchapter %Chapter 5
\startabstract
  Librum septem sigillīs obsignātum, quem nēmō aperīrī poterat,
  \AGNUS ille \DEĪ, dignus quī aperiat, omnium caelestium dēcantātur vōce.
\stopabstract

\vs%{Apc-5-1}
Item vīdī ad sedentis in soliō dexteram librum intus et extrā scrīptum,
septem sigillīs obsignātum.
\vs%{Apc-5-2}
Vīdī‧que validum angelum magnā vōce prōnūntiantem:
\quote{%
  Quis dignus est, quī librum aperiat et ejus sigilla resignet?}.
\vs%{Apc-5-3}
Sed nūllus nec in caelō nec in terrā nec sub terrā librum
aperīre aut īnspicere poterat.
\vs%{Apc-5-4} % infinitive for finite form
Ego vērō multum plōrāre, quod nūllus inventus esset dignus,
quī librum aperīret \obsolete{et legeret} aut īnspiceret.
\vs%{Apc-5-5}
Tum senātōrum quīdam mē adloquēns:
\quote{%
  Nē plōrā: ― inquit ― ēn vīcit leō, quī est ex tribū Jūdae, Dāvīdīca stirps,
  quī librum aperiat ejus‧que septem sigilla referet}.

\vs%{Apc-5-6}
Tum ego adspiciēns animadvertī in mediō soliī
et quattuor animālium in‧que mediō senātōrum adstantem \AGNUM tamquam mactātum,
habentem septem cornua septem‧que oculōs,
quī sunt septem \DEĪ spīritūs in omn|īs dīmissī terrās.
\vs%{Apc-5-7}
Is vēnit adcēpit‧que librum dē dexterā sedentis in soliō.
\vs%{Apc-5-8}
Postquam is librum adcēpit, quattuor animālia
et vīgintī quattuor senātōrēs prōcidere ante \AGNUM,
habentēs singulī citharās aureās‧que phialās subfīmentōrum plēnās,
quae sunt sānctōrum subplicātiōnēs,
\vs%{Apc-5-9} % infinitive for finite form
novum‧que carmen canere, dīcentēs:
\startlines\quote{%
  Dignus es, quī librum adcipiās
  ejus‧que sigilla resolvās:
  quoniam mactātus es nōs‧que \DEŌ sanguine tuō comparāstī
  ex omnī nātiōne, linguā, populō, gente;
\vs%{Apc-5-10}
  nōs‧que fēcistī \DEŌ nostrō rēgēs et sacerdōtēs
  rēgnātūrōs in terrā}.
\stoplines

\vs%{Apc-5-11}
Deinde adspiciēns audīvī multōrum angelōrum vōcem circum solium
et animālium ac senātōrum,
quōrum numerus erat distribūtus in mīliēns centēna mīlia,
\vs%{Apc-5-12}
quī magnā vōce dīcēbant:
\startlines\quote{%
  Dignus est mactātus \AGNUS,
  quī potestātem adcipiat
  et dīvitiās et sapientiam
  et vim et honōrem
  et glōriam et laudem}.
\stoplines

\vs%{Apc-5-13}
Tum omnem rērum nātūram,
quae et in caelō est et in terrā et sub terrā et in marī,
haec et quae in iīs sunt omnia, audīvī dīcere:
\startlines\quote{%
  Sedentī in soliō et \AGNŌ
  laus et honōs
  et glōria et imperium
  in sempiterna saecula}.
\stoplines

\vs%{Apc-5-14}
Et quattuor animālia dīxērunt: \quote{Āmēn};
et vīgintī quattuor senātōrēs prōcubuērunt
et \obsolete{vīventem in sempiterna saecula} venerātī sunt.
\stopbookchapter

\startbookchapter %Chapter 6
\startabstract
  Aperit \AGNUS prīmum librī sigillum,
  secundum, tertium, quārtum, quīntum et sextum;
  quibus apertīs, caedēs, famēs, pestis, sānctōrum querēlae, terraemōtus,
  et varia caelī prōdigia oriuntur.
\stopabstract

\vs%{Apc-6-1}
Deinde adspiciēns, quum aperuisset \AGNUS ūnum dē septem sigillīs,
audīvī ūnum dē quattuor animālibus dīcēns quasi vōce tonitruī:
\quote{Venī \obsolete{spectātum}}. % supine in accusative
\vs%{Apc-6-2}
Ego adspiciēns animadvertī equum album, cui quī īnsidēbat,
arcum habēbat, eī‧que data corōna est, is‧que prōdiit vincēns et utī vinceret.

\vs%{Apc-6-3}
Deinde ubī̆ sigillum secundum aperuit, audīvī secundum animal dīcēns:
\quote{Venī \obsolete{spectātum}}. % supine in accusative
\vs%{Apc-6-4}
Tum prōdiit alius equus rūfus, quem īnsidentī conmissum est,
ut ex terrīs auferret pācem, ita ut aliī aliōs necārent;
eī‧que datus est magnus gladius.

\vs%{Apc-6-5}
Postquam tertium aperuit sigillum, audīvī tertium animal dīcēns:
\quote{Venī \obsolete{spectātum}}. % supine in accusative
Hīc ego adspiciō nigrum equum, cui quī īnsidēbat, lībram habēbat in manū;
\vs%{Apc-6-6}
audīvī‧que vōcem in mediō quattuor animālium dīcentem:
\quote{%
  Choenix trīticī dēnāriō,
  et trēs choenicēs hordeī dēnāriō;
  et oleum et vīnum nē laeserīs}.

\vs%{Apc-6-7} % or dīcentis
Ubī̆ quārtum aperuit sigillum, audīvī vōcem quārtī animālis dīcentem:
\quote{Venī \obsolete{spectātum}}. % supine in accusative
\vs%{Apc-6-8}
Animadvertī‧que pallidum equum, cui quī īnsidēbat,
nōmen habet \MORS, quam simul comitātur \ORCUS;
hīs data potestās est interficiendī ad quārtam partem terrae
gladiō, famē, lētō, ferīs‧que terrestribus.

\vs%{Apc-6-9}
Deinde ubī̆ quīntum aperuit sigillum,
vīdī sub ārā animās necātōrum propter \DEĪ sermōnem propter‧que testimōnium,
quod habēbant.
\vs%{Apc-6-10}
Quī ingentī vōce sīc clāmābant:
\quote{%
  Quoū̆sque tandem, ō \DOMINE sāncte atque vēre,
  nōn ulcīsceris ac vindicās sanguinem nostrum dē terrae incolīs?}.

\vs%{Apc-6-11}
Sed iīs singulīs datae sunt albae stolae; iīs‧que dictum est,
utī quiēscerent adhūc paulisper,
dōnec plēnus esset numerus etiam cōnservōrum eōrum atque frātrum,
quī interficiendī essent ut et ipsī.

\vs%{Apc-6-12}
Deinde adspiciente mē, ubī̆ sextum sigillum aperuit, exstitit magnus terraemōtus,
factus‧que est sōl tam āter quam est saccus cilicīnus,
et lūna facta est sanguinea,
\vs%{Apc-6-13}
et caelī stēllae cecidērunt ad terram, sīcutī fīcus abjicit suās grossōs,
quum vehementī ventō quatitur,
\vs%{Apc-6-14}
et caelum recessit, perinde acsī scheda convolvātur,
omnēs‧que et montēs et īnsulae ex suīs locīs mōtae sunt,
\vs%{Apc-6-15}
et terrārum rēgēs et \unclassical{summātēs}
et dīvitēs et tribūnī et potentēs et omnēs ---
tum servī, tum līberī --- sēipsōs in specūs rūp||īs‧que montium abdere,
\vs%{Apc-6-16}
\bibleallusion{et montibus rūpibus‧que dīcere:}
\quote{%
  \bibleallusion{obruitē nōs} et \bibleallusion{abdite}
  ex cōnspectū sedentis in soliō et ab īrā \AGNĪ:
\vs%{Apc-6-17}
  quoniam vēnit magna diēs ejus īrae,
  quam quis possit subsistere?}.
\stopbookchapter

\startbookchapter %Chapter 7
\startabstract
  Grassantēs in orbe angelī inhibentur,
  dōnec ēlēctī \DOMINĪ ex omnibus tribubus signentur.
  Deinde eōrum, quī Chrīstī causā persecūtiōnem sustinuērunt,
  fēlīcitās et gaudium dēscrībitur.
\stopabstract

\vs%{Apc-7-1}
Posteā vīdī quattuor angelōs stant|īs ad quattuor terrae angulōs,
tenent|īs quattuor terrae ventōs,
nē ventus terram aut mare aut ūllam arborem adflāret.
\vs%{Apc-7-2}
Item vīdī alium angelum ab ortū sōlis adscendentem,
habentem vīventis \DEĪ sigillum;
quī ad quattuor angelōs
--- quibus, utī terram et mare laederent, conmissum erat ---
magnā vōce clāmāvit hīs verbīs:
\vs%{Apc-7-3}
\quote{%
  Nē laeditōte terram nec mare nec arborēs,
  dōnec \DEĪ nostrī servōs in eōrum frontibus cōnsignēmus}.
\vs%{Apc-7-4}
Audīvī autem cōnsignātōrum numerum, quī cōnsignātī erant:
centum quadrāgintā quattuor mīlia ex omnibus Isrāēlītārum tribubus: % tribūbus?
\vs%{Apc-7-5}
ex tribū Jūdae cōnsignāta erant duodecim mīlia,
ex Rūbēnis duodecim mīlia,
ex Gādī duodecim mīlia,
\vs%{Apc-7-6}
ex Āsēris duodecim mīlia,
ex Nephtālis duodecim mīlia, % or gr. Nephthālis
ex Manassis duodecim mīlia,
\vs%{Apc-7-7}
ex Simeōnis duodecim mīlia,
ex Lēvis duodecim mīlia,
ex Issachāris duodecim mīlia,
\vs%{Apc-7-8}
ex Zabūlōnis duodecim mīlia,
ex Jōsēphī duodecim mīlia,
ex Benjāmīnis duodecim mīlia.

\vs%{Apc-7-9}
Posteā animadvertī tantam turbam, ut eam numerāre nūllus posset,
ex omnibus gentibus, nātiōnibus, populīs, et linguīs;
quī ante solium et \AGNUM stābant,
stolīs amictī candidīs, palmās gestantēs in manibus,
\vs%{Apc-7-10}
et magnā vōce clāmantēs in hunc modum:
\quote{Salūs sedentī in soliō \DEĪ nostrī et \AGNŌ}.

\vs%{Apc-7-11}
Omnēs autem angelī senātōrēs‧que et quattuor animālia solium circumstantēs
prōcubuērunt ante solium prōnī {\DEUM}‧que venerātī sunt dīcentēs:
\vs%{Apc-7-12}
\startlines\quote{%
  Āmēn!
  Faustitās et glōria et sapientia
  et grātiārum āctiō et honōs
  et potentia et vīrēs,
  \DEŌ nostrō in perpetua saecula. Āmēn}.
\stoplines

\vs%{Apc-7-13}
Tum ūnus dē senātōribus sīc mē adfātus est:
\quote{Hī, stolīs albīs amictī, quīnam sunt et unde vēnērunt?}.
\vs%{Apc-7-14}
Cui ego:
\quote{Domine, tūtē scīs}.
Et ille:
\quote{%
  Hī sunt ― inquit ― quī ex magnā calamitāte veniunt
  stolās‧que suās \AGNĪ sanguine perluērunt ac dealbārunt.
\vs%{Apc-7-15}
  Proptereā sunt ante \DEĪ solium eum‧que diēs et noctēs in ejus templō colunt;
  apud eōs‧que dēget is, quī sedet in soliō.
\vs%{Apc-7-16}
  \bibleallusion{Nōn} amplius \bibleallusion{ēsurient},
  \bibleallusion{nec} amplius \bibleallusion{sitient},
  \bibleallusion{nec eōs sōl aut} ūllus \bibleallusion{aestus feriet}:
\vs%{Apc-7-17}
  quoniam \AGNUS, quī est inter solium,
  \bibleallusion{pāscet eōs et ad vīvōs aquae font|īs dūcet},
  \bibleallusion{absterget‧que \DEUS} ex eōrum oculīs
  \bibleallusion{omnem lacrimam}}.
\stopbookchapter

\startbookchapter %Chapter 8
\startabstract
  Septimō sigillō apertō, obferuntur sānctōrum precēs per subfītūs.
  Prōcēdunt septem angelī cum tubīs et, quattuor prīmīs clangentibus,
  ignis dēcidit, mare turbātur, aquae amārēscunt, et stēllās obtenebrantur.
\stopabstract

\vs%{Apc-8-1}
Ut autem septimum aperuit sigillum,
exstitit in caelō silentium ferē dīmidiam hōram.
\vs%{Apc-8-2}
Vīdī‧que septem angelōs, quī in \DEĪ cōnspectū adstiterant,
quibus septem tubae datae sunt.
\vs%{Apc-8-3} % I prefer Greek forms: thus, thuribulum, etc.
Alius item angelus vēnit et ad āram cōnstitit thūribulum habēns aureum,
eī‧que data sunt multa subfīmenta,
utī sānctōrum omnium subplicātiōnēs ad auream āram
ante solium positam obmovēret;
\vs%{Apc-8-4}
adscendit‧que fūmus subfīmentōrum subplicātiōnum sānctōrum
ex angelī manū in \DEĪ cōnspectum.
\vs%{Apc-8-5}
Deinde sūmpsit angelus thūribulum
et altāris ign|ī replēvit ad terram‧que dējēcit;
et exstitērunt fragōrēs et tonitrua et fulgura et terraemōtus.

\vs%{Apc-8-6}
Item‧que septem angelī, quī septem tubās habēbant, sēsē ad clangendum parārunt.

\vs%{Apc-8-7}
% The problem appearing from now on is that verb ‘clangere’
% has no perfect form, thus ‘clanxisse’ is a neologism
Quum‧que prīmus angelus \unclassical{clanxisset},
exstitit grandō et ignis permixta sanguin|ī.
Quibus ad terram dējectīs,
tertia pars arborum combusta est et omnēs virentēs herbae cōnflagrārunt.

\vs%{Apc-8-8}
Deinde ubī̆ secundus \unclassical{clanxit} angelus,
quīdam quasi mōns ingēns ign|ī flagrāns in mare dējectus est
cōnversa‧que est pars tertia maris in sanguinem;
\vs%{Apc-8-9}
et mortua est tertia pars rērum, quae in marī animātae sunt,
tertia‧que nāvium pars conrupta.

\vs%{Apc-8-10}
Utī tertius \unclassical{clanxit} angelus, cecidit dē caelō magna stēlla,
facis in mōrem ārdēns,
quae in tertiam fluviōrum partem et in aquārum font|īs dēlāpsa est ---
\vs%{Apc-8-11}
est autem stēllae nōmen \name{Absinthium} ---
et conversa est tertia pars in absinthium;
multī‧que hominum ab aquīs interiērunt, quod amārae factae essent.
\vs%{Apc-8-12}
Ad quārtī angelī clangōrem icta est sōlis tertia pars
et lūnae tertia pars et stēllārum tertia pars,
adeō ut, obfuscātā eōrum parte tertiā,
diēs tertiā parte suī nōn lūcēret itidem‧que nox. % suī???
\vs%{Apc-8-13}
Tum adspiciēns audīvī quemdam angelum volantem per medium caelī,
magnā vōce dīcentem:
\quote{%
  Vae, vae, vae terrae habitātōribus ā reliquīs tubārum sōnīs trium angelōrum,
  quī jam clangent!}.
\stopbookchapter

\startbookchapter %Chapter 9
\startabstract
  Quīntō angelō clangente, locustae vastātrīcēs prōdeunt.
  Sextus autem clangēns equitēs prōdūcit, hūmānum genus dēvastant|īs.
\stopabstract

\vs%{Apc-9-1}
Tum ubī̆ \unclassical{clanxit} quīntus angelus,
vīdī stēllam dē caelō dēlāpsam ad terram,
cui data clāvis est puteī \TARTARĪ.
\vs%{Apc-9-2}
Quae postquam \TARTAREUM puteum aperuit,
% orig. is probably mistaken, saying:
% Quā, postquam \TARTAREUM putem aperuit,
exhālāvit ex puteō tālis fūmus, quālis est magnae fornācis;
quō puteī fūmō sōl est et āēr obscūrātus.
\vs%{Apc-9-3}
Ex eō fūmō prōdiērunt in terram locustae,
quibus tālis est data vīs, quālem habent terrestrēs scorpiī:
\vs%{Apc-9-4}
Iīs‧que praeceptum est, nē terrestrēs herbās laederent
nēve ūllam viriditātem ūllam‧ve arborem,
sed sōlōs hominēs, quī \DEĪ signum in frontibus suīs nōn habērent.
\vs%{Apc-9-5}
Est autem mandātum iīs, nē eōs interficerent,
sed utī quīnque mēnsēs cruciārentur;
ita ut essent eōrum cruciātūs, quālis scorpiī cruciātus est,
quum hominem pupugit. % orig. pūnxit
\vs%{Apc-9-6}
Atque illā tempestāte quaerent hominēs mortem nec invenient;
cupient‧que mōrī, et eōs fugiet mors.

\vs%{Apc-9-7}
Erat autem locustārum ea speciēs, utī similēs essent equīs ad proelium parātīs,
erant‧que in eārum capitibus quaedam quasi corōnae aurī similēs,
vultūs erant utī vultūs hominum;
\vs%{Apc-9-8}
capillōs habēbant muliebrium simil|īs, dentēs erant quālēs sunt leōnum;
\vs%{Apc-9-9}
habēbant et lōrīcās, cujusmodī sunt ferreae lōrīcae,
erat‧que eārum ālārum sonitus quālis est curruum sonitus
multōrum equōrum ad pugnam currentium.
\vs%{Apc-9-10}
Caudās habēbant scorpiōnum simil|īs et in caudīs erant aculeī:
eārum potestās est hominēs laedere per quīnque mēnsēs.
\vs%{Apc-9-11}
Habent‧que sibi praefectum rēgem angelum \TARTARĪ,
cui nōmen est Hebraecē \name{Abaddōn}, Graecō autem sermōne \name{Apollȳōn}.

\vs%{Apc-9-12}
Ūna miseria praeteriit. Ventūrae sunt adhūc posthāc duae miseriae.

\vs%{Apc-9-13}
Deinde ubī̆ sextus \unclassical{clanxit} angelus,
audīvī ex quattuor altāris aureī cornibus ante \DEUM positīs % orig. positī
vōcem quamdam
\vs%{Apc-9-14}
dīcentem sextō angelō, quī tubam habēbat:
\quote{Solve quattuor angelōs ad magnum flūmen Euphrātēn conligātōs}.
\vs%{Apc-9-15}
Itaque solūtī sunt quattuor angelī,
quī ad hōram et diem et mēnsem et annum parātī erant
ad hominum tertiam partem necandam.
\vs%{Apc-9-16}
Erat autem exercitūs equitum numerus bis mīliēns centēna mīlia,
eōrum‧que numerum audīvī.
\vs%{Apc-9-17}
Atque ita vīsus sum vidēre equōs et in iīs equitant|īs,
quī lōrīcās habēbant igneās et hyacinthinās ac sulpureās;
equōrum capita erant leōnīnōrum capitum similia
ex‧que eōrum ōribus mānābat ignis et fūmus et sulpur.
\vs%{Apc-9-18}
Ab hīs tribus clādibus interfecta est hominum pars tertia,
vidēlicet ign|ī et fūmō et sulpure, quae prōdībant ex eōrum ōribus.
\vs%{Apc-9-19}
Potestās enim eōrum in ōre eōrum erat:
sī̆quidem eōrum caudae erant anguium similēs capita habentēs, quibus nocērent.

\vs%{Apc-9-20}
Reliquī hominum, quī hīs calamitātibus obcīsī nōn sunt,
nōn dēstitērunt ā suīs āctiōnibus, nē daemonia venerārentur et
\bibleallusion{%
  simulācra aurea, argentea, aerea, lapidea, lignea, quae nec vidēre}
possunt
\bibleallusion{nec audīre nec ingredī},
\vs%{Apc-9-21}
nec sua homicīdia nec venēficia nec stupra nec fūrta omīsērunt.
\stopbookchapter

\startbookchapter %Chapter 10
\startabstract
  Alius adpāret angelus amictus nūbe, libellum apertum tenēns.
  Inclāmat et Jōhannī, ōrāculī jussū, dēvorandum trādit.
\stopabstract

\vs%{Apc-10-1}
Deinde vīdī alium angelum validum dē caelō dēscendentem indūtum nūbe,
cujus capītī īrīs incumbēbat, faciēs erat sōlī similis,
pedēs ut igneae columnae.
\vs%{Apc-10-2}
Habēbat autem in manū libellum apertum,
posuit‧que pedem suum dextrum in marī, sinistrum in terrā,
\vs%{Apc-10-3}
et magnā vōce clāmāvit, quōmodo rugit leō.
Postquam clāmāvit, locūta sunt septem tonitrua suās vōcēs.
\vs%{Apc-10-4}
Quibus septem tonitruīs suās vōcēs locūtīs, quum ego scrīptūrus essem,
audīvī vōcem dīcentem mihi:
\quote{Cōnsignā, quae locūta sunt septem tonitrua, et ea nē scrīpserīs}.

\vs%{Apc-10-5}
Tum angelus, quem in mare et in terrā stantem vīderam,
\bibleallusion{%
  sublāta in caelum suā manū,
\vs%{Apc-10-6}
  jūrāvit per \VĪVENTEM in perpetua saecula},
quī caelum et caelestia, terram et terrestria, mare et marīna condidit,
\indirectquote{futūrum esse, utī tempus deinceps nōn sit,
\vs%{Apc-10-7}
vidēlicet diēbus vōcis septimī angelī, quum \unclassical{clanxerit},
perāctum‧que fuerit arcānum \DEĪ, utī suīs servīs vātibus nūntiāvit}.

\vs%{Apc-10-8}
Atque eadem vōx, quam dē caelō audīveram, rūrsus mēcum locūta est in hunc modum:
\quote{%
  Ī, sūme libellum apertum, quī est in manū angelī in mare terrā‧que stantis}.
\vs%{Apc-10-9}
Tum ego angelum adgressus:
\quote{Dā mihi libellum}, inquam.
Et ille:
\quote{%
  Adcipe ― inquit ― eum‧que dēgluttī; % dēgluttīre according to the OLD
  et tibi ventrem amāritūdine torquēbit,
  quamquam in ōre tuō erit tam dulcis quam est mel}.
\vs%{Apc-10-10}
Hīc ego sūmptum dē angelī manū libellum dēgluttīvī,
quod mihi in ōre quidem tam dulce fuit, quam est mel;
% Corrected a mistake, originally:
% quī mihi in ōre quidem tam dulcis fuit, quam est mel;
sed ubī̆ id comēdī, \unclassical{amāruit} mihi venter.
\vs%{Apc-10-11}
Et ille mē adloquēns:
\quote{%
  Vāticinandum tibi rūrsus est ― inquit ―
  % using super here sound unclassical
  super populīs et gentibus et linguīs rēgibus‧que multīs}.
\stopbookchapter

\startbookchapter %Chapter 11
\startabstract
  Templum mētīrī jubētur.
  Dominus testīs duōs excitat, quōs ā \BĒSTIĀ laniātōs nēmō terrae mandat;
  \DEUS autem in vītam revocātōs ad caelum ēvocat.
  Terrentur impiī;
  et septimī angelī tuba, resubrēctiōnis et jūdiciī nūntia dēscrībuntur.
  % orig. dēscrībitur, most likely a mistake
\stopabstract

\vs%{Apc-11-1}
Tum datus est mihi calamus virgae similis. \obsolete{Et angelus stāns:}
\quote{%
  Surge ― inquit ― et \DEĪ mētīre templum āram‧que et adōrant|īs in eō.
\vs%{Apc-11-2}
  Ac exterius ātrium, templī forīs, exclūde neque id mētīre,
  quippe datum extrāneīs,
  quī sānctam urbem conculcābunt mēnsēs quadrāgintā duōs.
\vs%{Apc-11-3}
  Mandābō‧que duōbus meīs testibus,
  utī vāticinentur diēs mīlle ducentōs sexāgintā, centōnibus amictī}.
\vs%{Apc-11-4}
\bibleallusion{%
  Hī sunt olīvae duae duo‧que candēlābra in cōnspectū \DEĪ terrae stantia}.
\vs%{Apc-11-5}
Quodsī quis eōs laedere vult, mānat ex eōrum ōre ignis,
quī eōrum hostīs cōnficit;
et sī quis eōs laedere vult, sīc interficiendus est.
\vs%{Apc-11-6}
Hī potestātem habent claudendī caelī,
nē pluat imber tempore suae vāticinātiōnis;
habent‧que in aquās potestātem eās in sanguinem convertendī,
et terrae quamlibet clādem īnflīgendī, quotiēns velint.
\vs%{Apc-11-7}
Quum autem suum testimōnium perēgerint, \BĒSTIA ex \TARTARŌ adscendet;
quae eōs adgressa, duello vincet ac interficiet.
\vs%{Apc-11-8}
Jacēbunt‧que eōrum corpora in urbis magnae forō,
quae vocātur spīrituāliter Sodoma et Aegyptus,
ubī̆ etiam \DOMINUS noster crucifīxus est;
\vs%{Apc-11-9}
et adspicient hominēs dē populīs et nātiōnibus, linguīs‧que et gentibus
eōrum corpora tr|īs diēs et dīmidium,
nec sinent eōrum corpora monument|īs mandārī.
\vs%{Apc-11-10}
Gaudēbunt autem terrae incolae super iīs, % sounds unclassical
laetī‧que dōna mittent inter sēsē,
quod hī duo vātēs cruciāverint incolās terrae.

\vs%{Apc-11-11}
Vērum post diēs tr|īs et dīmidium, vītālī spīritū ā \DEŌ in eōs ingressō,
cōnstitērunt in pedēs suōs, magnō timōre conreptīs iīs, quī eōs spectābant.
\vs%{Apc-11-12}
Audīvī‧que vōcem magnam dē caelō dīcentem iīs:
\quote{Adscendite hūc};
atque illī in caelum in nūbe adscendērunt, spectantibus eōs hostibus eōrum.
\vs%{Apc-11-13}
Ac eādem hōrā tantus exstitit terraemōtus,
utī conlāpsa sit urbis pars decima,
obcīsīs in eō mōtū hūmānōrum capitum septem mīlibus.
Reliquī timōre perculsī, tribuērunt caelestī \DEŌ glōriam.

\vs%{Apc-11-14}
Miseria secunda praeteriit, ecce tertia miseria mox aderit.

\vs%{Apc-11-15}
Deinde septimus clanxit angelus, exstitērunt‧que magnae vōcēs in caelō dīcentēs:
\quote{%
  Facta sunt mundī rēgna \DOMINĪ nostrī ejus‧que Chrīstī,
  quī rēgnābit in omnem perpetuitātem}.

\vs%{Apc-11-16}
Et vīgintī quattuor senātōrēs, ante \DEUM suīs in soliīs sedentēs,
prōnī prōcubuērunt {\DEUM}‧que venerātī sunt, dīcentēs:
\vs%{Apc-11-17}
\startlines\quote{%
  Agimus tibi grātiās,
  \DOMINE, \DEE omnipotēns,
  quī es, quī fuistī\obsolete{, quī‧que futūrus es},
  quod magnam tuam potestātem adeptus rēgnum iniistī.
\vs%{Apc-11-18}
  Et gentēs īrātae sunt,
  et vēnit īra tua tempus‧que, quō mortuī jūdicentur,
  et mercēs dētur tuīs servīs vātibus et sānctīs,
  tuum‧que nōmen reverentibus --- tum parvīs, tum magnīs
  --- et conrumpantur conruptōrēs terrae}.
\stoplines

\vs%{Apc-11-19}
Tum, apertō \DEĪ templō in caelō,
vīsa est \ARCA\ \FOEDERIS ejus in ejus templō;
exstitērunt‧que fulgura et sonitūs \obsolete{et tonitrua}
et terraemōtus et magna grandō.
\stopbookchapter

\startbookchapter %Chapter 12
\startabstract
  Mulierī parturientis signum adpāret, cujus fīliō īnsidiātur \DRACŌ,
  quī ā Michāēle dēvictus, dējicitur;
  et quō magis vincendus exclūditur,
  eō ācrius suās nōn dēfīnit multiplicāre versūtiās.
\stopabstract

\vs%{Apc-12-1}
Vīsum‧que est ostentum ingēns in caelō:
mulier amicta sōle, ejus pedibus subjecta lūna, % should it be abl. abs.?
capitī autem imposita duodecim stēllārum corōna;
\vs%{Apc-12-2}
ea in aluō gestāns vōciferābātur, et parturiēns pariendī dolōribus cruciābātur.
\vs%{Apc-12-3}
Item cōnspectum est aliud ostentum in caelō:
\DRACŌ erat ingēns, rūfus, habēns septem capita et decem cornua,
et in captibus diadēmata septem;
\vs%{Apc-12-4}
cujus cauda tertiam caelestium sīderum partem dētrāxit ad terram‧que dējēcit.
Is \DRACŌ stetit ante mulierem, quae paritūra erat, utī,
quum ea suum \FĪLIUM peperisset, ipse eum dēvorāret. % should it be capitalized?
\vs%{Apc-12-5}
Peperit autem \FĪLIUM marem,
quī \bibleallusion{gent|īs} omn|īs \bibleallusion{ferreō scēptrō rēctūrus est};
quī ejus \FĪLIUS raptus est ad \DEUM ejus‧que solium.
\vs%{Apc-12-6}
Mulier autem fūgit in sōlitūdinem, ubī̆ locum habet ā \DEŌ parātum,
ut illīc alātur diēs mīlle ducentōs sexāgintā.

\vs%{Apc-12-7}
Factum est autem in caelō proelium:
Michāēl et ejus angelī adversus \DRACŌNEM pugnābant.
\vs%{Apc-12-8}
Et \DRACŌ ejus‧que angelī pugnārunt, et vīctī sunt,
nec ūllus exstitit jam eōrum locus in caelō.
\vs%{Apc-12-9}
Itaque dējectus est \DRACŌ ille magnus, serpēns antīquus,
quī \DIABOLUS et Sātānās adpellātur, quī tōtum orbem dēcipit;
dējectus est, inquam, ad terram, ejus‧que angelī dējēctī sunt.
\vs%{Apc-12-10}
Deinde magnam vōcem audīvī dīcentem:
% Versification differs from that of Nova Vulgata
\startlines\quote{%
  In caelō nunc facta est salūs et potestās et \RĒGNUM\ \DEĪ nostrī
  et potentia Chrīstī ejus,
  quoniam dējectus est frātrum nostrōrum adcūsātor, % nostrum?
  quī eōs apud \DEUM nostrum diēs ac noctēs adcūsābat.
\vs%{Apc-12-11}
  Eum illī \AGNĪ sanguine
  suī‧que testimōniī sermōne vīcērunt;
  animam‧que suam ū̆sque adeō nōn amārunt,
  utī mortem obpetiērunt.
\vs%{Apc-12-12}
  Proptereā laetāminī caelī et caelicolae.
  Vae terrae et maris incolīs!
  Quandōquidem dēscendit \DIABOLUS ad vōs, magnō furōre percitus,
  sciēns habēre sē breve tempus}.
\stoplines

\vs%{Apc-12-13}
Ubī̆ \DRACŌ sē ad terram dējectum vīdit,
mulierem persecūtus est, quae marem pepererat.
\vs%{Apc-12-14}
Sed mulierī datae sunt ālae duae ingentis aquilae,
utī sōlitūdinem in locum suum āvolāret;
illīc alenda tempus et tempora et dīmidium temporis ad vītandum serpentem.
\vs%{Apc-12-15}
Jēcit autem serpēns post mulierem ex ōre suō aquam flūminis īnstar,
quō eam flūmine submergeret.
\vs%{Apc-12-16}
Vērum terra mulierī subcurrit,
apertīs‧que suīs faucibus, flūmen hausit, quod \DRACŌ ōre suō ējaculātus erat.

\vs%{Apc-12-17}
% supine in accusative; orig. īvit
Tum īrātus mulierī \DRACŌ, iit inlātum duellum reliquīs ejus sēminis,
quī \DEĪ praeceptīs oboedīrent et Jēsū Chrīstī testimōnium habērent.
\vs%{Apc-12-18}
Ego in maris harēnā stābam.
% Originally, the last sentence was
% Ego in maris harēnā stāns,
% and was the first part of the next sentence
\stopbookchapter

\startbookchapter %Chapter 13
\startabstract
  Dēscrībitur \BĒSTIA illa multiceps,
  quae maximam mundī partem in īdōlolatrīam sēdūcit,
  quae ab aliā \BĒSTIĀ adscendente cōnfirmātur.
\stopabstract

\vs%{Apc-13-1}
Vīdī adscendentem ex marī \BĒSTIAM, habentem capita septem et cornua decem,
et in cornibus diadēmata decem, et super capita nōmen contumēliōsum.
\vs%{Apc-13-2} % gr. πᾰ́ρδᾰλῐς
Erat autem \BĒSTIA, quam vīdī, pardalī similis,
pedibus ursīnīs, ōre leōnīnō similī.
Eī \DRACŌ suās vīrīs suum‧que solium et magnam dedit potentiam.
\vs%{Apc-13-3}
Deinde vīdī ūnum ejus capitum paene morte exstīnctum,
sed cujus lētiferum vulnus sānātum est.

Itaque tantae fuit per tōtam terram admīrātiōnī \BĒSTIA,
\vs%{Apc-13-4}
utī \DRACŌNEM venerārentur, quī potentiam dedisset \BĒSTIAE,
et \BĒSTIAM item venerārentur dīcentēs:
\quote{Quis pār est \BĒSTIAE? Quis cum eā pugnāre possit?}.

\vs%{Apc-13-5}
Eī autem datum est ōs magna et īnfanda loquēns,
et potestās agendī mēnsēs quadrāgintā duōs,
\vs%{Apc-13-6}
ea‧que ōs suum aperuit in contumēliam adversus \DEUM,
in ejus nōmen in‧que tabernāculum et caelicolās,
contumēliōsē loquēns.
\vs%{Apc-13-7}
Item eī datum est duellum cum sānctīs gerere eōs‧que vincere,
eīdem‧que potestās est data in omn|īs nātiōnēs et linguās atque gent|īs,
\vs%{Apc-13-8}
Ita ut eam adōrātūrī sint omnēs terrārum incolae,
quōrum nōmina ab orbe conditō scrīpta nōn sunt in vītae librō \AGNĪ mactātī.
\vs%{Apc-13-9}
Sī quis aurem habet, audiat:
\startlines
\vs%{Apc-13-10}
  \bibleallusion{sī quis captīvōs abigit},
  \bibleallusion{in captīvitātem} abit;
  \bibleallusion{sī quis gladiō obcīderit},
  eum \bibleallusion{gladiō} oportet obcīdī.
\stoplines
Hic est patientia fidēs‧que sānctōrum.

\vs%{Apc-13-11}
Item vīdī aliam \BĒSTIAM ex terrā adscendentem,
quae duo cornua habēbat agnī similia, et loquēbātur utī \DRACŌ.
\vs%{Apc-13-12}
Ea omnem priōris \BĒSTIAE potestātem exercēbat in ejus cōnspectū;
ecficiēbat‧que, utī terra et ejus incolae priōrem \BĒSTIAM,
cujus mortiferum vulnus erat sānātum, venerārentur.
\vs%{Apc-13-13}
Magna‧que ostenta faciēbat,
adeō ut etiam ign|im dē caelō ad terram dēvocāret in hominum cōnspectū,
\vs%{Apc-13-14}
et terrae incolās ostentīs, quae eī in \BĒSTIAE cōnspectū facere datum erat,
fallēbat, jubēns terrae incolīs, utī simulācrum facerent \BĒSTIAE,
quae gladiō vulnerāta revaluerat.
\vs%{Apc-13-15}
Atque eī datum est spīritū dōnāre simulācrum \BĒSTIAE,
adeō utī loquerētur simulācrum \BĒSTIAE;
ecficeret‧que, utī quīcumque nōn venerārentur \BĒSTIAE simulācrum obcīderentur.
\vs%{Apc-13-16}
Atque eō adigēbat omn|īs:
parvōs juxtā ac magnōs, dīvitēs ac pauperēs, līberōs ac servōs,
ut eōrum manuī dexterae aut frontibus notam inūreret;
\vs%{Apc-13-17}
nēve cui emere licēret aut vēndere,
nisi quī notam aut nōmen habēret \BĒSTIAE aut ejus nōminis numerum.
\vs%{Apc-13-18}
% orig. Hic est acūmen, which sounds bad
Hoc est acūmen: quī habet ingenium, computet \BĒSTIAE numerum.
Numerus enim hominis est: est‧que ejus numerus sescentī sexāgintā sex. %666
\stopbookchapter

\startbookchapter %Chapter 14
\startabstract
  Agnō stante super montem Siōn ūnā cum castīs suīs cultōribus,
  angelus ūnus \ĒVANGELIUM nūntiat;
  alius Babylōnis cāsum praedīcit;
  tertius \BĒSTIAM vītārī monet.
  Vōx ē caelō beātōs illōs prōnūntiat, quī \DOMINĪ causā moriuntur.
  Mittitur posteā falx \DOMINĪ in messem et in vīndēmiam.
\stopabstract

\vs%{Apc-14-1}
Deinde animadvertī \AGNUM stantem in Siōne monte,
et cum eō centum quadrāgintā quattuor mīlia,
\PATRIS ejus nōmen in suīs frontibus scrīptum habentia.
\vs%{Apc-14-2}
Audīvī‧que dē caelō sonitum,
quālis multārum aquārum ingentis‧que tonitruī sonitus est;
itemque vōcem audīvī citharoedōrum, citharīs suīs canentium,
\vs%{Apc-14-3}
quī velutī novum carmen canēbant ante solium
et ante quattuor animālia ac senātōrēs,
quod carmen nūllus discere poterat,
% orig. redēmptī, most likely a mistake,
% as after mīlia a plural noun in genetive is expected
nisi illa centum quadrāgintā quattuor mīlia redēmptōrum ā terrā.
\vs%{Apc-14-4}
Hī sunt, quī cum sēminis nōn sunt pollūtī: sunt enim virginēs.
Hī sunt, quī \AGNUM, quōcumque vadat, sequuntur.
Hī ex hominibus redēmptī sunt, prīmitiae \DEŌ et \AGNŌ;
\vs%{Apc-14-5}
quōrum in ōre nūlla reperta fraus est: sunt enim inculpātī cōram \DEĪ soliō.

\vs%{Apc-14-6}
Vīdī item alium angelum mediō caelō volantem, \ĒVANGELIUM aeternum habentem,
quod terrae incolās gent|īs‧que et nātiōnēs et linguās et populōs omn|īs docēret,
\vs%{Apc-14-7}
dīcentem magnā vōce:
\quote{%
  Timēte \DEUM eī‧que glōriam tribuite: nam vēnit hōra ejus jūdiciī;
  et adōrāte caelī, terrae, maris, et fontium aquārum conditōrem}.

\vs%{Apc-14-8}
Item alius angelus secūtus est, dīcēns:
\quote{%
  Cecidit, cecidit Babylōn, urbs illa magna,
  quoniam saevō impudīcitiae suae vīnō gent|īs omn|īs pōtiōnāvit}.
\vs%{Apc-14-9}
Item tertius angelus eōs secūtus est, magnā vōce dīcēns:
\quote{%
  Sī quis \BĒSTIAM et ejus imāginem venerātur,
  notam‧que in frontem manum‧ve suam admittit,
\vs%{Apc-14-10}
  is etiam saevum \DEĪ vīnum mixtum merum‧que pōtābit in ejus īrae pōculō,
  et ign|ī ac sulpure torquēbitur in sānctōrum angelōrum {\AGNĪ}‧que cōnspectū.
\vs%{Apc-14-11}
  Eōrum‧que tormentī fūmus exhālābit in omnem perpetuitātem
  nec ūllam diē nocte‧ve requiētem habēbunt, % requiem is fine as well
  quī \BĒSTIAM et ejus imāginem venerantur,
  aut sī quis ejus nōminis notam adcipit}.

\vs%{Apc-14-12}
Hic est sānctōrum patientia, hī sunt, quī \DEĪ praecepta, Jēsū‧que fidem tuentur.
\vs%{Apc-14-13}
Tum vōcem audīvī dē caelō dīcentem mihi:
\quote{%
  Scrībe: Beātī mortuī, quī in \DOMINŌ moriuntur.
  Iī deinceps ― adfirmat \SPĪRITUS ― ā suīs requiēscent labōribus
  et sua eōs comitantur opera}.
\vs%{Apc-14-14}
Praetereā animadvertī nūbem candidam,
et nūbī īnsidentem quemdam homine \FĪLIŌ similem,
habentem in capite corōnam auream et in manū falcem acūtam.
\vs%{Apc-14-15}
Tum alius angelus exiit ex fānō, magnā vōce clāmāns ad eum, quī nūbī īnsidēbat:
\quote{%
  Inmitte falcem tuam et mete:
  nam tibi vēnit hōra metendī, quandōquidem āruit terrae seges}.
\vs%{Apc-14-16}
Itaque inmīsit, quī in nūbe sedēbat, falcem suam in terram; messa‧que terra est.

\vs%{Apc-14-17}
Item alius prōdiit ex fānō, quod in caelō erat,
angelus habēns et ipse falcem acūtam.
\vs%{Apc-14-18}
Et alius prōdiit angelus ex ārā, ignipotēns, quī magnā vōce clāmāns,
eum, quī falcem habēbat, sīc adlocūtus est:
\quote{%
  Injice tuam istam falcem acūtam, et terrae ūvās vīndēmiā,
  quoniam ejus mātūruērunt racēmī}.
\vs%{Apc-14-19}
Ita angelus ille, suā falce in terram injectā, terrae vīneam vīndēmiāvit
et ūvās in ingēns saevitiae \DEĪ torcular conjēcit.
\vs%{Apc-14-20}
Quō torculārī extrā urbem calcātō,
mānāvit ex torculārī sanguis ū̆sque ad frēnōs equōrum
per mīlle sescenta stadia.
\stopbookchapter

\startbookchapter %Chapter 15
\startabstract
  Adpārent septem angelī, habentēs plāgās septem ultimās.
  Victōrēs \BĒSTIAE\ \DEUM laudant.
  Angelīs septem, septem dantur dīvīnī furōris phialae.
\stopabstract

\vs%{Apc-15-1}
Vīdī et aliud in caelō ostentum ingēns ac mīrābile:
angelōs septem habent|īs septem clād||īs ultimās,
in quās \DEĪ saevitia absūmātur.

\vs%{Apc-15-2}
Item‧que vīdī quasi mare vitreum ign|ī mixtum;
% sound unclassical (should it be bēstiae, simulācrī, etc.?)
et ex \BĒSTIĀ ejus‧que simulācrō et notā et nōminis numerō, victōrēs,
stant|īs super mare vitreum, dīvīnās habent|īs citharās,
\vs%{Apc-15-3}
et Mōysis, \DEĪ servī, carmen {\AGNĪ}‧que carmen ita cantant|īs:
\startlines\quote{%
  Magna et mīrābilia sunt opera tua,
  \DOMINE, \DEE omnipotēns;
  jūsta vēra‧que sunt īnstitūta tua,
  sānctōrum \RĒX!
\vs%{Apc-15-4}
  Quis tē nōn metuat, \DOMINE,
  tuum‧que nōmen celebret?
  Nam tū sōlus \SĀNCTUS,
  nam tē gentēs omnēs cōram
  adōrātum venient, % supine in accusative
  quod tua jūra patefacta sint}.
\stoplines

\vs%{Apc-15-5}
Posteā vīdī fānum aperīrī \unclassical{ōrāculāris} tabernāculī in caelō,
\vs%{Apc-15-6}
et ex fānō prōdīre septem angelōs, septem ultimās clād||īs habent|īs,
pūrō nitidō‧que vestītōs līnō, et pectora cingulīs aureīs cīnctōs.
\vs%{Apc-15-7}
Hīs septem angelīs dedit ūnum quattuor animālium
septem phialās aureās,
plēnās saevitiā \DEĪ vīventis in omnem saeculōrum seriem;
\vs%{Apc-15-8}
est‧que fānum tantō fūmō prae \DEĪ splendōre atque potentiā complētum,
utī nūllus posset in fānum intrāre,
dōnec septem angelōrum septem clādēs perāctae forent.
\stopbookchapter

\startbookchapter %Chapter 16
\startabstract
  Ecfundunt angelī septem conmissās sibi furōris dīvīnī phialās,
  quibus ecfusīs variae plāgae in orbe terrārum exoriuntur
  ad terrendōs improbōs et magnae urbis incolās.
\stopabstract

\vs%{Apc-16-1}
Tum magnam ex fānō vōcem audīvī, dīcentem septem angelīs:
\quote{Īte, ecfundite phialās \DEĪ saevitiae in terram}.

\vs%{Apc-16-2}
Itaque prīmus abiit, et phialam suam in terram ecfūdit;
dīrō‧que et taetrō vulnere adfectī sunt hominēs
\BĒSTIAE notam habentēs et ejus simulācrum colentēs.

\vs%{Apc-16-3}
Deinde secundus angelus ecfūdit suam phialam in mare;
id‧que in cruōrem, quālis mortuōrum esse solet, mūtātum est,
omnēs‧que animantēs mortuae sunt in marī.

\vs%{Apc-16-4}
Item tertius angelus phialam suam in fluviōs et aquārum font|īs ecfūdit;
iī‧que mūtātī sunt in sanguinem.
\vs%{Apc-16-5}
Tum aquārum angelum audīvī dīcentem:
\quote{%
  Jūstus es, \DOMINE, quī es quī‧que fuistī quī‧que \SĀNCTUS es,
  quī haec ita facienda jūdicāverīs,
\vs%{Apc-16-6}
  utī, quī sānctōrum vātum‧que sanguinem ecfundēbant,
  iīs sanguinem pōtandum dederīs, utī dignī sunt!}
\vs%{Apc-16-7}
Deinde alium audīvī ex ārā dīcentem:
\quote{%
  Etiam, \DOMINE, \DEE omnipotēns:
  vēra jūsta‧que sunt tua jūdicia!}.

\vs%{Apc-16-8}
Item quārtus angelus ecfūdit phialam suam in sōlem;
eī‧que mandātum est, ut hominēs ign|ī ūreret.
\vs%{Apc-16-9}
Itaque magnō ārdōre ustī sunt hominēs;
{\DEĪ}‧que nōminī contumēliam dīxērunt, quī habēret in eās clādīs potestātem,
nec ad frūgem rediērunt, ut eī glōriam tribuerent.

\vs%{Apc-16-10}
Deinde quīntus angelus phialam suam ecfūdit in \BĒSTIAE solium;
adeō‧que obscūrātum est ejus rēgnum,
ut ipsī suās prae dolōre linguās mordērent;
\vs%{Apc-16-11}
caelī‧que \DEŌ prae dolōribus vulneribus‧que suīs convīciārentur,
nec ā suīs āctiōnibus dēsisterent.

\vs%{Apc-16-12}
Deinde sextus angelus phialam suam ecfūdit in magnum flūmen Euphrātēn;
et āruit ejus aqua, utī parārētur via rēgum, quī essent ā sōlis ortū.
\vs%{Apc-16-13}
Item vīdī ex ōre \DRACŌNIS et ex ōre \BĒSTIAE
et ex ōre falsī vātis tr|īs spīritūs inmundōs, rānārum simil|īs;
\vs%{Apc-16-14}
sunt autem daemonum spīritūs, quī spīritūs ecficiunt,
ut edantur ostenta ad terrārum et tōtīus orbis rēgēs,
quō eōs congregent in proelium illī̆us magnae diēī omnipotentis \DEĪ.

\vs%{Apc-16-15}
Ventūrus sum utī fūr: fēlīx, quī vigilat, sua‧que vestīmenta servat,
nē nūdus incēdat, ejus‧que obscēna cernantur.

\vs%{Apc-16-16}
% Ἁρμαγεδών, G717
Congregāvit autem eōs in locum, quī Hebraecē vocātur \name{Harmageddōn}.

\vs%{Apc-16-17}
Deinde septimus angelus phialam suam in āerem ecfūdit;
et prōdiit magna vōx ex caelī fānō, ex soliō, dīcēns:
\quote{Factum est!}.
\vs%{Apc-16-18}
Exstitērunt‧que sonitūs et tonitrua ac fulgura;
tantus‧que terraemōtus factus est, utī post hominēs in terrīs nātōs tantus,
tam ingēns terraemōtus nūllus exstiterit.
\vs%{Apc-16-19}
Dīvīsa‧que est urbs magna in tr|īs part|īs, et gentium urbēs conruērunt:
et Babylōn magna \DEŌ vēnit in mentem,
ut eī pōculum daret vīnī suae saevae īrācundiae.
\vs%{Apc-16-20}
Et omnēs īnsulae fūgērunt, et montēs nusquam exstitērunt.
\vs%{Apc-16-21}
Et ingēns grandō ad talentī magnitūdinem dē caelō dēlāta est in hominēs,
fēcērunt‧que hominēs convīcium \DEŌ ob grandinis clādem:
tam gravis erat illa clādēs grandinis.
\stopbookchapter

\startbookchapter %Chapter 17
\startabstract
  Dēscrībitur \MERETRĪX illa magna, cum quā rēgēs terrae scortantur,
  ēbria piōrum sanguine, et \BĒSTIA, quae portat illam,
  mystērium utrī̆usque et exitium, postrēmō \AGNĪ victōria.
\stopabstract

\vs%{Apc-17-1}
Vēnit autem ūnus septem angelōrum, septem phialās habentium,
quī mē sīc adlocūtus est:
\quote{%
  Ades, ostendam tibi subplicium \MERETRĪCIS magnae, sedentis in multīs aquīs,
\vs%{Apc-17-2}
  cum quā scortātī sunt rēgēs terrae,
  cujus‧que impudīcitiae vīnō inēbriātī sunt incolae terrae}.
\vs%{Apc-17-3}
Deinde mē dīvīnō adflātū abstulit in sōlitūdinem,
ubī̆ mulierem vīdī īnsidentem \BĒLUAE coccinae,
nefāriīs nōminibus plēnae, habentī septem capita et decem cornua.
\vs%{Apc-17-4}
Erat ea mulier indūta purpurā et coccinō,
aurō‧que deaurāta et gemmīs ac margarītīs,
habēns aureum pōculum in suā manū,
flāgitiīs et turpitūdine ejus stuprōrum refertum;
\vs%{Apc-17-5}
et īnscrīptum suae frontī nōmen, mystērium:
\quote{Babylōn magna, meretrīcum flāgitiōrum‧que terrae māter}.

\vs%{Apc-17-6}
Eam ego mulierem vīdī sanguine sānctōrum testium‧que Jēsū ēbriam.
Quā vīsā, quum vehementer admīrārer, angelus sīc mēcum loquitur:
\vs%{Apc-17-7}
\quote{%
  Quid mīrāris? Ego tibi mulieris mystērium ēloquar et portantis eam \BĒLUAE,
  habentis septem capita et cornua decem:
\vs%{Apc-17-8}
  \BĒLUA, quam vīdistī, fuit et nōn est,
  et adscēnsūra est ex \TARTARŌ et in perniciem abitūra;
  mīrābuntur‧que terrae incolae,
  quōrum nōmina in vītae librō ab orbe conditō scrīpta nōn sunt,
  videntēs \BĒLUAM fuisse, nec esse, tametsī est.
\vs%{Apc-17-9}
  Hic est sententia, quae acūmen habet:
  septem capita septem sunt montēs, in quibus sedet mulier;
\vs%{Apc-17-10}
  septem‧que rēgēs sunt, quōrum quīnque cecidērunt, et ūnus est;
  alius nōndum vēnit, et quum vēnerit, paululum dūrātūrus est.
\vs%{Apc-17-11}
  Bēstia autem, quae fuit et nōn est,
  is est octāvus et ex septem est in‧que perniciem abit.
\vs%{Apc-17-12}
  Decem autem, quae vīdistī, cornua, decem rēgēs sunt,
  quī rēgnum nōndum adeptī sunt;
  sed potestātem utī rēgēs eādem hōrā adipīscentur cum \BĒLUĀ.
\vs%{Apc-17-13}
  Iī eumdem animum habēbunt, suās‧que vīrīs ac potestātem \BĒLUAE trādent;
\vs%{Apc-17-14}
  iī cum \AGNŌ duellum gerent, eōs‧que vincet \AGNUS,
  utpote dominōrum \DOMINUS et rēgum \RĒX,
  necnōn vocātī cum eō ēlēctī‧que ac fīdentēs.

\vs%{Apc-17-15}
  Quās autem aquās vīdistī ― inquit mē adloquēns ― ubī̆ sedet \MERETRĪX,
  populī sunt et plēbēs et gentēs et linguae.
\vs%{Apc-17-16}
  Item decem cornua, quae super \BĒLUĀ vīdistī, iī \MERETRĪCEM ōderint,
  eam‧que dēsertam ac nūdam reddent,
  et ejus carnēs comedent, et eam ign|ī combūrent:
\vs%{Apc-17-17}
  iīs enim \DEUS eam mentem injēcit, ut ejus sententiam exsequantur,
  eōdem‧que animō agant, et rēgnum suum \BĒLUAE trādant,
  dōnec \DEĪ mandāta peragantur.
\vs%{Apc-17-18}
  Mulier autem, quam vīdistī, est urbs magna, rēgnum in rēgēs terrae obtinēns}.
\stopbookchapter

\startbookchapter %Chapter 18
\startabstract
  Horrendum Babylōnis excidium prōpōnitur, lūgent negōtiātōrēs terrae,
  quī ex pompā et luxū ipsī̆us dītātī sunt.
  Exsultant vērō omnēs ēlēctī ex tam jūstā \DEĪ ultiōne.
\stopabstract

\vs%{Apc-18-1}
Posteā vīdī alium angelum dē caelō dēscendentem, magnā potestāte praeditum,
cujus splendōre conlūstrāta terra est.
\vs%{Apc-18-2}
Is vehementer, magnā vōce clāmāns, dīxit:
\quote{%
  Cecidit, cecidit magna Babylōn, facta‧que est daemonum domicilium
  et omnium impūrōrum spīrituum
  omnium‧que obscēnārum et invīsārum volucrum cūstōdia!
\vs%{Apc-18-3}
  Quoniam taetrō stuprōrum suōrum vīnō gent|īs omn|īs pōtiōnāvit,
  rēgēs‧que terrārum cum eā scortātī sunt,
  et terrārum mercātōrēs ejus dēliciārum cōpiā locuplētātī}.

\vs%{Apc-18-4}
Item aliam vōcem audīvī dē caelō dīcentem:
\quote{%
  Migrāte ex eā, mī \POPULE,
  nē sītis ejus peccātōrum participēs nēve ejus clādibus adficiāminī:
\vs%{Apc-18-5}
  quoniam ejus peccāta ad caelum ū̆sque \unclassical{pertigērunt},
  ejus‧que scelera recordātur \DEUS.
\vs%{Apc-18-6}
  Adficite eam, ut et ipsa vōs adfēcit,
  eī‧que duplicem prō ejus factīs grātiam referte;
  eōdem pōculō, quō miscuit, miscēte eī duplum.
\vs%{Apc-18-7}
  Quantum ipsa sē honōrāvit et luxuriāta est,
  tantum eī cruciātum lūctum‧que date.
  Quoniam ipsa sīc cum animō suō cōgitat:
  \subquote{Sedeō rēgīna neque vidua futūra sum neque lūctū adficiar},
\vs%{Apc-18-8}
  proptereā ūnō diē venient ejus clādēs, mors, lūctus, famēs,
  ea‧que ign|ī cremābitur:
  quoniam potēns est \DOMINUS\ \DEUS, quī eam ulcīscētur}.

\vs%{Apc-18-9}
Eam‧que terrārum rēgēs, quī cum eā scortātī luxuriātī‧que fuerint,
dēplōrābunt ac lāmentābuntur, quum ejus incendiī fūmum vidēbunt,
\vs%{Apc-18-10}
procul, metū ejus cruciātūs, absistentēs ac dīcentēs:
\quote{%
  Ēheu, urbs illa magna Babylōn, urbs illa potēns,
  cujus subplicium ūnā vēnit hōrā!}.

\vs%{Apc-18-11}
Item eam terrārum mercātōrēs dēflēbunt ac lūgēbunt,
quod eōrum mercem nēmō jam emet:
\vs%{Apc-18-12}
mercem aurī et argentī et gemmārum et margarītārum,
et byssī et purpurae et sēricī et coccinī,
et omnis generis lignum thȳinum,
eburnea‧que et ex pretiōsissimīs lignīs cōnfecta vāsa
et aere et ferrō et marmore,
\vs%{Apc-18-13}
et cinnamōmum et odōrēs et unguentum et thūs,
et vīnum et oleum et similam et trīticum,
et jūmenta et ovēs et equōrum et raedārum,
et corporum et animās hominum.
\vs%{Apc-18-14}
Discessērunt ā tē cupīta animō tuō pōma,
omnia dēnique opīma et praeclāra ā tē discessērunt,
nec ea amplius inveniēs.

\vs%{Apc-18-15}
Hōrum mercātōrēs ab eā locuplētātī, procul, prae ejus cruciātūs metū,
adstābunt, plōrantēs ac lūgentēs,
\vs%{Apc-18-16}
et ita dīcentēs:
\quote{%
  Heu, heu, urbs illa magna, indūta byssinō et purpureō et coccinō,
  et aurō deaurāta gemmīs‧que et margarītīs!
\vs%{Apc-18-17}
  Ūnā hōrā dēpopulātae sunt tantae opēs!}.

Item omnēs nāviculāriī omnis‧que cōpia nāvālis et nautae
et quīcumque in marī negōtiantur, procul adstitērunt,
\vs%{Apc-18-18}
et ejus incendiī fūmum spectantēs, ita vōciferārī īnstitērunt:
\quote{Ecquae pār fuit urbī illī magnae?}.
\vs%{Apc-18-19}
% deponent verb quirītārī, but OLD prefers active quirītāre (p. 1716);
% infinitive for finite form
Tum sparsō super capita sua pulvere, flentēs ac lūgentēs ita quirītārī:
\quote{%
  Heu, heu, urbs magna, in quā omnēs habentēs in marī nāvigia,
  dītātī sunt ejus pretiīs! Ea ūnā hōrā dēpopulāta est!
\vs%{Apc-18-20}
  Laetāre super ea caelum, sānctī‧que apostolī atque vātēs,
  quoniam \DEUS vestrās injūriās in eam ultus est!}.

\vs%{Apc-18-21}
Deinde sustulit rōbustus quīdam angelus saxum ingēns utī molam,
et in mare dējēcit dīcēns:
\quote{%
  Tālī impetū jaciētur Babylōn urbs illa magna nec amplius exstābit,
\vs%{Apc-18-22}
  nec amplius in tē citharoedōrum et mūsicōrum et tībīcinum et tubicinum
  vōx audiētur,
  nec amplius in tē ūllius artis artifex exstābit,
  nec amplius in tē molae sonus audiētur,
\vs%{Apc-18-23}
  nec amplius in tē lucernae lūmen lūcēbit,
  nec amplius in tē spōnsī spōnsae‧ve vōx audiētur!
  Tuī mercātōrēs erant hominum optimātēs,
  tuīs venēficiīs dēceptae sunt gentēs omnēs,
\vs%{Apc-18-24}
  et in eādem tē vātum sānctōrum‧que cruor inventus est,
  et omnium in terrīs occīsōrum!}.
\stopbookchapter

\startbookchapter %Chapter 19
\startabstract
  Laudant caelestēs \DEUM, quod vindicāvit sanguinem suōrum ē manū \MERETRĪCIS.
  Beātī dēscrībuntur, quī ad cēnam nuptiārum \AGNĪ vocātī sunt.
  Angelus adōrārī sē prohibet.
  Ē caelō summus ille \RĒX rēgum adpāret,
  duellum oritur in quō adprehenditur \BĒSTIA, et in stāgnum ārdēns dējicitur.
\stopabstract

\vs%{Apc-19-1}
 Posteā audīvī magnam ingentis multitūdinis vōcem in caelō dīcentis:
\startlines\quote{%
  Hallēluja!
  Salūs et glōria et honōs et potestās \obsolete{\DOMINŌ,} \DEŌ nostrō,
\vs%{Apc-19-2}
  cujus vēra jūsta‧que sint jūdicia,
  quī magnam illam \MERETRĪCEM,
  quae terram impudīcitiā suā conrumpēbat,
  damnāverit et ab eā poenās sanguinis suōrum repetīverit!}.
\stoplines

\vs%{Apc-19-3}
Iterum‧que dīxērunt:
\quote{Hallēluja! Ejus autem fūmus exhālat in omnia saecula!}

\vs%{Apc-19-4}
Tum vīgintī quattuor senātōrēs, et quattuor animālia prōcubuērunt,
{\DEUM}‧que in soliō sedentem venerātī sunt, dīcentēs:
\quote{Āmēn. Hallēluja!}.

\vs%{Apc-19-5}
Et vōx ex soliō prōdiit dīcēns:
\startlines\quote{%
  Laudāte \DEUM nostrum omnēs ejus servī
  eum‧que metuentēs, parvī juxtā ac magnī!}.
\stoplines

\vs%{Apc-19-6}
Item audīvī quasi ingentis multitūdinis vōcem quasi‧que multārum aquārum
et vehementium tonitruōrum sonitum, quī dīcēbant:
\startlines\Lquote{%
  Hallēluja!
\vs%{Apc-19-7}
  Rēgnum enim adeptus est \DOMINUS, \DEUS omnipotēns.
  Gaudeāmus et exsultēmus eī‧que glōriam tribuāmus,
  quoniam \AGNĪ vēnērunt nuptiae ejus‧que conjūnx sēsē parāvit.
\vs%{Apc-19-8}
  Eī‧que mandātum est, utī byssinō pūrō splendidōque induātur;
  est autem byssinum virtūtēs sānctōrum}.
\stoplines

\vs%{Apc-19-9}
\Rquote{
  Scrībe: ― inquit mē adloquēns ― Beātī,
  quī ad nuptiārum \AGNĪ cēnam vocātī sunt!
  Haec sunt ― inquit mihi ― vēra dicta \DEĪ}.
\vs%{Apc-19-10}
Hīc ego eī ad pedēs adcidī, ejus venerandī grātiā.
Sed ille:
\quote{%
  Cavē ― inquit ― nē faciās!
  Cōnservus tuus sum tuōrum‧que frātrum, Jēsū testimōnium habentium.
  Deum venerātor: nam Jēsū testimōnium, ōrāculī spīritus est}.

\vs%{Apc-19-11}
Deinde vīdī apertum caelum: et ecce albus equus, cui quī īnsidēbat
vocātur \FIDĒLIS et \VĒRUS, jūstē‧que et jūdicat et duelligerat.

\vs%{Apc-19-12}
Ejus oculī sunt igneae flammae similēs, in capite īnsunt multa diadēmata;
habet nōmen scrīptum, quod nūllus nōvit, nisi ipse;
\vs%{Apc-19-13}
indūtus est palliō tīnctō sanguine, et nōmine vocātur \SERMŌ\ \DEĪ.
\vs%{Apc-19-14}
Eum‧que sequuntur in equīs albīs exercitūs, candidō mundō‧que byssinō indūtī.
\vs%{Apc-19-15}
Ex ejus ōre prōdit gladius acūtus, quō gent|īs feriat,
\bibleallusion{quōs} ipse \bibleallusion{ferreō baculō reget};
īdem‧que calcābit torcular vīnī saevitiae īrae‧que \DEĪ omnipotentis.
\vs%{Apc-19-16}
% ablative of femur; orig. femore
Habet autem in palliō et femine suō scrīptum nōmen:
rēgum \RĒX et dominōrum \DOMINUS.

\vs%{Apc-19-17}
Praetereā vīdī alium angelum in sōle stantem, quī magnā vōce clāmāns,
omn|īs mediō caelō volant|īs av|īs sīc adlocūtus est:
\quote{%
  Adeste et coīte ad magnī \DEĪ cēnam:
\vs%{Apc-19-18}
% orig. potentum, probably a mistake
  utī rēgum, utī tribūnōrum, utī potentium, ut equōrum equitum‧que,
  ut omnium --- tum līberōrum, tum servōrum, tum parvōrum, tum magnōrum ---
  carnēs comedātis}.

\vs%{Apc-19-19}
Vīdī et \BĒLUAM, terrae‧que rēgēs et eōrum exercitūs congregātōs,
ad duellum gerendum cum equī equite ac ejus exercitū.
cd\vs%{Apc-19-20}
Vērum comprehēnsa est \BĒLUA, et cum eā falsus vātēs,
quī in ejus cōnspectū fēcerat ostenta, quibus dēcēperat eōs,
quī \BĒSTIAE notam adcēperant quī‧que ejus venerābantur imāginem;
duo illī vīvī conjectī sunt in stāgnum ignis ārdēns sulpure.
\vs%{Apc-19-21}
Reliquī gladiō necātī sunt sedentis in equō, quī gladius ex ejus ōre prōdībat,
eōrum‧que carnibus satiātae sunt omnēs ālitēs.
\stopbookchapter

\startbookchapter %Chapter 20
\startabstract
  Angelus Sātānān vinculīs cōnstringit ad mīlle annōs;
  quibus solūtus, Gōg et Māgōg,
  id est, obcultōs et apertōs hostēs in sānctōs concitat.
  Sed \DOMINĪ vindicta coërcet illōrum īnsolentiam.
  Aperiuntur librī ex quibus jūdicantur mortuī.
\stopabstract

\vs%{Apc-20-1}
Deinde vīdī angelum dē caelō dēscendentem,
clāv|im habentem \TARTARĪ et ingentem catēnam in manū suā.
\vs%{Apc-20-2}
Is cēpit \DRACŌNEM, serpentem antīquum, quī est \DIABOLUS et Sātānās,
eum‧que conligāvit in mīlle annōs;
\vs%{Apc-20-3}
et in \TARTARUM conjēcit conclūsit‧que et īnsuper obsignāvit,
nē deinceps gent|īs dēciperet, dōnec perāctī forent mīlle annī,
post quōs solvendus est ad exiguum tempus.
\vs%{Apc-20-4}
Tum sellās vīdī in quibus cōnsessum est, illīs‧que jūdicium datum est;
item‧que vīdī animōs dēconlātōrum
propter Jēsū testimōnium propter‧que \DEĪ sermōnem,
et quī neque \BĒSTIAM aut ejus imāginem venerātī sunt,
neque notam in suam frontem aut manum adcēpērunt:
iī revīxērunt et cum Chrīstō mīlle annōs rēgnārunt.
\vs%{Apc-20-5}
Reliquī mortuōrum nōn revīxērunt, dōnec trānsāctī forent mīlle annī.
Haec est prīma resubrēctiō.
\vs%{Apc-20-6}
Fēlīcēs sānctī‧que sunt, quī sunt prīmae resubrēctiōnis compotēs!
In eōs secunda mors potestātem nōn habet,
quia erunt \DEĪ et Chrīstī sacerdōtēs et cum eō rēgnābunt mīlle annōs.

\vs%{Apc-20-7}
Perāctīs iīs mīlle annīs, solvētur Sātānās ex suō carcere
\vs%{Apc-20-8}
et proficīscētur ad dēcipiendās gent|īs,
quae quattuor terrae angulīs continēbuntur, Gōgum et Māgōgum,
ut eōs ad duellum convocet, quōrum tantus erit numerus,
quantus est harēnae maris.
\vs%{Apc-20-9}
Hī, factā, quam lātē tellūs patet, expedītiōne,
sānctōrum castra et urbem amātam circumsēdērunt.
Sed eōs dēlāpsus ā \DEŌ ignis dē caelō cōnfēcit;
\vs%{Apc-20-10}
et \DIABOLUS, quī eōs dēcipiēbat, dējectus est in igneum sulpureum‧que stāgnum,
ubī̆ erat \BĒLUA falsus‧que vātēs;
cruciābuntur‧que diēs ac noctēs in omnem perennitātem.

\vs%{Apc-20-11}
Item vīdī solium album ingēns et quemdam in eō sedentem,
cujus ex cōnspectū terra caelum‧que fūgit, nec ūllō jam locō exstitērunt.
\vs%{Apc-20-12}
Vīdī item mortuōs, parvōs juxtā ac magnōs, ante \DEUM adstant|īs;
apertī‧que sunt librī, et alius liber apertus, quī est vītae liber,
et jūdicātī sunt mortuī ex librōrum scrīptīs prō suīs factīs;
\vs%{Apc-20-13}
reddidit‧que mare suōs mortuōs, et \MORS et \ORCUS mortuōs reddidēre suōs:
dē quibus, ut erant cujusque facta, ita jūdicātum est.

\vs%{Apc-20-14}
Mors autem et \ORCUS conjectī sunt in stāgnum igneum. Haec est secunda mors.
\vs%{Apc-20-15}
Quodsī quis in vītae librō scrīptus inventus nōn est,
is in igneum stāgnum conjectus est.
\stopbookchapter

\startbookchapter %Chapter 21
\startabstract
  Novam \JEROSOLYMAM dēscendentem ē caelō \SPŌNSAM, uxōrem \AGNĪ,
  ac illī̆us strūctūram dēscrībit pretiōsīs lapidibus ōrnātam,
  cujus templum est \AGNUS.
\stopabstract

\vs%{Apc-21-1}
Vīdī autem novum caelum, novam‧que terram:
nam prius caelum prior‧que terra perierat, nec usquam jam erat mare.
\vs%{Apc-21-2}
Ego Jōhannēs vīdī urbem sānctam \JEROSOLYMAM novam
ā \DEŌ dē caelō dēscendentem,
parātam utī \SPŌNSAM virō suō ōrnātam.
\vs%{Apc-21-3}
Audīvī‧que magnam vōcem dē caelō dīcentem:
\quote{%
  \bibleallusion{Ecce} \DEĪ apud hominēs \bibleallusion{tabernāculum}!
  \bibleallusion{Quid apud eōs habitābit, et ipsī ejus \POPULUS erunt},
  et ipse \DEUS erit apud eōs \DEUS eōrum;
\vs%{Apc-21-4}
  \bibleallusion{absterget‧que} \DEUS
  \bibleallusion{omnem lacrimam} ex eōrum oculīs,
  nec amplius mors erit neque lūctus neque quirītātus neque labor erit amplius,
  quod priōra abierint.

\vs%{Apc-21-5}
  Ego ― inquit is, quī in soliō sedēbat ― factūrus sum omnia nova}.
Tum mē adloquēns:
\quote{Scrībe ― inquit ― haec dicta vēra fīda‧que esse}.
\vs%{Apc-21-6}
Tum mihi dīxit:
\quote{%
  Factum est! Ego sum \ALPHA et \ŌMEGA, \PRĪNCIPIUM et \FĪNIS.
  Ego sitientī dabō dē fonte aquae vītae grātīs.
\vs%{Apc-21-7}
  Victor possidēbit omnia,
  \bibleallusion{eī‧que ego \DEUS erō et ipse mihi fīlius erit}.
\vs%{Apc-21-8}
  Sed ignāvōs et īnfidēlīs et exsecrābilīs et homicīdās et impudīcōs
  et venēficōs et \unclassical{deastricolās} et omn|īs mendācēs:
  hōs, inquam, sua sors manet in stāgnō quod ign|ī ārdet et sulpure,
  quae secunda mors est}.

\vs%{Apc-21-9}
Tum vēnit ad mē ūnus septem angelōrum,
quī septem phialās habēbant septem postrēmārum clādium plēnās,
quī mē sīc adlocūtus est:
\quote{Ades, mōnstrābō tibi \SPŌNSAM, \AGNĪ uxōrem}.
\vs%{Apc-21-10}
Deinde mē dīvīnō adflātū in magnum arduum‧que montem sustulit
et mihi mōnstrāvit urbem magnam,
sānctam \JEROSOLYMAM dē caelō ā \DEŌ dēscendentem, dīvīnā glōriā praeditam;
\vs%{Apc-21-11}
ejus lūmināre simile erat lapidī pretiōsissimō,
velutī jaspidī crystallum imitantī;
\vs%{Apc-21-12}
habēbat‧que mūrum ingentem et altum, et \bibleallusion{portās} duodecim,
super‧que portīs angelōs totidem, et īnscrīpta \bibleallusion{nōmina},
vidēlicet duodecim \bibleallusion{Isrāēlīticārum tribuum}.
\vs%{Apc-21-13}
\bibleallusion{%
  Ab oriente trēs erant portae,
  ā boreā trēs,
  ab austrō trēs,
  ab occidente trēs};
\vs%{Apc-21-14}
urbis mūrus habēbat fundāmenta duodecim,
et in iīs nōmina duodecim apostolōrum \AGNĪ.

\vs%{Apc-21-15}
Quī autem mēcum loquēbātur, calamum habēbat aureum,
quō urbem ejus‧que portās et mūrum mētīrētur.
\vs%{Apc-21-16}
Cujus urbis is situs erat, ut esset quadrāta:
tantā longitūdine, quantā lātitūdine.
Hanc urbem calamō dīmēnsus est ad stadiōrum duodecim mīlia,
parī longitūdine et lātitūdine ac altitūdine.
\vs%{Apc-21-17}
Ejus mūrum centum quadrāgintā quattuor cubitīs dīmēnsus est,
hominis, hoc est angelī, mēnsūrā.
\vs%{Apc-21-18}
Ejus mūrī strūctūra erat ex jaspide, urbs ipsa ex aurō pūrō, pūrī vitrī similis.
\vs%{Apc-21-19}
Mūrī urbis fundāmenta omnibus ōrnatā gemmīs:
prīmum fundāmentum erat jaspis,
alterum sapphīrus,
tertium chalcēdonius,
quārtum smaragdus,
\vs%{Apc-21-20}
quīntum sardonyx,
sextum sardius,
septimum chrȳsolithus,
octāvum bēryllus,
nōnum topazus, %orig. topazium
decimum chrȳsoprasus,
ūndecimum hyacinthus,
duodecimum amethystus.
\vs%{Apc-21-21}
Duodecim portae duodecim erant ūniōnēs,
ex singulīs ūniōnibus singulae cōnstantēs;
urbis ārea erat aurum pūrum, perlūcida utī vitrum.

\vs%{Apc-21-22}
In eā fānum nōn vīdī: nam \DOMINUS, \DEUS omnipotēns, ejus fānum est, et \AGNUS.
\vs%{Apc-21-23}
Urbs sōle nōn eget neque lūnā, quī in eā lūceant:
nam \DEĪ splendor eam conlūstrat, ejus‧que lucerna \AGNUS est.
\vs%{Apc-21-24}
Atque in ejus urbis lūmine gentēs servātōrum ingredientur,
in eam‧que cōnferent terrārum rēgēs suam glōriam ac honōrem;
\vs%{Apc-21-25}
nec ejus portae claudentur interdiū: nam nox quidem ibī̆ nūlla erit,
\vs%{Apc-21-26}
et in eam cōnferētur gentium glōria et honōs,
\vs%{Apc-21-27}
nec in eam intrābit quidquam profānum quod‧ve scelus conmittat aut mendācium,
sed dumtaxat scrīptī in \AGNĪ librō vītae.
\stopbookchapter

\startbookchapter %Chapter 22
\startabstract
  Ostenditur aquae vīvae fluvius et arbor vītae.
  Sequitur prophētīae hujus conclūsiō,
  in quā Jōhannēs ostendit vērissima esse quae hīc continentur.
  Et jam tertiō repetit, dictāta haec omnia ab eō esse,
  quī est \ALPHA et \ŌMEGA. % wrong kerning
\stopabstract

\vs%{Apc-22-1}
Deinde mihi mōnstrāvit vītālis aquae pūrum fluvium, limpidum utī crystallum,
ex \DEĪ et \AGNĪ soliō mānantem.
\vs%{Apc-22-2}
Et \bibleallusion{in mediō} ejus āreae et
\bibleallusion{fluviī hinc atque hinc vītae arborem},
frūctūs duodecim parientem,
singulīs \bibleallusion{mēnsibus suum frūctum} edentem.
Arboris \bibleallusion{folia ad} gentium \bibleallusion{cūrātiōnem} pertinent.
\vs%{Apc-22-3}
In eā nec ūlla deinceps erit dēvōtiō et \DEĪ\ {\AGNĪ}‧que solium erit;
eum‧que suī servī colent
\vs%{Apc-22-4}
et ejus faciem vidēbunt et ejus nōmen gestābunt in suīs frontibus.
\vs%{Apc-22-5}
Nec erit ibī̆ nox, nec lucernā sōlis‧ve lūmine indigēbunt,
quippe quōs conlūstrābit \DOMINUS\ \DEUS, et rēgnābunt in omnia saecula.

\vs%{Apc-22-6}
\quote{Haec sunt ― inquit mē adloquēns ― certa vēra‧que dicta};
et \DOMINUS, sānctōrum vātum \DEUS,
angelum suum mīsit ad mōnstrandum suīs servīs,
quae brevī futūra sunt.
\vs%{Apc-22-7}
\christquote{%
  Ego brevī ventūrus sum: fēlīx, quī verbīs pāreat ōrāculī hujus librī}.

\vs%{Apc-22-8}
Ego vērō Jōhannēs, quī haec et adspexī et audīvī,
quum audīvissem adspexissem‧que, angelō, quī haec mihi mōnstrābat,
ad pedēs adcidī, ut eum venerārer.
\vs%{Apc-22-9}
Sed ille:
\quote{%
  Cavē, nē faciās: ― inquit ―
  tuus enim cōnservus sum tuōrum‧que frātrum vātum et eōrum,
  quī istī̆us librī dictīs pārent. Deum venerātor!}.

\vs%{Apc-22-10}
 Tum illud addidit:
\quote{%
  Ōrāculī istī̆us librī dicta nē obsignātō: nam tempus in propinquō est!
\vs%{Apc-22-11}
  Quī nocet, nocēre pergat; et quī sordet, sordēre pergat;
  et jūstus, jūstus esse pergat; et sānctus, sānctus esse pergat}.

\vs%{Apc-22-12}
\christquote{%
  Equidem brevī ventūrus sum, meam mēcum mercēdem ferēns,
  utī prō suō quemque meritō remūnerer.
\vs%{Apc-22-13}
  Ego sum \ALPHA et \ŌMEGA, \PRĪMUS et \ULTIMUS, \PRĪNCIPIUM et \FĪNIS}.
\vs%{Apc-22-14}
  Beātī, quī ejus praecepta obeunt,
  utī potestātem habeant in arborem vītae, et per portās intrent in urbem.
\vs%{Apc-22-15}
  Forīs quidem erunt canēs et venēficī et inpudīcī et homicīdae
  et \unclassical{deastricolae} et quisquis falsa amat atque facit!

\vs%{Apc-22-16}
\christquote{%
  Ego Jēsus meum angelum mīsī, quī vōbīs haec super \ECCLĒSIĪS testārētur.
  Ego sum Dāvīdis stirps atque genus, stēlla clāra et mātūtīna}.

\vs%{Apc-22-17}
Et \SPĪRITUS et \SPŌNSA dīcunt: \quote{Venī!};
et quī audit, dīcat: \quote{Venī!};
et quī sitit, veniat; et quī vult, sūmat aquam vītae grātīs.

\vs%{Apc-22-18}
Testificor autem omnibus ōrāculī hujus librī dicta audientibus:
sī quis ad haec addiderit, addet eī \DEUS clādīs in hōc librō scrīptās;
\vs%{Apc-22-19}
et sī quis dē verbīs librī hujus ōrāculī adēmerit,
adimet \DEUS sortem ejus dē librō vītae dē‧que sacrā urbe,
et scrīptīs in hōc librō.

\vs%{Apc-22-20}
Dīcit, quī haec testātur: \christquote{Etiam, veniam brevī}.
Āmēn: \obsolete{etiam} venī, \DOMINE Jēsū!

\vs%{Apc-22-21}
Dominī nostrī Jēsū Chrīstī favor adsit vōbīs omnibus. \obsolete{Āmēn.}
\stopbookchapter
\stopbook
\stopcomponent

%%% Local Variables:
%%% coding: utf-8
%%% mode: context
%%% TeX-master: t
%%% End:
