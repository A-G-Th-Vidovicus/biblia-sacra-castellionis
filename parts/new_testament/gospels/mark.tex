% Copyright (C) 2024 A. G. Th. Vidovicus.

% This program is free software: you can redistribute it and/or modify it
% under the terms of the GNU General Public Licēnse as published
% by the Free Software Foundation, either version 3 of the Licēnse,
% or (at your option) any later version.

% This program is distributed in the hope that it will be useful,
% but WITHOUT ANY WARRANTY; without even the implied warranty
% of MERCHANTABILITY or FITNESS FOR A PARTICULAR PURPOSE.
% See the GNU General Public Licēnse for more details.

% You should have received a copy of the GNU General Public Licēnse along
% with this program. If not, see <https://www.gnu.org/licēnses/>.

% ┌──────────┐
% │01.07.2024│
% └──────────┘

\startcomponent mark
\startbook[
  bookmark={Evangelium secundum Marcum},
   marking={Evangelium secundum Marcum},
     title={Ēvangelium auctōre Mārcō},
]
 % Sadly, I have no clue who Cosmas Indicopolita is
\startintroduction[title={In tōtum Mārcum argūmentum Cosmae Indicopolitae}]
  Hīc secundus ēvangelista Mārcus, Petrō eī Rōmae mandante,
  cōnscrīpsit \ĒVANGELIUM, incipiēns suam ēvangelicam nārrātiōnem ā Baptismate:
  quī est typus resubrēctiōnis ex mortuīs,
  per quam ad inmortālem et inmūtābilem vītam regenerāmur.
  Post quod prīncipium etiam ipse nārrat tum tentātiōnēs ac victōriam,
  tum et Jūdaeōrum īnsidiās, invāsiōnem‧que aut captīvitātem, mortem‧que,
  tum dēnique etiam resubrēctiōnem: atque ita suam historiam absolvit.
  Meminit ipse quoque dē Jōhanne Baptistā praedicante
  adpropinquāre \RĒGNUM caelōrum.
  Quae omnia concorditer cum beātō Matthaeō prōnūntiat:
  ūnus enim scopus est tōtī̆us \SACRAE\ \SCRĪPTŪRAE, nempe Chrīstus.
  Hic igitur quoque Novī \TESTĀMENTĪ praecō existēns,
  eadem cum praecēdente nōbīs dēscrīpsit, ab historiā Baptismī ōrdiēns:
  quod Baptisma est typus resubrēctiōnis ex mortuīs,
  dīcō autem novae et caelestis conversātiōnis.
  Praetereā exposuit, quōmodo sit baptīzātus, in terrā versātus,
  interfectus, resuscitātus, et dēnique in caelum regressus;
  ubī̆ fēcundae futūrae‧que vītae locus et polītīa est.
  Glōria autem sit \DEŌ, quī ista omnia nōbīs inde ab initiō praeparāvit,
  praeadnūntiāvit, et dēnique inplēvit.
  Āmēn.
\stopintroduction

\startbookchapter %Chapter 1
\startabstract
  Dē Jōhanne vāticinium: ejusdem locus, āctiō, conciō, vestis, cibus.
  Chrīstī baptismus et Jōhannis. Baptīzātus Jēsus. Vōx cum columbā.
  % ‘baptisma’ may be better
  Chrīstī tentātiō; ad hunc angelī; ejus doctrīna.
  Simōnis, Andreae, Jācōbī, Jōhannis vocātiō.
  Spīritus inpūrus. Petrī socrus. Ad sacerdōtem leprōsus.
\stopabstract

\vs%{Mc-1-1}
Initium \ĒVANGELIĪ Jēsū Chrīstī, \DEĪ\ \FĪLIĪ,
\vs%{Mc-1-2}
utī scrīptum est in vātibus:
\startlines
\quote{%
  \bibleallusion{Ego tibi meum praemittam nūntium,
  quī} tibi \bibleallusion{viam praeparet;}
\vs%{Mc-1-3}
  \bibleallusion{vōx clāmantis in sōlitūdine:}
  \subquote{%
    \bibleallusion{parāte viam \DOMINĪ, dīrigite sēmitās} ejus}}.
\stoplines

\vs%{Mc-1-4}
Baptīzābat Jōhannēs in sōlitūdine
et ēmendātiōnis vītae baptisma pūblicābat ad peccātōrum veniam.
\vs%{Mc-1-5}
Ad eum‧que proficīscēbātur tōta Jūdaea regiō ac Hierosolymītānī,
et ab eō baptīzābantur omnēs in Jordāne fluviō cōnfitentēs peccāta sua.
\vs%{Mc-1-6}
Erat autem Jōhannēs indūtus camēlīnīs pilīs, lateribus pelliciō cingulō cīnctīs,
vēscēbātur‧que locustīs et melle silvestrī.

\vs%{Mc-1-7}
Atque hujusmodī verbīs pūblicē docēbat:
\quote{%
  Venit quīdam post mē, adeō mē praestantior,
  ut ego nōn sim dignus quī ejus calceōrum conrigiam prōnus solvam.
\vs%{Mc-1-8}
  Ego quidem vōs aquā baptīzāvī; at is vōs \SĀNCTŌ\ \SPĪRITŪ baptīzābit}.

\vs%{Mc-1-9}
Adcidit autem utī per eōs diēs venīret Jēsus ā Nāzaretā Galilaeae
isque ā Jōhanne baptīzātus est in Jordāne.
\vs%{Mc-1-10}
Quī Jōhannēs simul ac ex aquā ēgressus est,
vīdit caelum findī et \SPĪRITUM quasi columbam dēscendere in eum;
\vs%{Mc-1-11}
exstitit‧que vōx ex caelō:
\godquote{Tū es meus cārissimus \FĪLIUS; quī mihi adceptus es}.

\vs%{Mc-1-12}
Statim‧que \SPĪRITUS eum ēgit in sōlitūdinem.
\vs%{Mc-1-13}
Ibī̆‧que fuit in sōlitūdine diēs quadrāgintā tentātus‧que est ā Sātānā;
erat‧que cum ferīs, et eī ministrābant angelī.

\vs%{Mc-1-14}
Postquam autem captus est Jōhannēs, vēnit Jēsus in Galilaeam
dīvīnī‧que \RĒGNĪ\ \ĒVANGELIUM pūblicāvit,
\vs%{Mc-1-15}
dīcēns
\indirectchristquote{%
  complētum esse tempus, et īnstāre \DEĪ\ \RĒGNUM;
  conrigerent sē et \ĒVANGELIŌ crēderent}.

\vs%{Mc-1-16}
Ambulāns autem apud Galilaeae lacum vīdit Simōnem et Andream ejus frātrem
jacient|īs in lacum rētia; erant enim piscātōrēs.
\vs%{Mc-1-17}
Et Jēsus dīxit iīs:
\christquote{Venīte post mē, et ecficiam, utī sītis hominum piscātōrēs}.
\vs%{Mc-1-18}
Atque illī cōnfestim, relictīs suīs rētibus, eum secūtī sunt.
\vs%{Mc-1-19}
Inde paululum prōgressus vīdit Jācōbum Zebedaeī fīlium
et Jōhannem ejus frātrem, quī et ipsī in nāve reficiēbant fundās,
\vs%{Mc-1-20}
eōs‧que prōtinus vocāvit.
Atque illī, patre suō Zebedaeō in nāve
cum mercēnāriīs relictō, eum subsecūtī sunt.

\vs%{Mc-1-21}
Profectī sunt autem Capharnāum.
Atque ille continuō ingressus Sabbatīs in conlēgium docet,
\vs%{Mc-1-22}
illīs ad ejus doctrīnam adtonitīs:
quippe quī eōs docēret utī potestāte praeditus, nōn utī scrībae.
\vs%{Mc-1-23}
Erat autem in eōrum conlēgiō homō inpūrum habēns spīritum, quī sīc exclāmāvit:
\vs%{Mc-1-24}
\quote{%
  Heu, quid tibi nōbīscum reī est, Jēsū Nāzarēne? Vēnistī perditum nōs?
  Sciō tē quis sīs: \SĀNCTUS\ \DEĪ}.
\vs%{Mc-1-25}
At Jēsus eum increpuit, dīcēns:
\christquote{Obmūtēsce et ex istō migrā!}.
\vs%{Mc-1-26}
Tum illum spīritus inpūrus lacerāvit et ingentī cum clāmōre exiit ex eō.
\vs%{Mc-1-27}
Id quod admīrātī sunt omnēs, ita ut inter sēsē disceptārent,
\indirectquote{quidnam id esset; quaenam ea foret nova doctrīna;
quod per potestātem etiam inpūrīs spīritibus inperāret, iī‧que eī pārērent}.
\vs%{Mc-1-28}
Itaque dīmānāvit celeriter ejus fāma in omniā Galilaeā fīnitimā.

\vs%{Mc-1-29}
Simul ac autem ex conlēgiō ēgressī sunt,
vēnērunt in aed|īs Simōnis et Andreae cum Jācōbō et Jōhanne.
\vs%{Mc-1-30}
Socrus autem Simōnis jacēbat febrīcitāns; id quod illī eī statim dīxērunt.
\vs%{Mc-1-31}
Atque ipse eam adgressus, ejus manū prehēnsā adlevāvit;
ea‧que prōtinus līberāta febre, coepit iīs ministrāre.

\vs%{Mc-1-32}
Sub vesperum autem, obcidente sōle,
adferēbantur ad eum omnēs male adfectī ac furiōsī;
\vs%{Mc-1-33}
tōta‧que cīvitās convēnerat ad for|īs.
\vs%{Mc-1-34}
Atque ille multōs, quī variīs morbīs vexābantur, sānāvit,
et multa daemonia ējēcit neque sinēbat dīcere daemonia,
sē scīre eum esse Chrīstum.

\vs%{Mc-1-35}
Māne autem, multā adhūc nocte, subrēxit exiit‧que Jēsus,
et abiit in dēsertum locum atque illīc ōrābat.
\vs%{Mc-1-36}
Quem cōnsecūtī Simōn et ejus comitēs,
\vs%{Mc-1-37}
eum‧que nactī, dīcunt eī \indirectquote{eum ab omnibus conquīrī}.
\vs%{Mc-1-38}
Quibus ille:
\christquote{%
  Eāmus ― inquit ― in fīnitima obpidula, ut illīc quoque praedicem:
  hāc enim dē causā profectus sum}.
\vs%{Mc-1-39}
Itaque pūblicē docēbat in eōrum conlēgiīs per tōtam Galilaeam
et daemonia ējiciēbat.

\vs%{Mc-1-40}
Adgressus autem eum leprōsus congenuclāns, hujusmodī verbīs rogāvit:
\quote{Sī vīs, potes mē pūrgāre}.
\vs%{Mc-1-41}
Et Jēsus miseritus, illum porrēctā manū tetigit:
\christquote{Volō ― inquit eī ― pūrgātor!};
\vs%{Mc-1-42}
et simul eō locūtō, discessērunt ab illō leprae, is‧que pūrgātus est.
\vs%{Mc-1-43}
Eī autem interminātus, eum prōtinus ējēcit,
\vs%{Mc-1-44}
et, \christquote{%
  Vidē ― inquit ― nē cui quid dīcās; sed ī ostēnsum tē sacerdōtī
  et obfer ob tuī pūrgātiōnem, quae praecēpit Mōysēs, ut iīs essent testimōniō}.
\vs%{Mc-1-45}
At ille ēgressus coepit multa praedicāre et rem ita dīvulgāre,
utī Jēsus nōn jam posset palam in urb|īs ingredī,
sed forīs dēsertīs in locīs esset; quum quidem ad eum venīrētur undique.
\stopbookchapter

\startbookchapter %Chapter 2
\startabstract
  Paralyticus. In peccāta Jēsus. Lēvī. Cum Chrīstō peccātōrēs. Cui Chrīstus.
  Prō discipulīs dē jejūniō ac spīcīs.
\stopabstract

\vs%{Mc-2-1}
Quum autem rūrsus post aliquot diēs ingressus esset Capharnāum,
\vs%{Mc-2-2}
ubī̆ audītum est, eum esse domī, prōtinus tam multī convēnērunt,
ut eōs nē vestibulum quidem caperet.
\vs%{Mc-2-3}
Ad quōs quum verba faceret, veniunt ad eum quīdam membrīs captum ferentēs,
quī ā quattuor portābātur.
\vs%{Mc-2-4}
Quī quum ad eum adcēdere propter hominum turbam nōn possent,
dētēxērunt tēctum ubī̆ erat ille; factō‧que forāmine,
dēmīsēre grabātum in quō jacēbat sīderātus.
\vs%{Mc-2-5}
Eōrum vīsā fīdūciā, Jēsus dīcit sīderātō:
\christquote{Gnāte, ignōscuntur tibi peccāta tua}.

\vs%{Mc-2-6}
Erant autem scrībārum quīdam illīc sedentēs, quī cum suīs animīs sīc cōgitābant:
\vs%{Mc-2-7}
\quote{%
  Quī fit, ut hic ita loquātur inpiē?
  Quis potest ignōscere peccāta nisi ūnus \DEUS?}.
\vs%{Mc-2-8}
Tum Jēsus, quum prōtinus animō suō eōs sīc sēcum cōgitāre cognōvisset,
dīxit iīs: \christquote{%
  Quid ista cōgitātis cum vestrīs animīs?
\vs%{Mc-2-9}
  Utrum facilius est dīcere sīderātō: \subchristquote{Ignōscuntur tibi peccāta},
  an dīcere: \subchristquote{Surge et tolle grabātum tuum et ambulā}?
\vs%{Mc-2-10}
  Atque utī sciātis potestātem habēre \FĪLIUM hominis ignōscendī in terrīs peccāta,
\vs%{Mc-2-11}
  tibi dīcō: ― inquit sīderātō ― surge et, sublātō grabātō tuō, abī domum tuam}.
\vs%{Mc-2-12}
Atque ille prōtinus subrēxit, sublātōque grabātō, exiit cōram omnibus,
adeō ut omnēs adtonitī \DEUM conlaudārent,
\indirectquote{sēsē nihil umquam tāle vīdisse} cōnfitentēs.

\vs%{Mc-2-13}
Profectus est autem rūrsus ad lacum; et ad eum omnēs vulgō veniēbant
is‧que eōs docēbat.
\vs%{Mc-2-14}
Praeteriēns autem Jēsus cōnspicātus Lēvīn Halphaeī fīlium
ad pūblicānōrum mēnsam sedentem, dīcit eī:
\christquote{Sequere mē}. Atque ille subrēxit, eum‧que secūtus est.
\vs%{Mc-2-15}
Adcidit autem utī, Jēsū in illī̆us aedibus convīvante,
quum multī pūblicānī et inprobī cum eō ejus‧que discipulīs adcumberent,
--- erant enim multī quī eum secūtī fuerant ---
\vs%{Mc-2-16}
Scrībae et Pharīsaeī,
eō vīsō cum pūblicānīs et inprobīs cibum capiente,
quaererent ex ejus discipulīs,
\indirectquote{%
  quī fī̆eret, utī cum pūblicānīs et inprobīs
  cibum pōtiōnem‧que sūmeret}.
\vs%{Mc-2-17}
Quō audītō, Jēsus dīxit iīs:
\christquote{%
  Nōn egent valentēs medicō, sed male adfectī;
  nōn vēnī vocātum īnsont||īs sed sont||īs \obsolete{ad frūgem}}.

\vs%{Mc-2-18}
Quum solērent autem Jōhannis et Pharīsaeōrum discipulī jejūnāre,
eum hujusmodī verbīs convēnērunt:
\quote{%
  Quī fit, utī Jōhannis et Pharīsaeōrum discipulī jejūnent,
  tuī discipulī nōn jejūnent?}.
\vs%{Mc-2-19}
Quibus Jēsus: \christquote{%
  Possunt‧ne quī nuptiās celebrant, ― inquit ―
  dum iīs spōnsus adest, jejūnāre?
  Quamdiū habent sēcum spōnsum, nōn possunt jejūnāre.
\vs%{Mc-2-20}
  Sed veniet tempus, quum iīs spōnsus auferētur; quō quidem tempore jejūnābunt.
\vs%{Mc-2-21}
  Neque vērō quisquam pannī rudis pittacium adsuit veterī vestīmentō;
  aliōquī novum illud ejus subplēmentum auferret veteris partem,
  atque ita dēterior fī̆eret fissūra.
\vs%{Mc-2-22}
  Neque quisquam vīnum novum in veterēs utr|īs recondit,
  aliōquī rumperet utr|īs vīnum novum et vīnum ecfunderētur et utrēs perderentur;
  sed est vīnum novum in novōs utr|īs condendum}.

\vs%{Mc-2-23}
Adcidit autem ut, eō per sata trānseunte, quum Sabbatī diēs esset,
ejus discipulī iter facientēs spīcās vellerent.
\vs%{Mc-2-24}
Itaque Pharīsaeī dīxērunt eī:
\quote{Ēn, cūr faciunt Sabbatīs, quod nōn licet?}.
\vs%{Mc-2-25}
Quibus ille:
\christquote{%
  Numquam‧ne lēgistis ― inquit ― quid fēcerit Dāvīd,
  quum aliquandō egēret et ēsurīret ipse ejus‧que comitēs?
\vs%{Mc-2-26}
  Utī \DEĪ domum ingressus, Abiāthāre pontifice,
  \unclassical{adposititiōs} pān|īs, quibus nōnnisi sacerdōtibus vēscī fās est,
  comēderit et suīs etiam comitibus inpertīverit?
\vs%{Mc-2-27}
  Sabbatum ― inquit iīs ― propter hominem factum est, nōn homō propter Sabbatum;
\vs%{Mc-2-28}
  itaque dominus \FĪLIUS est hominis etiam Sabbatī}.
\stopbookchapter

\startbookchapter %Chapter 3
\startabstract
  Ārida manus. Concilium in Jēsum. Discipulōrum numerus, lēgātiō, virtūtēs.
  Calumnia in Chrīstum dē Beelzebul. Peccātum inexpiābile.
  Quis Chrīstō cognātus.
\stopabstract

\vs%{Mc-3-1}
Ingressus est autem rūrsus in conlēgium.
Ubī̆, quum homō esset āridam habēns manum,
\vs%{Mc-3-2}
illī eum observābant, an Sabbatō eum sānāret, ut eum incūsārent.
\vs%{Mc-3-3}
Igitur ille hominī siccam habentī manum dīxit: \christquote{Surge in medium}.
\vs%{Mc-3-4}
Tum illōs adloquēns:
\christquote{%
  Licet‧ne Sabbatīs bene facere an male facere?
  Vītam servāre an interficere?}.
\vs%{Mc-3-5}
Quum‧que illī tacērent, ille eōs īrātē circum intuitus,
eōrum animī callum dolēns, dīcit hominī:
\christquote{Extende manum tuam}. Atque ille extendit,
ea‧que \obsolete{tam} sāna reddita est\obsolete{, quam erat altera}.
\vs%{Mc-3-6}
Pharīsaeī vērō ēgressī
prōtinus cum Hērōdiānīs cōnsilium dē eō perimendō cēpērunt.

\vs%{Mc-3-7}
At Jēsus sēsē cum suīs discipulīs recēpit ad lacum.
Eum‧que ingēns hominum multitūdō ā Galilaeā secūta est, et ā Jūdaeā,
\vs%{Mc-3-8}
et ā Hierosolymīs, et ab Idūmaeā, et ab agrō Trānsjordānīnō,
Tyrō‧que et Sīdōnī fīnitimī, quum audīvissent quae faceret,
ad eum ingentī multitūdine vēnērunt.
\vs%{Mc-3-9}
Itaque ille suīs discipulīs mandāvit,
utī sibi praestō esset nāvicula propter turbam, nē ab illīs premerētur.

\vs%{Mc-3-10}
Multōs enim sānāverat, itaque ruēbant in eum, ut eum tangerent,
quotquot erant in aliquō cruciātū.
\vs%{Mc-3-11}
Item spīritūs inpūrī, quum eum adspexerant, adcidēbant eī, et ita clāmābant:
\quote{Tū es \DEĪ\ \FĪLIUS!}.
\vs%{Mc-3-12}
Quōs ille multum objūrgābat, nē sē manifēstum facerent.

\vs%{Mc-3-13}
Ascendit autem in montem et advocāvit quōs voluit,
atque ubī̆ illī ad eum vēnērunt.
\vs%{Mc-3-14}
Cōnstituit \DUODECIM, quī sē comitārentur,
et quōs ad pūblicē docendum dīmitteret.
\vs%{Mc-3-15}
Quī‧que potestātem habērent \obsolete{cūrandī morbōs, et} daemonia ējiciendī:
\vs%{Mc-3-16}
inposuit autem Simōnī nōmen Petrum;
\vs%{Mc-3-17}
item‧que Jācobum Zebedaeī fīlium et Jōhannem Jācōbī frātrem,
quibus nōmina inposuit \implication{Boanerges}, hoc est \explication{tonitrūs fīliī};
\vs%{Mc-3-18}
et Andream et Philippum et Bartholomaeum et Matthaeum et Thōmān
et Jācōbum Halphaeī fīlium et Thaddaeum et Simōnem Cananitam
\vs%{Mc-3-19}
et Jūdam Iscariōtam: eum ā quō prōditus est.

\vs%{Mc-3-20}
Deinde ubī̆ domum vēnērunt, convēnit rūrsum multitūdō,
adeō utī nē cibum quidem capere possent.
\vs%{Mc-3-21}
Quō audītō, suī ad eum capiendum profectī sunt:
dīcēbant enim \indirectquote{eum esse īnsānum}.

\vs%{Mc-3-22}
Scrībae autem, quī ā Hierosolymīs dēscenderant, dīcēbant
\indirectquote{eum habēre Beelzebūlem}
et \indirectquote{per daemoniōrum prīncipem ējicere daemonia}.
\vs%{Mc-3-23}
At ille eōs advocātōs sīc est per similitūdinēs adlocūtus:
\christquote{%
  Potest‧ne Sātānās Sātānān ējicere?
\vs%{Mc-3-24}
  Quodsī quod rēgnum ab sēsē dissīdet, id rēgnum stāre nōn potest;
\vs%{Mc-3-25}
  aut sī qua domus sēcum ipsa dissīdent, ea domus stāre nōn potest.
\vs%{Mc-3-26}
  Quodsī Sātānās sēcum ipse pugnat ac dissīdet,
  stāre nequit, sed fīnem habēbit.
\vs%{Mc-3-27}
  Nōn potest quisquam alicujus potentis supellectilem,
  in ejus aed|īs ingressus dīripere, nisi prius potentem illum vinciat;
  atque ita dēmum ejus aed|īs dīripiat.
\vs%{Mc-3-28}
  Certō scītōte: omnium hominum generī peccāta
  et quaecumque maledicta dīxerint, ea esse ignōscenda;
\vs%{Mc-3-29}
  sed quī \SĀNCTŌ\ \SPĪRITUĪ maledīxerit, numquam adsecūtūrus est veniam,
  sed obnoxius est subpliciō sempiternō}.
\vs%{Mc-3-30}
Quoniam dīcēbant \indirectquote{eum inmundum habēre spīritum}.

\vs%{Mc-3-31}
Vēnērunt autem ejus māter et frātrēs,
forīs‧que manentēs cūrārunt eum arcessendum,
\vs%{Mc-3-32}
quum quidem eum circumsedēret hominum multitūdō.
Quum igitur illī eī dīxissent
\indirectquote{adesse forīs ejus mātrem et frātrēs \added{et sorōrēs}
  quī eum quaererent},
\vs%{Mc-3-33}
ille sīc iīs respondit:
\christquote{Quae est mea māter meī‧que frātrēs?}.
\vs%{Mc-3-34}
Tum circumspectīs discipulīs, quī sē circumsedēbant:
\christquote{%
  Ecce mea māter ― inquit ― meī‧que frātrēs:
\vs%{Mc-3-35}
  quisquis enim \DEĪ voluntātī pāret, is meus frāter et soror et māter est}.
\stopbookchapter

\startbookchapter %Chapter 4
\startabstract
  Sēmen incientis parabola. Cūr quibusdam per parabolās Chrīstus. %???
  Obscūrī dēclārātiō. Lucernae locus, ūsus. An quid vērē latēns. Relātiō pār.
  Sēminātor ac messor. Sināpis sēmen. Jēsus in tempestātēs.
\stopabstract

\vs%{Mc-4-1}
Rūrsus autem coepit ad lacum docēre.
Tanta‧que ad eum convēnit hominum multitūdō,
utī nāvim ingressus sedēret in lacū,
et omnem multitūdinem, quae apud lacum in terrā erat,
\vs%{Mc-4-2}
multa per similitūdinēs docēret, et inter docendum sīc iīs dīceret:
\vs%{Mc-4-3}
\christquote{%
  Audīte. Profectus est aliquandō quīdam sator ad serendum.
\vs%{Mc-4-4}
  Atque inter serendum adcidit, ut aliud caderet apud viam,
  id quod comēdērunt āereae volucrēs quae supervēnērunt.
\vs%{Mc-4-5}
  Aliud cecidit in saxētum, ubī̆ nōn multam habēbat humum,
  id‧que statim exortum est, quod profundam terram nōn habēret;
\vs%{Mc-4-6}
  deinde ortō sōle, aestuāvit et, quoniam rādīcem nōn habēbat, āruit.
\vs%{Mc-4-7}
  Aliud cecidit in spīnās; quae spīnae crēvērunt id‧que subfōcārunt,
  ita utī frūctum nōn ēdiderit.
\vs%{Mc-4-8}
  Aliud cecidit in bonam terram, ēdidit‧que frūctum:
  quī tantō incrēmentō auctus est, ut extulerit partim cum trīcēsimō,
  partim cum sexāgēsimō, partim cum centēsimō.
\vs%{Mc-4-9}
  Quī aur|īs habet ad audiendum, ― inquit iīs ― audiat}.

\vs%{Mc-4-10}
Quum autem seorsum esset,
interrogārunt eum suī cum \DUODECIM dē eā similitūdine.
\vs%{Mc-4-11}
Quibus ille:
\christquote{%
  Vōbīs datum est ― inquit ― \RĒGNĪ\ \DEĪ arcānum nōsse;
  at cum illīs exterīs omnia per similitūdinēs aguntur,
  \startlines%
\vs%{Mc-4-12}
  \bibleallusion{%
    ut adspiciant neque videant,
    et audiant nec intellegant,
    nē ad frūgem redeant,
    atque ita iīs ignōscantur peccāta}.
  \stoplines

\vs%{Mc-4-13}
  An nescītis hanc similitūdinem, ― inquit iīs ―
  quō pactō vērō similitūdinēs omn|īs cognōscētis?
\vs%{Mc-4-14}
  Sator sermōnem serit.
\vs%{Mc-4-15}
  Secundum viam autem ubī̆ seritur sermō, iī sunt quī simul ac audīvērunt,
  venit Sātānās tollit‧que satum in eōrum animīs sermōnem.
\vs%{Mc-4-16}
  Similiter, quī in saxētum seruntur, iī sunt quī, quum sermōnem audiunt,
  eum prōtinus adcipiunt cum laetitiā;
\vs%{Mc-4-17}
  vērum nōn habent in sēsē rādīcem, sed temporāriī sunt;
  deinde ortā propter sermōnem calamitāte aut īnsectātiōne, cito dimoventur.
\vs%{Mc-4-18}
  Quī autem in spīnās seruntur, iī sunt quī sermōnem audiunt;
\vs%{Mc-4-19}
  sed superveniunt hujus vītae cūrae et dīvitiārum fallācia
  reliquārum‧que rērum cupiditātēs, quae sermōnem ita subfōcant,
  utī sit īnfrūctuōsus.
\vs%{Mc-4-20}
  At in bonam terram satī, iī sunt quī sermōnem et audiunt et admittunt,
  et frūctum ferunt partim trīcēcuplum, partim sexāgēcuplum, partim centuplum}.
%???

\vs%{Mc-4-21}
Item hoc dīxit iīs:
\christquote{%
  Numquid adcenditur lucerna, utī sub modiō pōnātur aut sub lectō?
  Nōn ut in candēlābrō pōnātur?
\vs%{Mc-4-22}
  Nihil enim est tam obcultum, quod nōn sit patefaciendum;
  neque quidquam tam arcānē sit, quod nōn palam futūrum sit.
\vs%{Mc-4-23}
  Sī quis aur|īs ad audiendum habet, audiat.

\vs%{Mc-4-24}
  Vidēte quid audiātis: ― inquit iīs ― quā mēnsūrā mētiēminī,
  mētiendum vōbīs est et vōbīs addētur audientibus.
\vs%{Mc-4-25}
  Nam quī habet, eī dabitur; quī vērō nōn habet, etiam quod habet ēripiētur eī}.

\vs%{Mc-4-26}
Deinde dīcēbat:
\christquote{%
  Perinde est caeleste \RĒGNUM, acsī quis jaciat humī sēmen;
\vs%{Mc-4-27}
  deinde dormiat atque surgat noctēs et diēs,
  sēmen‧que germinet et grandēscat, eō nesciente.
\vs%{Mc-4-28}
  Nam terra sponte suā parit prīmum herbam, deinde spīcam,
  deinde plēnum in spīcā frūmentum.
\vs%{Mc-4-29}
  Simul ac autem frūctus adolēvit, ille falcem inmittit, quoniam messis adest}.

\vs%{Mc-4-30}
Item dīcēbat:
\christquote{%
  Cui adsimilābimus \RĒGNUM\ \DEĪ aut quā id comparātiōne comparābimus?
\vs%{Mc-4-31}
  Perinde est ac grānum sināpī quod, quum in terrā satum est,
  est quidem omnium terrestrium sēminum vel minimum;
\vs%{Mc-4-32}
  sed quum est satum,
  crēscit fit‧que omnium maximum holerum tantōs‧que rāmōs prōcreat,
  utī possint sub ejus umbrā āereae volucrēs nīdulārī}.

\vs%{Mc-4-33}
Hujusmodī similitūdinibus multīs ad illōs verba faciēbat, prout audīre poterant;
\vs%{Mc-4-34}
nec eōs absque similitūdine adloquēbātur,
sed seorsum discipulīs suīs explicābat omnia.

\vs%{Mc-4-35}
Dīxit autem iīs illō diē sub vesperum: \christquote{Trājiciāmus}.
\vs%{Mc-4-36}
Itaque illī, omissā hominum multitūdine, eum admīsērunt, ut erat in nāve;
quum quidem aliae quoque nāviculae essent ūnā cum eō.
\vs%{Mc-4-37}
Exstitit autem tanta ventī procella, utī flūctūs adeō nāvim invāderent,
ut ea jam replērētur,
\vs%{Mc-4-38}
quum esset ipse in puppī super cervīcālī dormiēns.
Igitur illī eum excitant, et:
\quote{Magister, ― inquiunt ― nihil‧ne hoc cūrās quod perīmus?}.
\vs%{Mc-4-39}
At ille expergēfactus ventum objūrgāvit et lacuī dīxit:
\christquote{Tacē, obmūtēsce!}.
Itaque, sēdātō ventō, magna tranquillitās exstitit.
\vs%{Mc-4-40}
Et ille:
\christquote{%
  Adeō‧ne timidī estis; ― inquit iīs ―
  quī fit, utī fīdūciam nōn habeātis?}.
\vs%{Mc-4-41}
Illī vērō ingentī metū perculsī, dīcere inter sēsē:
\quote{Et quis est hic, ut eī ventus et lacus oboediant?}.
\stopbookchapter

\startbookchapter %Chapter 5
\startabstract
  Daemoniacus ā legiōne. Dēmersī suēs. Jēsus dīmissus.
  Persōnātī obficium grātī. Fīlia redivīva. Prōfluvium vetus.
\stopabstract

\vs%{Mc-5-1}
Vēnērunt autem trāns lacum in agrum Gadarēnōrum.
\vs%{Mc-5-2}
Eī‧que, simul ac ex nāve ēgressus est,
obcurrit ex monumentīs homō inpūrum habēns spīritum,
\vs%{Mc-5-3}
in monumentīs dēgere solitus; quem nē catēnīs quidem poterat vincere quisquam,
\vs%{Mc-5-4}
quippe ā quō saepe compedibus catēnīs‧que vīnctō, discerptae fuerant catēnae,
et compedēs conminūtae, nec eum quisquam domāre potuerat;
\vs%{Mc-5-5}
semper autem, noctēs diēs‧que in montibus et sepulcrētīs agēns, clāmābat
et sē ipsum lapidibus contundēbat.
\vs%{Mc-5-6}
Is vīsō procul Jēsū adcurrit eī‧que honōrem praebuit
\vs%{Mc-5-7}
et ingentī vōce clāmāns, inquit:
\quote{%
  Quid tibi mēcum reī est, Jēsū, suprēmī \DEĪ\ \FĪLĪ?
  Obtestor tē per \DEUM, nē mē torqueās}.
\vs%{Mc-5-8}
Ille enim eī dīcēbat: \christquote{Exī, inpūre spīritus, ex homine},
\vs%{Mc-5-9}
eum‧que interrogābat \indirectchristquote{quod eī nōmen esset}.
Cui ille respondit: \quote{Legiō mihi nōmen est, quoniam multī sumus}
\vs%{Mc-5-10}
eum‧que multum ōrābat, nē eōs extrā eam regiōnem mitteret.

\vs%{Mc-5-11}
Erat autem illīc apud mont|īs porcōrum grex ingēns quī pāscēbat.
\vs%{Mc-5-12}
Igitur eum rogārunt daemonēs omnēs,
utī \indirectquote{sē in porcōs mitteret, ut in eōs ingrederentur}.
\vs%{Mc-5-13}
Id quod simul ac iīs permīsit Jēsus, ēgressī spīritūs inpūrī
ingressī sunt in porcōs. Atque grex per praeceps in lacum dēcurrit
--- erant autem circiter duo mīlia --- in‧que lacū subfōcātī sunt.
\vs%{Mc-5-14}
Subulcī autem fūgērunt, id‧que et in urbe et per rūra nūntiārunt.
Itaque profectī hominēs ad videndum quod adcidisset
\vs%{Mc-5-15}
veniunt ad Jēsum, spectant‧que furiōsum illum sedentem vestītum‧que
et sānae mentis quī legiōnem habuerat, timuērunt‧que. %???
\vs%{Mc-5-16}
Quum‧que iīs nārrāssent quī vīderant, quid adcidisset furiōsō
item‧que dē porcīs,
\vs%{Mc-5-17}
coepērunt eum ōrāre, ut ex suīs fīnibus discēderet.
\vs%{Mc-5-18}
Eum autem in nāvim ingressum, rogābat quī furiōsus fuerat, ut esset eī comes.
\vs%{Mc-5-19}
Vērum Jēsus nōn permīsit eī, sed dīxit:
\christquote{%
  Abī domum tuam ad tuōs iīs‧que nūntiātō quae tibi \DOMINUS,
  tuī miseritus, fēcerit}.
\vs%{Mc-5-20}
Ita ille discessit coepit‧que per Decapolin praedicāre
quae sibi fēcisset Jēsus; id quod mīrābantur omnēs.

\vs%{Mc-5-21}
Jēsū autem, rūrsus in nāve trājectō, convēnit ad eum frequēns hominum multitūdō.
\vs%{Mc-5-22}
Quum‧que esset apud lacum, vēnit quīdam magistrōrum conlēgiī nōmine Jaīrus
quī, eō vīsō, eī ad pedēs adcidit
\vs%{Mc-5-23}
eum‧que multum ōrāvit, dīcēns
\indirectquote{fīliolam suam in extrēmō esse perīculō;
  venīret, eī‧que manūs inpōneret, utī servārētur ac revalēsceret}.
\vs%{Mc-5-24}
Ita ille cum eō abiit, tantā eum sequente hominum multitūdine, ut eum premerent.

\vs%{Mc-5-25}
Quaedam autem mulier, quae sanguinis prōfluviō duodecim jam annōs labōrābat
\vs%{Mc-5-26}
et multa ā multīs passa medicīs, omnī suā rē cōnsūmptā,
nōn sōlum nihil prōfēcerat, sed in dēterius vēnerat,
\vs%{Mc-5-27}
quum dē Jēsū audīvisset, vēnit ā tergō inter turbam ejus‧que vestīmentum tetigit:
\vs%{Mc-5-28}
cōgitābat enim, \indirectquote{sī vel ejus vest|īs adtingeret, sē salvam fore}.
\vs%{Mc-5-29}
Et prōtinus āruit ejus fōns sanguinis,
sēnsit‧que sē corpore esse ab eā miseriā sānātam.
\vs%{Mc-5-30}
Et Jēsus, statim apud sēsē vim ā sē profectam cognōscēns,
conversus in turbam dīxit: \christquote{Quis mea vestīmenta tetigit?}.
% orig. in turbā, probably an error
\vs%{Mc-5-31}
Cui ejus discipulī: \quote{Vidēs tē turbā premī et quaeris, quis tē tetigerit?}.
\vs%{Mc-5-32}
Sed quum ille circumspiceret, ut eam vidēret, quae id fēcerat,
\vs%{Mc-5-33}
territa mulier et tremebunda, sciēns quid sibi adcidisset, vēnit et eī adcidit,
rem‧que omnem, utī sēsē habēbat, dīxit.
\vs%{Mc-5-34}
Et ille:
\christquote{%
  Fīlia, ― inquit eī ― tua tē fīdūcia servāvit.
  Abī salva et estō ā tuō malō sāna}.

\vs%{Mc-5-35}
Adhūc eō loquente, veniunt ā magistrō conlēgiī quī eī dīcunt:
\quote{Tua fīlia mortua est; quid jam molestus es magistrō?}.
\vs%{Mc-5-36}
At Jēsus, simul ac quid dīcerētur audīvit, dīxit magistrō conlēgiī:
\christquote{Nē metue; tantum crēde!}.
\vs%{Mc-5-37}
Nōn sīvit autem quemquam sē subsequī,
praeter Petrum et Jācōbum et Jōhannem Jācōbī frātrem.
\vs%{Mc-5-38}
Deinde, ubī̆ vēnit in aed|īs magistrī conlēgiī, videt tumultum
et plōrant|īs multum‧que ejulant|īs,
\vs%{Mc-5-39}
ingressus‧que dīcit iīs:
\christquote{Quid tumultuāminī et plōrātis? Puella nōn est mortua, sed dormit}.
\vs%{Mc-5-40}
At illī eum dērīdēre. Ipse, ējectīs omnibus, adsūmptō puellae patre atque mātre
et suīs comitibus, ingreditur eō ubī̆ jacēbat puella;
\vs%{Mc-5-41}
ejus‧que manū prehēnsā, dīcit eī: \christquote{Talīthā, cūmī!},
quae verba sīc sonant: \christquote{Puella, inquam tibi: surge!}.
\vs%{Mc-5-42}
Statim‧que subrēxit puella, et ambulāre coepit; erat autem annōrum duodecim.
Unde illī vehementer adtonitī sunt.
\vs%{Mc-5-43}
Praecēpit autem iīs Jēsus magnopere, nē quis id rescīsceret,
eī‧que cibum darī jussit.
\stopbookchapter

\startbookchapter %Chapter 6
\startabstract
  In patriā prophēta.
  Apostolōrum missiō, virtūtēs, habitus, egestās, mandāta, opera.
  Dē Jēsū sententiae. Jōhannis vincula, mors, humātiō.
  Pānēs quīnque, ac duo piscēs. Malacia, alia.
\stopabstract

\vs%{Mc-6-1}
Illinc deinde profectus vēnit in suam patriam, sequentibus eum ejus discipulīs.
\vs%{Mc-6-2}
Atque ubī̆ advēnit Sabbatum, coepit in conlēgiō docēre,
ita utī multī audientēs adtonitī dīcerent:
\quote{%
  Unde eī haec? Et quae est quae eī data est sapientia,
  et quae fīunt ejus manibus mīrācula?
\vs%{Mc-6-3}
  An nōn is est faber, Marīae fīlius, Jācōbī et Jōsētos et Jūdae
  et Simōnis frāter, et nōnne sunt hīc apud nōs ejus sorōrēs?}.
Itaque in eō obfendēbantur.
\vs%{Mc-6-4}
Et Jēsus dīcēbat iīs,
\indirectchristquote{nōn esse vātem inhonōrātum nisi in suā patriā,
et apud cōnsanguineōs domī‧que suae}.
\vs%{Mc-6-5}
Itaque nōn potuit illīc ūllum facere mīrāculum, nisi quod paucōs aegrōtōs
inpositīs manibus sānāvit,
\vs%{Mc-6-6}
admīrātus eōrum \unclassical{īnfidentiam}.

Obībat autem vīcōs undique fīnitimōs docēns.

\vs%{Mc-6-7}
Vocātōs‧que \DUODECIM coepit bīnōs dīmittere
iīs‧que in spīritūs inpūrōs potestātem dedit;
\vs%{Mc-6-8}
praecēpit‧que, nē quid ad iter circumferrent nisi baculum tantum:
nōn pēram, nōn pānem, nōn aes in crumīnā,
\vs%{Mc-6-9}
sed sandaliīs calceārentur neque bīnās induerent tunicās.
\vs%{Mc-6-10}
Item iīs dīxit:
\christquote{%
  Ubicumque domum ingressī fueritis,
  illīc manētōte, dōnec inde migrētis.
\vs%{Mc-6-11}
  Et quīcumque vōs nōn admīserint nec audīverint,
  vōs illinc proficīscentēs excutite pulverem, quī suberit vestrīs pedibus,
  quod sit iīs testimōniō.
  \obsolete{Hoc vōbīs cōnfirmō:
  mītius āctum īrī cum Sodomīs et Gomorrhīs in diē jūdiciī,
  quam cum illā cīvitāte}}.

\vs%{Mc-6-12}
Igitur illī profectī, mōrēs conrigere pūblicē docēbant;
\vs%{Mc-6-13}
multa‧que daemonia ējiciēbant et multōs aegrōtōs oleō ungēbant ac sānābant.

\vs%{Mc-6-14} %quī baptīzāre solitus fuisset
Id quum audīvisset rēx Hērōdēs, --- nam celebre factum erat ejus nōmen --- dīcēbat
\indirectquote{Jōhannem, quī baptīzāre solitus fuerat, % FUI + PARTICIPIUM
ex mortuīs subrēxisse, ideō‧que eum tam esse factīs potentem}.
%DUBIOUS: ideōque %or ideō‧que
\vs%{Mc-6-15}
Aliī dīcēbant \indirectquote{Ēliān esse},
aliī \indirectquote{vātem esse, aut ut aliquem vātum}.
\vs%{Mc-6-16}
Quō audītō, Hērōdēs dīcēbat
\indirectquote{quem ipse dēcollāsset Jōhannem eum esse, ex mortuīs subrēxisse}.

\vs%{Mc-6-17}
Nam ipse Hērōdēs Jōhannem capiendum cūrāverat et in cūstōdiā vinciendum
propter Hērōdiada Philippī suī frātris uxōrem, quam ipse dūxerat in mātrimōnium.
\vs%{Mc-6-18}
Dīcēbat enim Jōhannēs Hērōdī,
\indirectquote{nōn licēre eī uxōrem habēre suī frātris}.
\vs%{Mc-6-19}
Hērōdias autem inminēbat Jōhannī eum‧que interficere volēbat; sed nōn poterat,
\vs%{Mc-6-20}
proptereā quod Hērōdēs Jōhannem verēbātur,
sciēns virum esse jūstum atque sānctum, eum‧que observābat
eī‧que multīs in rēbus auscultābat et eum libenter audiēbat.
\vs%{Mc-6-21}
Adcidit autem obportūna diēs,
quā Hērōdēs optimātibus tribūnīs‧que et prīmātibus Galilaeae nātālicia dabat.
\vs%{Mc-6-22}
Et ingressa ipsī̆us Hērōdiados fīlia saltāvit,
Hērōdī‧que et convīvīs ita placuit, utī dīxerit rēx puellae:
\quote{Posce mē quidquid volēs, tibi dabō},
\vs%{Mc-6-23}
eī‧que sīc jūrāvit:
\quote{Quidquid ā mē petieris, id tibi dabō: etiamsī sit rēgnī meī dīmidium}.
\vs%{Mc-6-24}
At illa ēgressa, cōnsuluit suam mātrem,
\indirectquote{ecquid peteret}.
Et māter jussit, utī \indirectquote{Jōhannis Baptistae caput}.
\vs%{Mc-6-25}
Itaque illa statim, festīnanter ad rēgem ingressa, petiit hīs verbīs:
\quote{Volō utī mihi dēs prōtinus in catīnō caput Jōhannis Baptistae}.
\vs%{Mc-6-26}
Id quod dolēns rēx, tamen propter jūsjūrandum et convīvās nōluit eī dēnegāre;
\vs%{Mc-6-27}
mīsit‧que cōnfestim rēs speculātōrem et eī, ut illī̆us caput adferret, inperāvit.
Ille eum in carcere dēcollātum iit
\vs%{Mc-6-28}
et ejus caput in catīnō adtulit ac puellae trādidit,
quod puella porrō suae mātrī dedit.
\vs%{Mc-6-29}
Hōc audītō, Jōhannis discipulī vēnērunt sublātum ejus corpus,
id‧que in monumentō posuērunt.

\vs%{Mc-6-30}
Convēnērunt autem ad Jēsum \APOSTOLĪ eī‧que omnia,
quaeque fēcerant quaeque docuerant, renūntiāvērunt.
\vs%{Mc-6-31}
Et Jēsus jussit iīs, ut
\indirectchristquote{adessent ipsī seorsum in dēsertum locum
paululum‧que requiēscerent}.
Erant enim quī veniēbant et quī abībant tam multī,
ut iīs nē cibī quidem capiendī ōtium subpeteret.
\vs%{Mc-6-32}
Ita discessit in dēsertum locum in nāve seorsum.

\vs%{Mc-6-33}
Quōs quum vulgus abīre vidēret, agnōvērunt eum multī
et illūc pedestrēs ex omnibus urbibus concurrērunt,
eōs‧que adgressī ad eum convēnērunt.
\vs%{Mc-6-34}
Itaque ēgressus Jēsus, tantam cōnspicātus multitūdinem, miseritus est eōrum,
quod erant perinde ac pāstōrem nōn habentēs ovēs,
eōs‧que multa docēre adgressus est.
\vs%{Mc-6-35}
Postquam autem jam sēra vēnit hōra,
ejus discipulī eum hujusmodī verbīs sunt adgressī:
\quote{%
  Dēsertus est hic locus et jam sēra hōra;
\vs%{Mc-6-36}
  dīmitte eōs, ut in fīnitimōs pāgōs vīcōs‧que profectī pān|īs sibi compārent%
  \obsolete{; nam quod comedant nōn habent}}.
\vs%{Mc-6-37}
Et Jēsus iīs respondēns:
\christquote{Praebēte vōs iīs cibum}, inquit.
Cui illī:
\quote{Eāmus ēmptum ducentīs dēnāriīs pān|īs, quibus eōs pāscāmus?}.
\vs%{Mc-6-38}
Et ille: \christquote{Quot pān|īs habētis? ― inquit iīs ― Īte vīsum}.
Quō cognitō, quum \indirectquote{quīnque et duōs pisc|īs} dīxissent,
\vs%{Mc-6-39}
jussit illōs omn|īs discumbere catervātim in viridī grāmine.
\vs%{Mc-6-40}
Ac, postquam in class|īs distribūtī centēnī et quīnquāgēnī discubuērunt,
\vs%{Mc-6-41}
ille, sūmptīs quīnque pānibus et duōbus piscibus, caelum suspiciēns
laudēs ēgit frēgit‧que pān|īs et suīs discipulīs trādidit,
ut illīs adpōnerent, duōs‧que pisc|īs distribuit omnibus.
\vs%{Mc-6-42}
Comēdērunt‧que cūnctī ad satietātem;
\vs%{Mc-6-43}
et sublāta sunt fragmentōrum duodecim canistra plēna, necnōn ex piscibus;
\vs%{Mc-6-44}
Et erant, quī comēderant, virōrum circiter quīnque mīlia.

\vs%{Mc-6-45}
Deinde cōnfestim coēgit discipulōs suōs nāvim cōnscendere
et ante trājicere ad Bēthsaidam, dum ipse turbam illam dīmitteret.
\vs%{Mc-6-46}
Postquam illīs valē dīxit, discessit in montem ōrandī grātiā.
\vs%{Mc-6-47}
Sub vesperum quum esset nāvis in mediō lacū, et ipse sōlus in terrā,
\vs%{Mc-6-48}
vīdit eōs in rēmigandō vexārī, quippe quum ventus esset iīs contrārius.
Deinde circā quārtam noctis vigiliam, vēnit ad eōs per lacum ingrediēns
eōs‧que volēbat praeterīre.
\vs%{Mc-6-49}
At illī eum lacū ambulantem cōnspicātī, et spectrum esse ratī, exclāmāre:
\vs%{Mc-6-50}
omnēs enim eum vīdērunt turbātī‧que sunt. Tum ille continuō eōs adlocūtus:
\christquote{Bonō este animō, ― inquit ― ego sum; nē timēte!}.
\vs%{Mc-6-51}
Ad eōs‧que ascendit in nāvim et ventus posuit,
id quod illī suprā modum adtonitī apud sēsē mīrārī:
\vs%{Mc-6-52}
nōn enim cōnsīderābant illōs pān|īs,
utpote quī torpentibus essent animīs.

\vs%{Mc-6-53}
Postquam trājēcērunt, vēnērunt in agrum Genezarethānum.
\vs%{Mc-6-54}
Ac simul ac adpulsī ex nāve ēgressī sunt, incolae, eō cognitō,
\vs%{Mc-6-55}
tōtam illam fīnitimam regiōnem percurrērunt
coepērunt‧que in grabātīs male adfectōs eō adferre, ubī̆ eum esse audiēbant.
\vs%{Mc-6-56}
Ac quōcumque intrābat in vīcōs, in urb|īs aut pāgōs,
prōpōnēbant per fora aegrōtant|īs, eum‧que rogābant,
utī vel ejus togae limbum tangerent;
ac quīcumque eum tangēbant sānābantur.
\stopbookchapter

\startbookchapter %Chapter 7
\startabstract
  Prō discipulīs inlōtīs. Pharīsaeōrum increpātiō. Nōs foedantia.
  Syrophoenissae gnāta, fidēs, grātia.
\stopabstract

\vs%{Mc-7-1}
Convēnērunt autem ad eum Pharīsaeī et quīdam scrībārum,
quī Hierosolymīs vēnerant.
\vs%{Mc-7-2}
Quī, quum vīdissent quōsdam ejus discipulōrum profānīs,
hoc est, inlōtīs manibus cibum capere, reprehendērunt:
\vs%{Mc-7-3}
nam Pharīsaeī Jūdaeī‧que omnēs nōn, nisi saepe lōtīs manibus,
epulantur, tenentēs veterum īnstitūtum;
\vs%{Mc-7-4}
item ā forō nōn, nisi lōtī, epulantur; alia‧que multa sunt,
quae tenenda adcēpērunt:
pōculōrum lōtiōnēs et urceōrum et aerāmentōrum et lectōrum.
\vs%{Mc-7-5}
Deinde sīc eum interrogant Pharīsaeī et scrībae:
\quote{%
  Cūr nōn agunt tuī discipulī, utī veterum postulat īnstitūtiō,
  sed inlōtīs manibus cibum capiunt?}.
\vs%{Mc-7-6}
Quibus ille sīc respondit:
\christquote{%
  Rēctē dē vōbīs simulātōribus vāticinātus est Ēsaiās,
  utī scrīptum est:
  \startlines\subbiblequote{%
    Hic populus mē labiīs honōrat,
    quum eōrum cor absit ā mē procul;
\vs%{Mc-7-7}
    vērum frūstrā mē venerantur,
    docentēs doctrīnās quae sunt hominum praecepta}.
   \stoplines

\vs%{Mc-7-8}
  Nam, relictō \DEĪ praeceptō, tenētis hominum īnstitūtum%
  \obsolete{; urceōrum pōculōrum‧que lōtiōnēs
            et alia hīs cōnsimilia multa facitis}.
\vs%{Mc-7-9}
  Probē ― inquit iīs ― \DEĪ praeceptum antīquātis,
  utī vestram cōnservētis īnstitūtiōnem.
\vs%{Mc-7-10}
  Mōysēs enim dīxit:
  \subbiblequote{Honōrā tuum patrem atque mātrem}, et:
  \subbiblequote{Quī patrī mātrī‧ve maledīxerit, morte plectātur}.
\vs%{Mc-7-11}
  At vōs dīcitis:
  \subquote{Sī quis patrī mātrī‧ve dīxerit: \implication{Corbān}, hoc est:
  \explication{dōnārium}, quod ā mē proficīscētur, id tibi prōderit}.
\vs%{Mc-7-12}
  Ita nōn sinitis eum jam quidquam suō patrī mātrī‧ve praestāre,
\vs%{Mc-7-13}
  \DEĪ dictum istā, quam trādidistis, īnstitūtiōne rescindentēs;
  atque hujuscemodī multa facitis}.

\vs%{Mc-7-14}
Tum advocātā omnī multitūdine, dīxit iīs:
\christquote{%
  Audīte mē, omnēs, et intellegite:
\vs%{Mc-7-15}
  nihil est extrā hominem quod in eum intret,
  quod eum possit inquināre;
  sed quae ex eō prōdeunt, ea sunt quae hominem polluunt!
\vs%{Mc-7-16}
  \obsolete{Sī quis aur|īs habet ad audiendum, audiat}}.

\vs%{Mc-7-17}
Ubī̆ autem ā turbā domum ingressus est,
interrogārunt eum ejus discipulī dē illā similitūdine.
\vs%{Mc-7-18}
Quibus ille:
\christquote{%
  Adeō‧ne vōs quoque tardī estis? An nōndum intellegitis nihil,
  quod extrīnsecus in hominem intrat, posse eum coinquināre?
\vs%{Mc-7-19}
  Utpote, quum nōn in ejus cor sed in ventrem et in lātrīnam ferātur,
  expūrgāns omn|īs cibōs.
\vs%{Mc-7-20}
  Sed quod ex homine exit ― inquit ― polluit hominem:
\vs%{Mc-7-21}
  internē enim ex hominum corde prōdeunt
  prāvae cōgitātiōnēs, adulteria, stupra, caedēs,
\vs%{Mc-7-22}
  fūrta, avāritiae, malitiae, fraus, libīdō,
  malus oculus, maledicta, superbia, dēmentia;
\vs%{Mc-7-23}
  haec omnia mala intrīnsecus prōdeunt et hominem polluunt}.

\vs%{Mc-7-24}
Illinc profectus, discessit in Tyrī Sīdōnis‧que cōnfīnia.
Ingressus‧que domum nōlēbat id scīre quemquam, sed nōn potuit latēre.
\vs%{Mc-7-25}
Nam quum dē eō audīvisset quaedam mulier, cujus fīlia spīritum habēbat inpūrum,
vēnit et eī ad pedēs adcidit.
\vs%{Mc-7-26}
Erat autem mulier Graeca, genere Syrophoenissā.
Ea rogābat eum, utī daemonium expelleret ex suā fīliā.
\vs%{Mc-7-27}
At Jēsus dīxit eī:
\christquote{%
  Sine prius satiārī gnātōs:
  nōn enim convenit sūmere gnātōrum pānem et catellīs objicere}.
\vs%{Mc-7-28}
Et illa: \quote{%
  Etsī ita est, \DOMINE, ― inquit eī respondēns ―
  tamen catellī sub mēnsā comedunt dē puerōrum mīcīs}.
\vs%{Mc-7-29}
Tum ille: \christquote{%
  Propter istud dictum ― inquit eī ― abī;
  migrāvit daemonium ex tuā fīliā}.
\vs%{Mc-7-30}
Itaque illa domum suam profecta, invēnit ēgressum daemonium
et jacentem in lectō fīliam.

\vs%{Mc-7-31}
Rūrsum profectus ex Tyriīs Sīdōnis‧que fīnibus vēnit ad Galilaeae lacum
per mediōs fīn|īs Decapoleōs.
\vs%{Mc-7-32}
Eī‧que adductus est surdus tardiloquus, cui, utī manum inpōneret, rogātus est.
\vs%{Mc-7-33}
Atque ille eō seorsum extrā turbam sēductō, inmīsit suōs digitōs in ejus aurīs
spuit‧que et ejus linguam tetigit
\vs%{Mc-7-34}
et, in caelum intuēns, ingemuit eī‧que dīxit:
\christquote{Ephphatha}, hoc est: \christquote{Aperītor}.
\vs%{Mc-7-35}
Ac prōtinus illī̆us et aurēs apertae et solūtum est linguae vinculum,
ita utī rēctē loquerētur.
\vs%{Mc-7-36}
Praecēpit autem iīs, ut id nēminī dīcerent;
sed quō magis iīs praecipiēbat, eō magis praedicābant
\vs%{Mc-7-37}
et majōrem in modum obstupēscēbant, dīcentēs
\indirectquote{eum omnia rēctē fēcisse, utī surdī audīrent,
et utī mūtī loquerentur, ecficere}.
\stopbookchapter

\startbookchapter %Chapter 8
\startabstract
  Septem pānēs. Ostentōrum cupidī. Fermentum Pharsīaeōrum. Caecus.
  Petrus dē Chrīstō. Jēsū mors praedicta. Petrus Sātānās, piōrum crux.
\stopabstract

\vs%{Mc-8-1}
Per diēs erat permagna hominum multitūdō nec habēbant, quod comēssent,
et Jēsus, advocātīs suīs discipulīs, inquit:
\vs%{Mc-8-2}
\christquote{%
  Miseret mē hujus multitūdinis,
  sī quidem jam tr|īs diēs mē comitantur nec habent, quod comedant;
\vs%{Mc-8-3}
  quodsī eōs domum jejūnōs dīmīserō, dēfetīscentur in viā:
  nam eōrum nōnnūllī procul vēnērunt}.
\vs%{Mc-8-4}
Et discipulī ejus sīc respondērunt:
\quote{Unde possit eōs quisquam pānibus hīc in sōlitūdine satiāre?}.
\vs%{Mc-8-5}
At ille eōs interrogāvit:
\christquote{Quot habētis pān|īs?}.
\quote{Septem}, inquiunt.
\vs%{Mc-8-6}
Tum ille jussit multitūdinem humī discumbere;
sūmptōs‧que septem pān|īs, āctīs grātiīs, frēgit suīs‧que trādidit discipulīs,
ut adpōnerent;
quōs illī adposuērunt multitūdinī.
\vs%{Mc-8-7}
Quum‧que paucōs habērent pisciculōs;
ille, āctīs laudibus, jussit eōs quoque adpōnī.
\vs%{Mc-8-8}
Illī vērō epulātī sunt ad satietātem;
sunt‧que sublātae reliquōrum fragmentōrum sportae septem.
\vs%{Mc-8-9}
Erant autem, quī comēdērunt, circiter quattuor mīlia.

\vs%{Mc-8-10}
Quibus ille dīmissīs, cōnfestim cōnscēnsā nāve cum suīs discipulīs,
vēnit in tractūs Dalmanūthānōs.
\vs%{Mc-8-11}
Eō profectī Pharīsaeī coepērunt cum eō contendere,
postulantēs ab eō signum dē caelō, ejus tentandī grātia.
\vs%{Mc-8-12}
At ille animō ingemuit et dīxit:
\christquote{%
  Quid signum quaerit haec nātiō?
  Prō certō habētōte: nūllum huic nātiōnī signum datum īrī}.
\vs%{Mc-8-13}
Deinde, relictīs iīs, rūrsum, cōnscēnsā nāve, trājēcit.

\vs%{Mc-8-14}
Oblītī autem erant pān|īs sūmere nec nisi ūnum pānem sēcum habēbant in nāve.
\vs%{Mc-8-15}
Itaque quum ille iīs praeciperet,
\indirectquote{vidērent utī Pharīsaeōrum Hērōdīs‧que fermentum cavērent},
\vs%{Mc-8-16}
illī inter sēsē cōgitābant ac dīcēbant id eō pertinēre,
quod nōn habērent pān|īs.
\vs%{Mc-8-17}
Quō cognitō, Jēsus iīs inquit:
\christquote{%
  Quid cōgitātis? Vōs nōn habēre pān|īs?
  An nōndum animadvertitis nec intellegitis? Adhūc torpentia habētis corda?
\vs%{Mc-8-18}
  \bibleallusion{Oculōs habentēs nōn vidētis, et aur|īs habentēs nōn audītis?}
  Neque meministis,
\vs%{Mc-8-19}
  quum illōs quīnque pān|īs ad hominum quīnque mīlia frēgī,
  quot canistra plēna fragmentōrum sustulerītis?}.
\quote{Duodecim}, inquiunt eī.
\vs%{Mc-8-20}
\christquote{%
  Quid quum septem ad quattuor mīlia?
  Quot plēnās fragmentōrum sportās sustulistis?}
\quote{Septem}, inquiunt.
\vs%{Mc-8-21}
Et ille: \christquote{Quī fit, ― inquit iīs ― utī nōn animadvertātis?}.

\vs%{Mc-8-22}
Deinde ubī̆ Bēthsaidam vēnit, adlātus est eī caecus,
quem utī tangeret, rogātus est.
\vs%{Mc-8-23}
Et ille, prehēnsā caecī manū, eum ēdūxit ex vīcō, et ejus oculōs cōnspuit
eī‧que manūs inposuit et eum interrogāvit, \indirectchristquote{ecquid vidēret}.
\vs%{Mc-8-24}
Et ille intuitus, dīxit:
\quote{Videō hominēs: nam arborum rītū videō ambulantēs}.
\vs%{Mc-8-25}
Deinde rūrsus ille ejus oculīs manū inposuit ecfēcit‧que, utī vidēret;
ita, utī restitūtus, omn|īs perspicuē discerneret.
\vs%{Mc-8-26}
Atque ille eum domum dīmīsit
\indirectquote{vetāns, nē vel in vīcum ingrederētur%
\obsolete{, vel id cuiquam in vīcō dīceret}}.

\vs%{Mc-8-27}
Prōfectus est autem Jēsus ejus‧que discipulī in vīcōs Caesarēae Philippī,
atque in viā suōs discipulōs hujusmodī verbīs interrogāvit:
\christquote{Quem mē dīcunt hominēs esse?}.
\vs%{Mc-8-28}
Cui illī respondērunt,
\indirectquote{Jōhannem Baptistam, aliī Ēliān, aliī vātem aliquem}.
\vs%{Mc-8-29}
Et ille:
\christquote{Vōs vērō, ― inquit iīs ― quem dīcitis mē esse?}.
Cui Petrus ita respondit:
\quote{Tū es Chrīstus}.
\vs%{Mc-8-30}
At ille iīs interdīxit, nē cui sē indicārent.

\vs%{Mc-8-31}
Eōs‧que docēre coepit:
\indirectchristquote{oportēre \FĪLIUM hominis patī multa,
et ā senātōribus et pontificibus et scrībīs inprobārī ac interficī,
et trīduō post resurgere}.
\vs%{Mc-8-31}
Dum ita apertē verba facit, prehendit eum Petrus et objūrgāre coepit.
\vs%{Mc-8-33}
At ille conversus discipulōs‧que suōs intuitus, Petrum hīs verbīs objūrgāvit:
\christquote{Abī post mē, Sātānā, quandōquidem nōn dīvīna, sed hūmāna, sapis}.

\vs%{Mc-8-34}
Tum advocātō vulgō cum suīs discipulīs, dīxit iīs:
\christquote{%
  Quī mē pōne vult sequī,
  renūntiet sibimet ipsī suam‧que crucem tollat et mē sequātur.
\vs%{Mc-8-35}
  Nam quī volet animam suam servāre, eam perdet;
  quī vērō animam suam meā et \ĒVANGELIĪ causā perdiderit, eam servābit.
\vs%{Mc-8-36}
  Quid vērō prōderit hominī, sī vel tōtum mundum lucrātus fuerit,
  et animae suae jactūram faciat?
\vs%{Mc-8-37}
  Aut quam dabit homō animae suae compēnsātiōnem?
\vs%{Mc-8-38}
  Nam quem meī meōrum‧que dictōrum
  in hāc adulterīnā inprobāque nātiōne puduerit,
  ejus vicissim pudēbit \FĪLIUM hominis,
  quum in \PATRIS suī glōriā vēnerit ūnā cum sānctīs angelīs}.
\stopbookchapter

\startbookchapter %Chapter 9
\startabstract
  Chrīstus trānsfigūrātur, caelitus celebrātur.
  Precum ac jejūniī contrā spīritūs virtus. Praedicta mors et exsuscitātiō.
  Piōrum dominātus. Puerōrum beātitās. Quis cum Jēsū. Obfēnsārum poena.
  Salis ac pācis mōmentum.
\stopabstract

\vs%{Mc-9-1}
Tum iīs dīcit:
\christquote{%
  Hoc certō scītōte: esse quōsdam eōrum quī hīc adsunt,
  quī nōn ante mortem sēnsūrī sint,
  quam \DEĪ\ \RĒGNUM videant vēnisse cum potentiā}.

\vs%{Mc-9-2}
Deinde sextō post diē adsūmit Jēsus Petrum et Jācōbum et Jōhannem,
iīs‧que in arduum montem sōlis seorsum ductīs, trānsfigūrātus est cōram iīs;
\vs%{Mc-9-3}
ejus‧que vestīmenta facta sunt fulgentia, tam alba,
quam est nix, quōmodo nūllus in orbe fūllō dealbāre possit;
\vs%{Mc-9-4}
iīs‧que cōnspectus est Ēliās cum Mōyse, quī cum Jēsū conloquēbantur.
\vs%{Mc-9-5}
Tum Petrus sīc Jēsum adlocūtus est:
\quote{%
  Rabbī, bonum est nōs hīc esse; itaque faciāmus tria tabernācula:
  tibi ūnum, Mōysī ūnum, et Ēliae ūnum}.
\vs%{Mc-9-6}
Nesciēbat autem quid dīceret, quippe quum essent territī.
\vs%{Mc-9-7}
Exstitit autem nūbēs, quae illōs inumbrāvit, vēnit‧que vōx ex nūbe dīcēns:
\godquote{Hic est meus cārissimus \FĪLIUS, hunc audīte}.
\vs%{Mc-9-8}
Et subitō circumspicientēs, nūllum jam vīdērunt nisi Jēsum sōlum cum ipsīs.

\vs%{Mc-9-9}
Dēscendentibus autem iīs dē monte, praecēpit iīs, nē cui, quae vīderant,
nārrārent, nisi postquam \FĪLIUS hominis ex mortuīs resurrēxisset.
\vs%{Mc-9-10}
Illī vērō apud sēsē id dictum retinuērunt,
quaerentēs quid esset: \indirectquote{ex mortuīs resurgere}.
\vs%{Mc-9-11}
Itaque eum interrogārunt dē eō quod dīcerent scrībae:
\indirectquote{Ēliān oportēre venīre prius}.
\vs%{Mc-9-12}
Et ille respondēns iīs:
\christquote{%
  Ēliās quidem prius ventūrus et omnia īnstaurātūrus est;
  id‧que quemadmodum dē \FĪLIŌ hominis scrīptum est,
  multa passūrum et contemnendum esse.
\vs%{Mc-9-13}
  Sed scītōte: Ēliān vēnisse, et illōs eī, quae voluerint, fēcisse,
  quemadmodum dē eō scrīptum est}.

\vs%{Mc-9-14}
Deinde ubī̆ vēnit ad discipulōs,
vīdit circā eōs ingentem hominum multitūdinem et cum iīs contendent|īs scrībās.
\vs%{Mc-9-15}
Ac vulgō, simul ac eum vīdērunt admīrātī, eum salūtātum adcurrunt.
\vs%{Mc-9-16}
Ipse scrībās interrogāvit, \indirectchristquote{quid cum illīs contenderent}.
\vs%{Mc-9-17}
Cui ūnus ex turbā respondit in hunc modum:
\quote{%
  Magister, addūxī tibi meum fīlium, quī mūtum habet spīritum;
\vs%{Mc-9-18}
  quī ubī̆ eum conripuit, ita laceret, utī spūmet et dentibus strīdeat ac tābēscat.
  Eum ut ējicerent, petiī ā tuīs discipulīs, sed nequīvērunt}.
\vs%{Mc-9-19}
Tum ille eī respondēns:
\christquote{%
  Ō diffīdēns nātiō, ― inquit ― quoū̆sque tandem apud vōs erō?
  Quōū̆sque tandem vōs feram? Addūcite eum ad mē}.
\vs%{Mc-9-20}
Atque ille ad eum addūcitur; et simul ac ille eum vīdit,
convulsit eum spīritus, ita utī conlāpsus humī volūtārētur spūmāns.
\vs%{Mc-9-21}
Ille ex ejus patre scīscitātus est,
\indirectquote{quamdiū esset quum id eī adcidisset}.
Et pater respondit:
\quote{%
  Jam ā puerīs;
\vs%{Mc-9-22}
  ac saepe eum et in ign|im jēcit et in aquam, ut eum perderet;
  sed sī quid potes, subcurre nōbīs, nostrī miseritus}.
\vs%{Mc-9-23}
Et Jēsus:
\christquote{Sī̆quidem potes ― inquit eī ― fīdere; omnia fī̆erī possunt fīdentī}.
\vs%{Mc-9-24}
Hic prōtinus exclāmāns puerī pater cum lacrimīs:
\quote{Fīdō, ― inquit ― \DOMINE! Subcurre meae diffīdentiae}.
\vs%{Mc-9-25}
Vidēns autem Jēsus vulgus adcurrere, increpuit inpūrum spīritum dīcēns eī:
\christquote{%
  Spīritus mūte et surde, ego tibi inperō,
  ut ex eō ēmigrēs nec amplius in eum intrēs}.
\vs%{Mc-9-26}
Tum ille, exclāmandō eum‧que vehementer laniandō, exiit.
Atque is perinde fuit ac mortuus,
ita utī multī \indirectquote{mortuum esse} dīcerent.
\vs%{Mc-9-27}
At Jēsus eum manū prehēnsum adlevāvit, atque ille subrēxit.

\vs%{Mc-9-28}
Eum autem domum ingressum, discipulī seorsum interrogārunt,
\indirectquote{cūr ipsī eum expellere nequīvissent}.
\vs%{Mc-9-29}
Quibus ille:
\christquote{%
  Hoc genus ― inquit ―
  nūllā rē ējicī potest nisi precibus et jejūniīs}.

\vs%{Mc-9-30}
Illinc profectī iter faciēbant per Galilaeam;
nec ille volēbat, ut id scīret quisquam.
\vs%{Mc-9-31}
Docēbat enim discipulōs suōs, iīs‧que dīcēbat
\indirectchristquote{\FĪLIUM hominis trādendum esse in hominum manūs,
  quī eum interficerent; is‧que interfectus, tertiō diē resurgeret}.
\vs%{Mc-9-32}
At illī eam rem ignōrābant et eum interrogāre verēbantur.

\vs%{Mc-9-33}
Vēnit autem Capharnāum. Atque ubī̆ domum vēnit, quaesīvit ex iīs,
\indirectchristquote{quid in viā inter sēsē disceptāssent}.
\vs%{Mc-9-34}
Quum‧que illī silērent,
--- nam inter sēsē disputāverant in viā, quis esset maximus ---
\vs%{Mc-9-35}
ille cōnsēdit et vocātīs \DUODECIM dīxit:
\christquote{Sī quis prīmus esse vult, erit omnium ultimus omnium‧que famulus}.
\vs%{Mc-9-36}
Tum puerum sūmpsit, et in eōrum mediō statuit, eum‧que complexus ulnīs,
inquit iīs:
\vs%{Mc-9-37}
\christquote{%
  Quī aliquem tālium puerōrum adcipit meō nōmine, mē adcipit;
  et quī mē adcipit, nōn mē adcipit, sed eum quī mē mīsit}.

\vs%{Mc-9-38}
Tum Jōhannēs sīc eum est adlocūtus:
\quote{%
  Magister, vīdimus quemdam in tuō nōmine ējicientem daemonia,
  quī nōn sequitur nōs;
  itaque eum prohibuimus, quoniam nōn sequitur nōs}.
\vs%{Mc-9-39}
At Jēsus:
\christquote{%
  Nē prohibēte eum; ― inquit ―
  nēmō enim est quī meō nōmine mīrāculum faciat,
  quī cito possit mihi maledīcere:
\vs%{Mc-9-40}
  nam quī contrā nōs nōn est, ā nōbīs est.
\vs%{Mc-9-41}
  Etenim quī vōbīs vel aquae pōculum meō nōmine bibendum dederit,
  quod Chrīstī sītis, certō scītōte: eum nōn perditūrum operam.

\vs%{Mc-9-42}
  Et quī vel minimum eōrum, quī mihi fidem habent, laeserit,
  satius esset eī, ejus collō molārem lapidem circumpōnī, eum‧que in mare dējicī.

\vs%{Mc-9-43}
  Quodsī tibi tua manus obficit, abscinde eam:
  satius enim est tibi mancum in vītam ingredī,
  quam duās habentem manūs abīre in gehennam, ign|im inexstīnctum\obsolete{;}
\vs%{Mc-9-44}
  \obsolete{ubī̆
    \bibleallusion{nec eōrum vermis interit, nec ignis exstinguitur}}.

\vs%{Mc-9-45}
  Et sī tibi pēs tuus obficit, abscinde eum:
  satius est tē in vītam claudum ingredī,
  quam duōs habentem pedēs in gehennam conjicī, in ign|im inexstīnctum\obsolete{;}
\vs%{Mc-9-46}
  \obsolete{ubī̆
    \bibleallusion{nec eōrum vermis interit, nec ignis exstinguitur}}.

\vs%{Mc-9-47}
  Et sī tibi tuus oculus obficit, erue eum:
  satius est tibi ūnoculum in \DEĪ\ \RĒGNUM ingredī,
  quam duōs habentem oculōs in ignis gehennam conjicī;
\vs%{Mc-9-48}
  ubī̆ \bibleallusion{nec eōrum vermis interit, nec ignis exstinguitur}:
\vs%{Mc-9-49}
  nam et omnēs ign|ī saliendī%
    \obsolete{, et omne sacrificium sale saliendum est}.
\vs%{Mc-9-50}
  Bonum est sāl; sed sī sāl īnsultum fīat, quō condiētur ipsum?
  Habēte in vōbīs salem et pācem inter vōs gerite}.
\stopbookchapter

\startbookchapter %Chapter 10
\startabstract
  Mōysis repudium. Chrīstī conjugium, repudium. Puerōrum salūs ac faustitās.
  Bonus quis. Fortūnārum discrīmen. Deī omnipotentia. Piōrum remūnerātiō.
  Īnstāns lētum Chrīstī. Ā quō piīs dignitās. Chrīstus ad quid.
  Caecus Bartīmaeus.
\stopabstract

\vs%{Mc-10-1}
Illinc profectus vēnit in fīn|īs Jūdaeae per Trānsjordānīnum agrum;
eum‧que rūrsum convēnit vulgus hominum, quōs ille rūrsum, utī solitus erat, docet.
\vs%{Mc-10-2}
Adgressī autem eum Pharīsaeī interrogārunt,
licēret‧ne virō uxōrem repudiāre, tentantēs eum.
\vs%{Mc-10-3}
Quibus ille sīc respondit:
\christquote{Quid vōbīs praecēpit Mōysēs?}.
\vs%{Mc-10-4}
Et illī:
\quote{Mōysēs permīsit \bibleallusion{scrīptō dīvortiī īnstrūmentō repudiāre}}.
\vs%{Mc-10-5}
At Jēsus respondēns:
\christquote{%
  Propter vestram ― inquit iīs ―
  pervicāciam scrīpsit ille vōbīs illud praeceptum.
\vs%{Mc-10-6}
  Sed ā creātiōnis prīmōrdiō \bibleallusion{marem et fēminam fēcit eōs}
  \obsolete{\DEUS}.
\vs%{Mc-10-7}
  \bibleallusion{Hujus causā homō, relictō suō patre atque mātre,
  haerēbit uxōrī suae,
\vs%{Mc-10-8}
  fīet‧que ex duōbus ūna carō}; itaque nōn jam sunt duo sed ūna carō.
\vs%{Mc-10-9}
  Quod ergō \DEUS conjūnxit, homō nē disjungat}.
\vs%{Mc-10-10}
Item‧que domī rūrsus eum eādem dē rē interrogārunt ejus discipulī.
\vs%{Mc-10-11}
Quibus ille:
\christquote{%
  Quisquis uxōrem suam repudiat aliam‧que dūcit, adulterat in eam;
\vs%{Mc-10-12}
  et sī uxor, repudiātō virō suō, nūpserit alterī, adulterat}.

\vs%{Mc-10-13}
Addūcēbantur autem eī puerī, ut eōs tangeret;
discipulī vērō eōs ā quibus addūcēbantur, objūrgābant.
\vs%{Mc-10-14}
Quō vīsō, indignātus Jēsus, dīxit iīs:
\christquote{%
  Sinite puerōs ad mē venīre, nēve eōs prohibēte;
  tālium enim est \DEĪ\ \RĒGNUM.
\vs%{Mc-10-15}
  Hoc prō certō habētōte: quī \DEĪ\ \RĒGNUM nōn adcēperit utī puer,
  nōn esse in id ingressūrum}.
\vs%{Mc-10-16}
Deinde eōs complexus ulnīs et manibus iīs inpositīs, iīs bene precātus est.

\vs%{Mc-10-17}
Eō autem iter faciente, adcurrit quīdam, quī eum flexīs ad eum genibus,
sīc interrogāvit:
\quote{Magister bone, quid faciam utī vītam adipīscar aeternam?}.
\vs%{Mc-10-18}
Cui Jēsus:
\christquote{%
  Quid mē dīcis bonum? ― inquit ― Nūllus est bonus, nisi ūnus \DEUS.
\vs%{Mc-10-19}
  Praecepta nōstī:
  \bibleallusion{nē adulterātō, nē obcīditō, nē fūrātor,
  nē falsum testimōnium dīcitō}, nē fraudātō,
  \bibleallusion{honōrātō tuum patrem atque mātrem}}.

\vs%{Mc-10-20}
Et ille respondēns: \quote{Magister, ― inquit ― haec omnia ā puerīs servāvī}.
\vs%{Mc-10-21}
Hīc Jēsus eum intuitus amāvit, eī‧que dīxit:
\christquote{%
  Ūnum tibi deest:
  ī vēnditum quidquid habēs, et in pauperēs ērogātō, habitūrus in caelō thēsaurum;
  deinde mē%
  \obsolete{, crucem ferēns,}
            secūtum venītō}.
\vs%{Mc-10-22}
At ille, eō dictō, maestus abiit dolēns:
quippe quī multīs praeditus esset opibus.

\vs%{Mc-10-23}
Et Jēsus circum intuitus dīcit discipulīs suīs:
\christquote{%
  Ō quam difficulter, quī pecūniās habent,
  in \DEĪ\ \RĒGNUM ingredientur}.
\vs%{Mc-10-24}
Hīc, quum discipulī ejus dicta dēmīrārentur, Jēsus rūrsum sīc eōs adloquitur:
\christquote{%
  Gnātī, quam difficile est frētōs pecūniīs in \DEĪ\ \RĒGNUM ingredī.
\vs%{Mc-10-25}
  Facilius est rudentem per forāmen acūs trājicī,
  quam dīvitem in \DEĪ\ \RĒGNUM intrāre}.
\vs%{Mc-10-26}
Quum‧que illī vehementius stupērent, sīc sēcum ipsī cōgitantēs:
\quote{Quis ergō servārī potest?}.
\vs%{Mc-10-27}
Intuitus eōs Jēsus dīxit:
\christquote{%
  Ab hominibus fī̆erī nōn potest, at ā \DEŌ potest:
  omnia enim ā \DEŌ possunt fī̆erī}.

\vs%{Mc-10-28}
Tum Petrus sīc dīcit eī:
\quote{Nōs quidem, relictīs omnibus, tē secūtī sumus}.
\vs%{Mc-10-29}
Et Jēsus:
\christquote{%
  Hoc vōbīs cōnfirmō: ― inquit respondēns ―
  nūllum esse, quī domum aut frātrēs aut sorōrēs aut patrem aut mātrem
  aut uxōrem aut līberōs aut agrōs, meā et \ĒVANGELIĪ causā, relīquerit,
\vs%{Mc-10-30}
  quīn sit et nunc hōc tempore centuplum domōs et frātrēs
  et sorōrēs et mātrēs et līberōs et agrōs inter adversa,
  et in futūrō aevō sempiternam vītam cōnsecūtūrus.
\vs%{Mc-10-31}
  Sed multī et prīmī erunt ultimī, et ultimī prīmī}.

\vs%{Mc-10-32}
Quum autem Hierosolyma ascenderent, et eōs antecēderet Jēsus,
pavēbant et inter sequendum formīdābant.
Et Jēsus adsūmptīs rūrsum \DUODECIM coepit iīs ēventūra sibi praedīcere:
\vs%{Mc-10-33}
\christquote{%
  Nōs quidem ascendimus Hierosolyma; ― inquit ―
  et \FĪLIUS hominis trādētur pontificibus atque scrībīs,
  quī eum morte damnābunt, exterīs‧que trādent,
\vs%{Mc-10-34}
  inlūdendum, verberandum, cōnspuendum, et interficiendum,
  et tertiā diē resubrēctūrum}.

\vs%{Mc-10-35}
Convēnērunt autem eum Jācōbus et Jōhannēs Zebedaeī fīliī in haec verba:
\quote{Magister, velimus, utī quod ā tē postulābimus, id nōbīs praestēs}.
\vs%{Mc-10-36}
Quibus ille:
\christquote{Quid vultis utī vōbīs praestem?}.
\vs%{Mc-10-37}
Et illī:
\quote{%
  Dā nōbīs, ― inquiunt eī ― utī tibi alter ad dexteram,
  alter ad sinistram adsideāmus in tuā glōriā}.
\vs%{Mc-10-38}
At Jēsus dīxit iīs:
\christquote{%
  Nescītis quid postulētis.
  Potestis pōculum pōtāre, quod ego pōtūrus sum,
  et eōdem mēcum baptismate baptīzārī?}.
\vs%{Mc-10-39}
\quote{Possumus}, inquiunt.
Et Jēsus:
\christquote{%
  Pōculum quidem, quod ego pōtūrus sum, pōtābitis,
  eōdem‧que mēcum baptismate baptīzābiminī;
\vs%{Mc-10-40}
  sed utī mihi dexterā laevā‧que adsideātur, nōn est meum dare,
  nisi quibus id parātum est}.

\vs%{Mc-10-41}
Hōc audītō, decem coepērunt dē Jācōbō et Jōhanne indignārī.
\vs%{Mc-10-42}
At Jēsus advocātīs iīs dīxit:
\christquote{%
  Scītis eōs, quōs vidēmus inperāre extrāneīs, in eōs dominārī,
  et eōrum maximōs quōsque, in eōs maximē potestātem obtinēre.
\vs%{Mc-10-43}
  At in vōbīs nōn sīc fīet,
  sed quī vestrum volet fī̆erī inter vōs maximus, erit vester famulus;
\vs%{Mc-10-44}
  et quī volet fī̆erī vestrum prīmus, erit omnium servus:
\vs%{Mc-10-45}
  nam \FĪLIUS hominis nōn vēnit, ut ipsī ministrētur,
  sed utī ministret ipse animam‧que suam in multōrum redēmptiōnem inpendat}.

% Iericho, incertum
\vs%{Mc-10-46}
Vēnērunt autem Jerichūntem. Eō‧que et ejus discipulīs
\dubious{ab} Jerīchūnte cum frequentī multitūdine discēdentibus, % ???
Tīmaeī fīlius Bartīmaeus caecus, quī apud viam sedēns, mendīcābat.
\vs%{Mc-10-47}
Ut audīvit Jēsum esse Nāzarēnum, coepit ita clāmāre:
\quote{Dāvīdide Jēsū, miserēre meī!}.
\vs%{Mc-10-48}
Ac multīs eum, utī tacēret, objūrgantibus, multō magis clāmābat:
\quote{Dāvīdide, miserēre meī!}.
\vs%{Mc-10-49}
Tum Jēsus restitit
\indirectchristquote{eum‧que jussit arcessī}.
Quum‧que illī eum arcesserent et:
\quote{Bonō es animō; ― dīcerent ― surge, arcessit tē},
\vs%{Mc-10-50}
ille, abjectā veste suā, subrēxit et ad Jēsum vēnit.
\vs%{Mc-10-51}
Et Jēsus sīc eum adloquitur:
\christquote{Quid tibi vīs faciam?}.
Cui caecus:
\quote{Magister, ― inquit eī ― utī dispiciam}.
\vs%{Mc-10-52}
Et Jēsus dīxit eī:
\christquote{Abī, tua tē fīdūcia servāvit}.
Atque ille prōtinus dispexit et Jēsum in viā secūtus est.
\stopbookchapter

\startbookchapter %Chapter 11
\startabstract
  Chrīstus ovāns in urbem. In fīcum dīrae. Pūrgātum templum.
  Conjūrātiō in Jēsum. Fideī virtūs ac precātiōnis. In invidōs dēmentia.
  Sacerdōtum cōnfūtātiō.
\stopabstract

\vs%{Mc-11-1}
Quum autem Hierosolymīs propinquārent, ad Bēthphagē et Bēthaniān
ad montem olīvārum, mittit suōrum discipulōrum duōs:
\vs%{Mc-11-2}
\christquote{%
  Īte ― inquit ― in vīcum quī est ē regiōne vestrī,
  et prōtinus in eum intrantēs inveniētis vīnctum asellum,
  quem nūllus hominum īnsēdit; eum solvitōte et mihi addūcitōte.
\vs%{Mc-11-3}
  Quodsī quis ex vōbīs quaesierit, \indirectsubquote{cūr id faciātis?},
  dīcitōte:
  \indirectsubquote{\DOMINUM eō egēre, et statim eum hūc dīmittet}}.
\vs%{Mc-11-4}
Ita illī profectī et asellum forīs ad portam vīnctum in biviō nactī solvunt.
\vs%{Mc-11-5}
Et eōrum quīdam quī illīc aderant, ex iīs quaesīvērunt,
\indirectquote{quid facerent, quī asellum solverent}.
\vs%{Mc-11-6}
Quibus illī dīxērunt quemadmodum mandāverat Jēsus.
\vs%{Mc-11-7}
Atque ita, permittentibus illīs, addūxērunt asellum ad Jēsum
eī‧que sua vestīmenta \unclassical{superinjēcērunt}; atque ille superīnsēdit.
\vs%{Mc-11-8}
Frequentēs autem suīs vestīmentīs viam īnsternēbant,
aliī abscissās ex arboribus frondēs per viam sternere.
\vs%{Mc-11-9}
Quīque antecēdēbant quīque subsequēbantur, sīc clāmāre:
\quote{%
  \bibleallusion{Hōsanna! Bene sit venientī in nōmine \DOMINĪ!}
\vs%{Mc-11-10}
  Fēlīx veniēns \RĒGNUM in nōmine \DOMINĪ patris nostrī Dāvīdis!
  \bibleallusion{Hōsanna} in suprēmīs!}.

\vs%{Mc-11-11}
Ingressus est autem Hierosolyma \DOMINUS et in templum;
omnibus‧que circumspectīs, quum jam sēra esset hōra,
profectus est Bēthaniān cum \DUODECIM.

\vs%{Mc-11-12}
Postrīdiē, iīs Bēthaniā ēgressīs, ēsurīvit
\vs%{Mc-11-13}
fīcum‧que procul foliōsam cōnspicātus adcessit,
sī quid in eā forte invenīret;
sed ut ad eam vēnit, nihil invēnit nisi folia:
nec enim tempus erat fīcuum.
\vs%{Mc-11-14}
Itaque sīc eam adlocūtus est Jēsus:
\christquote{Numquam deinceps quisquam ex tē frūctum comedat}.
Audiēbant autem ejus discipulī.

\vs%{Mc-11-15}
Deinde veniunt Hierosolyma.
Et ingressus in fānum Jēsus coepit ējicere vēndent|īs et ement|īs in fānō,
et nummulāriōrum mēnsās et sellās vēndentium columbās ēvertit;
\vs%{Mc-11-16}
neque sinēbat quemquam per fānum ferre vāsa.
\vs%{Mc-11-17}
Atque hīs verbīs eōs adloquēns docēbat:
\christquote{%
  Nōnne est scrīptum:
  \biblequote{Domus mea domus subplicātiōnis vocābitur cūnctīs gentibus}?
  Et vōs ex eā \bibleallusion{latrōnum spēluncam} fēcistis}.

\vs%{Mc-11-18}
Hōc audītō, scrībae et pontificēs eum perdere studēbant;
metuēbant enim eum: quippe cujus doctrīnam stupēret omne vulgus.
\vs%{Mc-11-19}
Sub vesperum autem ēgressus est ex urbe.

\vs%{Mc-11-20}
Māne praetereuntēs vident siccātam ā rādīcibus fīcum.
\vs%{Mc-11-21}
Id quod recordātus Petrus, dīcit eī:
\quote{Rabbī, ecce fīcus, quam tū exsecrātus es, āruit}.
\vs%{Mc-11-22}
Et Jēsus respondēns inquit iīs:
\christquote{%
  Habēte fidem \DEŌ!
\vs%{Mc-11-23}
  Hoc vōbīs cōnfirmō: sī quis jusserit huic montī,
  ut \indirectquote{āmoveātur, et in mare jaciātur},
  nec animō dubitāverit, sed quae dīcet crēdiderit ēventūra,
  continget eī quidquid dīxerit.
\vs%{Mc-11-24}
  Itaque sīc vōbīs dīcō:
  quidquid ōrantēs postulātis, crēdite vōs adsecūtūrōs, et vōbīs continget.
\vs%{Mc-11-25}
  Ōrātūrī autem agnōscite, sī quid habētis, quod dē quōpiam expostulētis,
  utī vester quoque \PATER, quī in caelīs est, vestra dēlicta vōbīs ignōscat.
\vs%{Mc-11-26}
  \obsolete{Quod nisi ignōscētis, nē \PATER quidem vester, quī in caelīs est,
  vōbīs dēlicta vestra ignōscet}}.

\vs%{Mc-11-27}
Deinde rūrsum veniunt Hierosolyma.
Eum‧que in fānō ambulantem pontificēs
scrībae‧que et senātōrēs conveniunt hīs verbīs:
\vs%{Mc-11-28}
\quote{%
  Quā potestāte ista facis?
  Et quis istam tibi ista faciendī potestātem dedit?}.
\vs%{Mc-11-29}
Quibus respondēns Jēsus:
\christquote{%
  Interrogābō ego quoque vōs quiddam; ― inquit ―
  quodsī mihi responderitis, dīcam vōbīs, quā potestāte haec faciam:
\vs%{Mc-11-30}
  Jōhannis baptisma ā caelō‧ne erat an ab hominibus? Respondēte mihi}.
\vs%{Mc-11-31}
At illī sīc sēcum cōgitāre:
\quote{%
  Sī \subquote{Ā caelō} dīxerimus, dīcet:
  \subquote{Cūr ergō eī nōn crēdidistis?}}.
\vs%{Mc-11-32}
Sīn \subquote{Ab hominibus} dīxissent, metuēbant populum:
omnēs enim Jōhannem vērē vātem esse dūcēbant.
\vs%{Mc-11-33}
Itaque Jēsū respondērunt \indirectquote{nescīre sē}.
Et Jēsus iīs sīc respondit:
\christquote{Nec ego vōbīs dīcō, quā haec potestāte faciam}.
\stopbookchapter

\startbookchapter %Chapter 12
\startabstract
  Vīnea bene īnstrūcta, servī, hērēs.
  Cultōrum punitiō, et vīneae locātiō. %orig. punitium
  Lapis obfendiculī. Deō Caesarī‧que danda. Saddūcaeōrum cōnfūtātiō.
  Deus quōrum sit. Praecepta duo. Quaestiō Chrīstī dē Dāvīde.
  Scrībārum ambitiō, avāritia, viduae mūnusculum.
\stopabstract

\vs%{Mc-12-1}
Tum per similitūdinēs eōs adloquī ōrsus est:
\Lchristquote{%
\bibleallusion{Vīneam cōnsēvit} quīdam
\bibleallusion{et saepe circumdedit et torcular dēfīxit turr|im‧que cōnstrūxit}
et eam locāvit agricolīs, deinde peregrē profectus est.
\vs%{Mc-12-2}
Post suō tempore mīsit ad agricolās servum,
ut ab iīs dē vīneae frūctū perciperet;
\vs%{Mc-12-3}
at illī eum cēpērunt verberārunt‧que et inānem dīmīsērunt.
\vs%{Mc-12-4}
Quum‧que rūrsus ad eōs mīsisset alium servum,
illī eī quoque caput lapidibus contudērunt eum‧que contumēliōsē dīmīsērunt.
\vs%{Mc-12-5}
Rūrsus alium mīsit, quem etiam interfēcērunt,
aliōs‧que multōs partim verberārunt, partim necārunt.
\vs%{Mc-12-6}
Igitur, quum adhūc ūnum habēret fīlium sibi cārissimum,
mīsit eum quoque ad illōs ultimum cōgitāns fore,
utī \indirectquote{suum fīlium reverērentur}.
\vs%{Mc-12-7}
At illī agricolae sīc inter sēsē dīcere:
\subquote{Hic est hērēs. Agite, interficiāmus eum, et nostra erit hērēditās}.
\vs%{Mc-12-8}
Itaque cēpērunt eum necārunt‧que et ex vīneā ējēcērunt.
\vs%{Mc-12-9}
Quid ergō faciet vīneae dominus?
Veniet agricolās‧que perdet et aliīs vīneam dabit.
\vs%{Mc-12-10}
An nē illud quidem scrīptum lēgistis:}
\startlines
\Rchristquote{\subbiblequote{%
  Quem lapidem inprobāverant strūctōrēs,
  is adhibitus est ad caput angulī;
\vs%{Mc-12-11}%
  id quod ā \DOMINŌ profectum est
  et nōbīs mīrum vidētur}?}.
\stoplines

\vs%{Mc-12-12} At illī eum capere cōnārī,
Quippe quī eum similitūdinem illam contrā ipsōs dīxisse intellegerent,
sed metuērunt plēbem. Igitur, eō relictō, dīgressī
\vs%{Mc-12-13}
mittunt ad eum Pharīsaeōrum et Hērōdiānōrum quōsdam, quī eum verbīs inrētīrent.
\vs%{Mc-12-14}
Hī eum hujusmodī ōrātiōne adgressī sunt:
\quote{%
  Magister, scīmus vērācem esse tē neque quemquam cūrāre;
  quippe quī ad hominum persōnam nōn spectēs,
  sed vērē dīvīnam vītae viam doceās.
  Licet‧ne cēnsum Caesarī pendere an nōn?
\vs%{Mc-12-15}
  Pendāmus an nōn pendāmus?}.
At ille eōrum simulātiōnem intellegēns:
\christquote{%
  Quid mē tentātis? ― inquit iīs ―
  Adferte mihi dēnārium, utī videam}.
\vs%{Mc-12-16}
Quum‧que illī adtulissent:
\christquote{Cujus est, ― inquit iīs ― imāgō haec et īnscrīptiō?}.
\quote{Caesaris}, inquiunt.
\vs%{Mc-12-17}
Et Jēsus respondēns iīs:
\christquote{Solvite igitur ― inquit ― Caesariāna Caesarī, et dīvīna \DEŌ}.
Illī vērō eum admīrārī.

\vs%{Mc-12-18}
Tum eum convēnērunt Saddūcaeī, quī quidem resubrēctiōnem esse negant,
et sīc interrogārunt:
\vs%{Mc-12-19}
\quote{%
  Magister, scrīpsit nōbīs Mōysēs,
  \bibleallusion{sī cujus frāter moriātur}, uxōre superstite,
  \bibleallusion{nec līberōs relinquat,
  utī dūcat ejus frāter ejus uxōrem frātrī‧que suō prōlem suscitet}.
\vs%{Mc-12-20}
  Septem frātrēs fuērunt:
  quōrum prīmus, ductā uxōre, mortuus est, nūllā prōle superstite;
\vs%{Mc-12-21}
  item‧que secundus, eādem ductā, mortuus est, nec ipse prōlem relīquit;
  similiter‧que tertius;
\vs%{Mc-12-22}
  dēnique septem illī eam dūxērunt neque prōgeniem relīquērunt.
  Postrēma omnium mortua est etiam fēmina.
\vs%{Mc-12-23}
  Igitur in resubrēctiōne, quum resubrēctum fuerit, cujus eōrum erit uxor,
  quum septem eam habuerint in mātrimōniō?}.
\vs%{Mc-12-24}
Et Jēsus respondēns:
\christquote{%
  Nīmīrum ideō errātis, ― inquit iīs ―
  quod scrīpta nōn intellegitis neque \DEĪ potentiam?
\vs%{Mc-12-25}
  Quum enim ā morte resubrēctum fuerit, nūlla conjungentur cōnūbia,
  sed erunt angelōrum similēs quī sunt in caelīs.
\vs%{Mc-12-26}
  Dē mortuīs autem quod resurgant, nōn lēgistis in Mōysis librō,
  in rubō ut eum sīc adlocūtus sit \DEUS:
  \biblequote{Ego \DEUS Abrāhāmī et Isaacī et Jācōbī}?
\vs%{Mc-12-27}
  Nōn est mortuōrum \DEUS sed vīventium \obsolete{\DEUS}!
  Itaque vōs multum errātis}.

\vs%{Mc-12-28}
Tum scrībārum quīdam, audītīs iīs disceptantibus,
vidēns eum illīs rēctē respondisse, adcessit et ex eō quaesīvit,
\indirectquote{quodnam esset omnium praeceptōrum prīmum?}.
\vs%{Mc-12-29}
Cui Jēsus sīc respondit:
\christquote{%
  Prīmum omnium praeceptōrum est:
  \subquote{%
    \bibleallusion{Audī, Isrāēlīta: \DOMINUS\ \DEUS vester, \DOMINUS ūnus est;
\vs%{Mc-12-30}
    itaque \DOMINUM\ \DEUM tuum
    tōtō corde, tōtō animō}, tōtā mente,
    \bibleallusion{tōtīs‧que vīribus amātō}}.
  \obsolete{Hoc prīmum est praeceptum.}
\vs%{Mc-12-31}
  Huic simile est alterum:
  \biblequote{Alterum utī tē ipsum amātō}.
  Hīs majus aliud praeceptum nūllum est}.
\vs%{Mc-12-32}
Et scrība:
\quote{%
  Rēctē, magister, profectō dīxistī.
  \biblequote{Nam ūnus est \DEUS, praetereā nūllus;
\vs%{Mc-12-33}
  quem tōtō corde, omnī studiō, tōtō animō, summāque ope amāre},
  necnōn \biblequote{aliōs utī sē ipsum amāre},
  plūs est quam omnēs victimae et sacrificia}.
\vs%{Mc-12-34}
Hīc Jēsus eum cordātē respondisse vidēns, dīxit eī:
\christquote{Nōn procul abes ā \RĒGNŌ\ \DEĪ}.
Ac deinceps eum interrogāre audēbat nēmō.

\vs%{Mc-12-25}
Et Jēsus in fānō docēns, sīc verba fēcit:
\christquote{%
  Quī fit, utī dīcant scrībae fore,
  utī Chrīstus genus dūcat ā Dāvīde,
\vs%{Mc-12-36}
  quum ipse Dāvīd, \SĀNCTĪ\ \SPĪRITŪS īnstīnctū, dīxerit:
  \startlines\subbiblequote{%
    Ait \DOMINUS\ \DOMINŌ meō: sēde ad meam dexteram,
    dōnec ecficiam ex hostibus tuīs tuōrum pedum subsellium}.
  \stoplines

\vs%{Mc-12-37}
  Quum igitur eum dīcat ipse Dāvīd \DOMINUM, unde fit, utī sit ejus prōgnātus?}.

Quum‧que frequēns hominum multitūdō eum libenter audīret,
\vs%{Mc-12-38}
sīc iīs inter docendum dīcēbat:
\christquote{%
  Cavēte ā scrībīs,
  quī stolātī incēdere amant et in forīs salūtātiōnēs
\vs%{Mc-12-39}
  et in conlēgiīs prīmās sessiōnēs et in cēnīs prīma adcubandī loca;
\vs%{Mc-12-40}
  quī viduārum domōs dēvorant et per ostentātiōnem prōlixē ōrant.
  Hīs graviōrēs poenās dabunt}. % orig. hī

\vs%{Mc-12-41}
Sedēns autem Jēsus ē regiōne fiscī spectābat,
utī multitūdō cōnferret aes in fiscum;
ac, multīs dīvitibus multa cōnferentibus,
\vs%{Mc-12-42}
vēnit quaedam vidua pauper quae duōs terunciōs contulit:
est autem teruncius idem quod quadrāns.
\vs%{Mc-12-43}
Tum ille advocātīs suīs discipulīs:
\christquote{%
  Certō scītōte: ― inquit iīs ―
  hanc pauperem viduam plūrimum omnium contulisse, quī contulērunt in fiscum.
\vs%{Mc-12-44}
  Omnēs enim ex eō, quod ipsīs supererat, contulērunt;
  at haec, suā pēnūriā, quidquid habēbat, tōtum vīctum suum contulit}.
\stopbookchapter

\startbookchapter %Chapter 13
\startabstract
  Templī ruīna. Pseudochristī. Orbis calamitātēs. Piōrum aerumnae, cōnsōlātiō.
  Familiae discidia, beātī. Caelī prōdigia, coāctī piī. % orig. dissidia
  Chrīstī vēritās perpetua. Jūdiciī hōra cui nota. Vigilantia.
\stopabstract

\vs%{Mc-13-1}
Exeuntī autem ex fānō, quīdam ejus discipulōrum sīc dīxit:
\quote{Magister, vidē quantī lapidēs et strūctūrae}.
\vs%{Mc-13-2}
Cui respondēns Jēsus:
\christquote{%
  Vidēs ― inquit ― hās ingent|īs strūctūrās?
  Ita dissolventur, utī nūllus lapidī lapis inpositus relinquātur}.
\vs%{Mc-13-3}
Deinde sedentem eum in monte olīvārum ē regiōne fānī
interrogārunt seorsum Petrus et Jācōbus et Jōhannēs et Andreās, hīs verbīs:
\vs%{Mc-13-4}
\quote{%
  Dīc nōbīs, quandō haec futūra sint,
  et quod signum significābit haec, perficienda omnia?}.

\vs%{Mc-13-5}
Et Jēsus iīs respondēns, sīc loquitur:
\christquote{%
  Vidēte, nē quis vōs fallat.
\vs%{Mc-13-6}
  Venient enim multī in meō nōmine dīcentēs \indirectquote{sē eum esse},
  multōs‧que fallent.
\vs%{Mc-13-7}
  Quum autem duella duellī‧que rūmōrēs audiētis, nōlīte turbārī;
  oportet enim ea fī̆erī sed nōndum erit fīnis.
\vs%{Mc-13-8}
  Cōnsurget enim gēns in gentem, et rēgnum in rēgnum,
  erunt‧que terraemōtūs certīs locīs et fam||īs \obsolete{et turbae};
  ea erunt dolōrum prīncipis.

\vs%{Mc-13-9}
  Sed prōspicitōte vōs vōbīs ipsīs:
  sistēminī enim in cōnsessūs et conlēgia;
  vāpulābitis et ad magistrātūs rēgēs‧que dūcēminī meā causā,
  quod sit iīs testimōniō.
\vs%{Mc-13-10}
  Ac per omn|īs gent|īs oportet prius \ĒVANGELIUM pūblicārī;
\vs%{Mc-13-11}
  Quum autem dūcēminī, ac sistēminī,
  nōlīte prius cōgitāre aut praemeditārī dīcenda,
  sed, quod vōbīs eādem hōrā subgerētur, id dīcitōte:
  nōn enim vōs eritis quī loquiminī, sed \SPĪRITUS\ \SĀNCTUS.
\vs%{Mc-13-12}
  Trādet autem frātrem frāter ad mortem, et pater fīlium;
  cōnsurgent‧que in parent|īs gnātī et eōs necābunt;
\vs%{Mc-13-13}
  eritis‧que omnibus invīsī propter nōmen meum.
  Sed quī ad fīnem pertulerit, is servābitur.

\vs%{Mc-13-14}
  Quum autem \bibleallusion{calamitōsum nefās}%
  \obsolete{, ā Dānīēle vāte praedictum,}
  stāre vidēbitis, ubī̆ nōn dēbet, quī legit, animadvertat:
  tum quī erunt in Jūdaeā, fugiant in mont|īs;
\vs%{Mc-13-15}
  et quī erit in tēctō, nē dēscendat in aed|īs
  nec ingrediātur ad auferendum aliquid ex suīs aedibus;
\vs%{Mc-13-16}
  et quī rūrī erit, nē gradum referat, suae vestis auferendae grātiā.
\vs%{Mc-13-17}
  Hei uterum ferentibus et lactantibus illō tempore!
\vs%{Mc-13-18}
  Optāte vērō, nē adcidat hieme vestra fuga:
\vs%{Mc-13-19}
  erit enim temporis illī̆us \bibleallusion{calamitās, quanta nec ā rērum},
  quās \DEUS creāvit, \bibleallusion{creātiōne fuit adhūc},
  nec posteā futūra est.
\vs%{Mc-13-20}
  Quod nisi \DOMINUS tempus illud dēcurtāret, nūllus mortālium ēvāderet.
  Sed propter ēlēctōs, quod dēlēgit, dēcurtābit id tempus.
\vs%{Mc-13-21}
  Tum sī quis vōbīs dīxerit \indirectquote{hīc aut illīc adesse Chrīstum},
  nē crēditōte.
\vs%{Mc-13-22}
  Exsistent enim falsī chrīstī falsī‧que vātēs,
  quī prōdigiōsa et portentōsa dēsignābunt,
  ad dēcipiendōs, sī fī̆erī posset, etiam ēlēctōs.
\vs%{Mc-13-23}
  Vōs vērō prōspicite: ēn praedīxī vōbīs omnia.

\vs%{Mc-13-24}
  Sed illō tempore post calamitātem illam
  \bibleallusion{sōl obscūrābitur, et lūna suum splendōrem nōn ēmittet,
\vs%{Mc-13-25}
  et caelī stēllae} dēcident,
  \bibleallusion{caelestēs‧que potestātēs} conmovēbuntur.

\vs%{Mc-13-26}
  Deinde cernētur \bibleallusion{\FĪLIUS hominis veniēns in nūbibus}
  cum magnā potentiā atque glōriā.
\vs%{Mc-13-27}
  Ac tum, suīs dīmissīs nūntiīs, congregābit ēlēctōs suōs ā quattuor ventīs,
  ab ultimīs terrīs ad extrēmum caelī.

\vs%{Mc-13-28}
  Ā fīcū autem discite exemplum:
  quum jam ejus rāmī tenerī sunt ea‧que folia prōcreat,
  intellegitis īnstāre aestātem.
\vs%{Mc-13-29}
  Sīc et vōs, quum haec fī̆erī vidēbitis, scītōte prope adesse prō foribus.
\vs%{Mc-13-30}
  Hoc vōbīs cōnfirmō:
  nōn ante praeteritūrum esse hoc saeculum, quam fīant haec omnia.
\vs%{Mc-13-31}
  Caelum licet et terra intereant, mea dicta nōn interībunt.
\vs%{Mc-13-32}
  Cēterum, diem illam et hōram nūllus scit,
  nē angelī quidem caelestēs neque \FĪLIUS, sed tantum \PATER.

\vs%{Mc-13-33}
  Vidēte utī vigilētis et ōrētis; nescītis enim quandō tempus futūrum sit.
\vs%{Mc-13-34}
  Quemadmodum peregrīnāns homō, quī suam domum relīquit
  suīs‧que servīs potestātem, et suum cuique mūnus adsignāvit,
  et jānitōrī, utī vigilāret, mandāvit.
\vs%{Mc-13-35}
  Vigilāte igitur; nescītis enim quandō domūs dominus ventūrus sit:
  sērō an mediā nocte an galliciniō an māne;
\vs%{Mc-13-36}
  nē, sī dē inprōvīsō vēnerit, obfendat vōs dormient|īs.
\vs%{Mc-13-37}
  Quae autem vōbīs dīcō, omnibus dīcō: vigilāte!}.
\stopbookchapter

\startbookchapter %Chapter 14
\startabstract
  In Chrīstum īnsidiae. Unguentum apud Simōnem; dēfēnsa mulier. %defēnsa???
  Jūdās prōditor. Pascha. Cēna dominica. Petrī temeritās.
  Jēsū angor, precātiō terna, discipulōrum sopor. Captus Jēsus, suōrum fuga.
  Falsī testēs. Jēsus dē sē ipsō, sprētus, ictus.
  Petrus dēficiēns, perjūrus, lūgēns.
\stopabstract

\vs%{Mc-14-1}
Quum autem futūrum esset Pascha et Azȳmālia post bīduum,
studēbant pontificēs et scrībae eum fraudulenter captum interficere;
\vs%{Mc-14-2}
sed
\indirectquote{id diē fēstō faciendum negābant, nē quis fī̆eret populī tumultus}.
\vs%{Mc-14-3}
Et dum erat Bēthaniae in domō Simōnis leprōsī, eō adcubante,
vēnit mulier habēns alabastrum unguentī spīcātae nardī pretiōsī,
quae, cōnfractō alabastrō, perfūdit ejus caput.
\vs%{Mc-14-4}
Erant autem quīdam quī apud sēsē indignābantur atque ita dīcēbant:
\quote{%
  Quōrsum haec unguentī jactūra facta est?
\vs%{Mc-14-5}
  Poterat hoc plūs quam trecentīs dēnāriīs vēndī et in pauperēs ērogārī}.

Dum illī sīc adversus eam fremunt,
\vs%{Mc-14-6}
Jēsus dīxit:
\christquote{%
  Sinite eam; quid negōtium eī facessitis?
  Bonō opere fūncta est ergā mē.
\vs%{Mc-14-7}
  Semper enim pauperēs habēbitis vōbīscum, quibus, quandōcumque volētis,
  bene facere possītis; at mē nōn semper habēbitis.
\vs%{Mc-14-8}
  Quod potuit haec, fēcit: anticipāvit meum corpus ad sepultūram ungere.
\vs%{Mc-14-9}
  Hoc vōbīs cōnfirmō: ubicumque tōtō orbe pūblicābitur hoc \ĒVANGELIUM,
  etiam quid ea fēcerit, dīcētur ad ejus memoriam}.
\vs%{Mc-14-10}
Jūdās autem Iscariōta, ūnus ex \DUODECIM, pontificēs adiit,
ut eum illīs trāderet.
\vs%{Mc-14-11}
Quō illī audītō laetātī, prōmīsērunt eī pecūniam sē datūrōs.
Itaque quaerēbat quō pactō eum posset obportūnē prōdere.

\vs%{Mc-14-12}
Prīmō autem Azȳmālium diē, quum Pascha inmolandum foret,
sīc eī dīcunt ejus discipulī:
\quote{Ubī̆ vīs, utī tibi epulandum Pascha praeparātum eāmus?}.

\vs%{Mc-14-13}
Tum ille discipulōrum suōrum duōs dīmittit iīs‧que dīcit:
\christquote{%
  Īte in urbem, obcurret vōbīs quīdam aquae fictile portāns;
  eum sequēminī
\vs%{Mc-14-14}
  et, quō intrāverit, sīc dīcitōte patrīfamiliae:
  \subquote{Quaerit magister,
  ubī̆ sit mānsiō, ubī̆ Pascha cum discipulīs suīs pollūceat}.
\vs%{Mc-14-15}
  Ipse vērō vōbīs ingēns cēnāculum ostendet, īnstrātum ac parātum;
  illīc nōbīs praeparātōte}.
\vs%{Mc-14-16}
Ita profectī ejus discipulī, vēnērunt in urbem et nactī,
quae ille iīs dīxerat, Pascha praeparārunt.

\vs%{Mc-14-17}
Deinde sub vesperum vēnit Jēsus cum \DUODECIM
\vs%{Mc-14-18}
ac, discumbentibus et epulantibus iīs, dīxit:
\christquote{%
  Certō scītōte: fore, ut ūnus ex vōbīs mē prōdat,
  \bibleallusion{quī mēcum epulātur}}.
\vs%{Mc-14-19}
Illī vērō dolēre coepērunt eī‧que singulī dīcere:
\quote{Is‧ne ego sum?} \obsolete{et alius: \quote{Isne ego sum?}}.
\vs%{Mc-14-20}
Quibus ille respondēns:
\christquote{%
  Ūnus est ex \DUODECIM, ― inquit ― quī mēcum intingit in catīnum.
\vs%{Mc-14-21}
  Fīlius quidem hominis abit, quemadmodum dē eō scrīptum est;
  sed vae hominī illī per quem \FĪLIUS hominis prōdendus est!
  Praestāret hominī illī numquam esse nātum}.

\vs%{Mc-14-22}
Epulantibus autem iīs, Jēsus sūmptum pānem, āctīs laudibus,
frēgit iīs‧que dedit et dīxit:
\christquote{%
  Adcipite%
  \obsolete{, comedite}: hoc est corpus meum}.
\vs%{Mc-14-23}
Deinde sūmptum pōculum, grātiīs āctīs, dedit iīs; bibērunt‧que ex eō omnēs.
\vs%{Mc-14-24}
Atque ille iīs dīxit:
\christquote{%
  Hic est sanguis meus novī foederis, prō multīs ecfundendus.
\vs%{Mc-14-25}
  Hoc vōbīs certō dīcō:
  nōn mē amplius bibitūrum ex fētū vīteō ū̆sque ad illam diem,
  quum eum bibam novum in \RĒGNŌ\ \DEĪ}.
\vs%{Mc-14-26}
Deinde, laudibus dictīs, profectī sunt in montem olīvārum.
\vs%{Mc-14-27}
Iīs‧que sīc dīcit Jēsus:
\christquote{%
  Vōs mē omnēs hāc nocte dēserētis, sī̆quidem scrīptum est:
  \biblequote{Percutiam pāstōrem, et dispergentur ovēs}.
\vs%{Mc-14-28}
  Sed postquam resubrēxerō, praecēdam vōs in Galilaeam}.
\vs%{Mc-14-29}
At Petrus dīxit eī:
\quote{Etiamsī omnēs dēserant, at ego nōn item}.
\vs%{Mc-14-30}
Cui Jēsus:
\christquote{%
  Hoc tibi cōnfirmō: ― inquit ―
  fore, ut hodiē in hāc nocte, priusquam bis gallus cantet, tū mē ter abnegēs}.
\vs%{Mc-14-31}
At ille multō magis dīcere:
\quote{Immō etiamsī mē ūnā tēcum morī oporteat, nōn abnegābō tē}.
Itidem‧que etiam omnēs aliī dīcēbant.
\vs%{Mc-14-32}
Deinde ubī̆ vēnērunt in locum, cui nōmen est Gethsēmānī, dīxit suīs discipulīs:
\christquote{Sedēte hīc, dum ego ōrātum eō}.
\vs%{Mc-14-33}
Adsūmptīs‧que ūnā sēcum Petrō et Jācōbō et Jōhanne coepit turbārī atque angī,
\vs%{Mc-14-34}
iīs‧que dīxit:
\christquote{Prae animī dolōre ēmorior; manēte hīc et vigilāte}.
\vs%{Mc-14-35}
Tum paululum prōgressus, prōcidit humī ōrāvit‧que, utī,
sī fī̆erī posset, hōram illam dēclīnāret;
\vs%{Mc-14-36}
et: \christquote{%
  Abbā, \PATER! ― inquit ― Omnia tū potes.
  Averte hoc ā mē pōculum; vērum nōn quod ego volō, sed quod tū}.
\vs%{Mc-14-37}
Deinde vēnit et eōs dormient|īs nactus, dīcit Petrō:
\christquote{%
  Simōn, dormīs? Nōn potuistī ūnam hōram vigilāre?
\vs%{Mc-14-38}
  Vigilāte et ōrāte, nē veniātis in tentātiōnem;
  spīritus quidem prōmptus est, sed carō īnfirma}.
\vs%{Mc-14-39}
Deinde rūrsus ōrātum iit eadem‧que verba dīxit.
\vs%{Mc-14-40}
Ac reversus eōs rūrsum dormient|īs obfendit;
eōrum oculīs ita gravātīs, utī nescīrent quid eī respondērunt.
\vs%{Mc-14-41}
Deinde tertiō vēnit, et:
\christquote{%
  Dormīte jam ― inquit ― et requiēscite.
  Satis est, vēnit hōra: ecce trāditur \FĪLIUS hominis in manūs inprobōrum.
\vs%{Mc-14-42}
  Surgite, eāmus; ēn, quī mē prōdit, adventat}.

\vs%{Mc-14-43}
Et prōtinus adhūc, eō loquente, advēnit Jūdās, quī ūnus erat ex \DUODECIM,
comitante frequentī multitūdine cum gladiīs et fustibus
ā pontificibus et scrībīs ac senātōribus.
\vs%{Mc-14-44}
Dederat autem illīs ejus prōditor signum dīcēns:
\quote{Quem osculātus fuerō, is est; eum capite et tūtō abdūcite}.
\vs%{Mc-14-45}
Igitur simul ac vēnit, eum adit et eī dīcit:
\quote{%
  Magister%
  \obsolete{, magister}};
eum‧que deosculātus est.
\vs%{Mc-14-46}
Illī vērō in eum manū injēcērunt eum‧que cēpērunt.
\vs%{Mc-14-47}
Quīdam autem eōrum quī aderant, strictō ēnse,
percussit pontificis servum eī‧que abstulit auriculam.
\vs%{Mc-14-48}
Et Jēsus illōs sīc est adlocūtus:
\christquote{%
  Tamquam ad latrōnem existis cum gladiīs et fustibus
  ad mē comprehendendum?
\vs%{Mc-14-49}
  Cottīdiē eram apud vōs in fānō docēns, neque mē cēpistis;
  sed ea fī̆erī oportet quae scrīpta sunt}.

\vs%{Mc-14-50}
Tum omnēs, eō relictō, fūgērunt.
\vs%{Mc-14-51}
Quum‧que adolēscēns quīdam eum sequerētur, linteō nūdum corpus indūtus,
adolēscentēs eum cēpērunt.
\vs%{Mc-14-52}
Ille, virō relictō linteō, nūdus eōs ecfūgit.

\vs%{Mc-14-53}
Abdūxērunt autem Jēsum ad pontificem;
ad eum‧que convēnērunt omnēs pontificēs et senātōrēs atque scrībae.
\vs%{Mc-14-54}
Eum procul secūtus Petrus ū̆sque in pontificis ātrium ūnā cum famulīs residet
sē‧que ad ign|im calefacit.
\vs%{Mc-14-55}
Pontificēs vērō tōtus‧que cōnsessus quaerēbant contrā Jēsum testimōnium,
ut eum necārent, nec inveniēbant.
\vs%{Mc-14-56}
Nam multī quidem adversus eum falsa testificābantur,
sed cōnsentānea nōn erant testimōnia.
\vs%{Mc-14-57}
Et quīdam exstitērunt, quī contrā eum falsa testābantur, ita dīcentēs:
\vs%{Mc-14-58}
\quote{%
  Nōs eum audīvimus dīcentem,
  \indirectquote{sē templum hoc manū factum esse dissolūtūrum
  et trīduō aliud nōn manū factum exstrūctūrum}}.
\vs%{Mc-14-59}
Sed nē sīc quidem cōnsentānea erant eōrum testimōnia.
\vs%{Mc-14-60}
Tum pontifex in medium subrēxit et Jēsum sīc interrogāvit:
\quote{Nihil‧ne respondēs? Quid est, quod hī contrā tē testantur?}.
\vs%{Mc-14-61}
At Jēsus silēbat nihil‧que respondēbat.
Rūrsum pontifex eum hujusmodī verbīs interrogāvit:
\quote{Tū‧ne es Chrīstus, celebrandī \DEĪ\ \FĪLIUS?}.
\vs%{Mc-14-62}
Et Jēsus:
\christquote{%
  Sum; ― inquit ― et
  \bibleallusion{vidēbitis hominis \FĪLIUM sedentem ad dextram potentiae
  et cum caelestibus venientem nūbibus}}.

\vs%{Mc-14-63}
Tum pontifex lacerāre suās vest|īs:
\quote{%
  Quid adhūc nōbīs opus est testibus? ― inquit ―
\vs%{Mc-14-64}
  Audīvistis inpiam vōcem. Quid vōbīs vidētur?}.
Illī vērō eum omnēs, utī morte dignum, damnārunt.

\vs%{Mc-14-65}
Coepērunt‧que eum quīdam cōnspuere eīque faciem obvēlāre et colaphōs inpingere,
ac dīcere:
\quote{Dīvīnā}; eī‧que alapās inpingēbant ministrī.

\vs%{Mc-14-66}
Quum esset autem Petrus īnfrā in ātriō, vēnit quaedam ancillārum pontificis
\vs%{Mc-14-67}
et, vīsō Petrō sēsē calefaciente, intuita eum dīcit:
\quote{Tū quoque cum Nāzarēnō Jēsu erās!}.
\vs%{Mc-14-68}
At ille negāvit, ita dīcēns:
\quote{Nesciō nec intellegō quid dīcās!}.
Deinde forās in vestibulum ēgressus est, et gallus cecinit.
\vs%{Mc-14-69}
Et ancilla, eō vīsō, rūrsum coepit dīcere iīs quī aderant,
\indirectquote{eum ex illīs esse}.
\vs%{Mc-14-70}
At ille rūrsum negāvit. Paulō post rūrsum, quī aderant, dīcunt Petrō:
\quote{%
  Profectō tū ex illīs es, nam Galilaeus es
  \obsolete{et locūtiō tua congruit}}.
\vs%{Mc-14-71}
At ille coepit dētestārī et jūrāre,
\indirectquote{sēsē nōn nōvisse hominem illum, quem illī dīcerent}.
\vs%{Mc-14-72}
Et iterum gallus cecinit; tum Petrus recordātus ejus, quod eī dīxerat Jēsus:
\indirectchristquote{fore, utī priusquam bis caneret gallus, ter eum negāret},
coepit flēre.
\stopbookchapter

\startbookchapter %Chapter 15
\startabstract
  Chrīstus ad Pīlātum. Jūdaeōrum rēx. Barabbās absolūtus.
  Jēsū subplicium, titulus, dictēria, tenebrae. Chrīstī gemitus, pōtātiō, mors.
  Vēlī scissūra, mīlitum professiō. Chrīstī monumentum.
\stopabstract

\vs%{Mc-15-1}
Māne prōtinus initō cōnsiliō, pontificēs ūnā cum senātōribus et scrībīs,
tōtus‧que cōnsessus, Jēsum conligant abductum‧que Pīlātō trādunt.
\vs%{Mc-15-2}
Eum Pīlātus interrogāvit \indirectquote{esset‧ne Jūdaeōrum rēx}.
Cui ille respondēns: \christquote{Utī tū dīcis}, inquit.
\vs%{Mc-15-3}
Quum‧que eum vehementer adcūsārent pontificēs,
\vs%{Mc-15-4}
Pīlātus eum rūrsus hīs verbīs interrogāvit:
\quote{Nihil‧ne respondēs? Vidē quanta contrā tē testificentur}.
\vs%{Mc-15-5}
At Jēsus nihil amplius respondit, adeō utī mīrārētur Pīlātus.

\vs%{Mc-15-6}
Solēbat autem in illīs fēstīs iīs laxāre ūnum vīnctum,
quemcumque postulāssent.
\vs%{Mc-15-7}
Erat quīdam nōmine Barabbās cum sēditiōsīs vīnctus,
quī in sēditiōne caedem fēcerant.
\vs%{Mc-15-8}
Itaque coepērunt vulgō clāmantēs id exigere,
quod ille semper iīs erat facere solitus.
\vs%{Mc-15-9}
Et Pīlātus sīc iīs respondit:
\quote{Vultis utī vōbīs Jūdaeōrum rēgem solvam?}.
\vs%{Mc-15-10}
Sciēbat enim, eum per invidiam ā pontificibus esse trāditum.
\vs%{Mc-15-11}
Pontificēs vērō plēbem īnstīgārunt,
utī sibi Barabbān potius ab illō laxārī postulāret.
\vs%{Mc-15-12}
Et Pīlātus rūrsus eōs adloquēns:
\quote{%
  Quid ergō vultis, ― inquit ― utī dē eō faciam,
  quem Jūdaeōrum rēgem dīcitis?}.
\vs%{Mc-15-13}
At illī dēnuō clāmābant: \quote{Crucifīge eum!}.
\vs%{Mc-15-14}
Et Pīlātus: \quote{Quid enim malī fēcit?}, inquit iīs.
At illī eō ācrius clāmāre: \quote{Crucifīge eum!}.
\vs%{Mc-15-15}
Igitur Pīlātus, volēns plēbī satisfacere,
laxāvit iīs Barabbān Jēsum‧que verberāvit et iīs crucifīgendum trādidit.

\vs%{Mc-15-16}
Et mīlitēs eō extrā ātrium, hoc est praetōrium, ēductō,
tōtam cohortem convocant,
\vs%{Mc-15-17}
eum‧que purpurā induunt et eī inplexam spīneam corōnam inpōnunt;
\vs%{Mc-15-18}
eum‧que sīc salūtāre incipiunt: \quote{Salvē, rēx Jūdaeōrum!},
\vs%{Mc-15-19}
ejus‧que caput arundine tundunt et eum cōnspuunt ac,
positīs genibus, venerantur.
\vs%{Mc-15-20}
Deinde, postquam eī inlūsērunt, spoliant eum purpurā,
ipsī̆us‧que vestīmentīs induunt; tum crucifīgendum ēdūcunt.

\vs%{Mc-15-21}
Et praetereuntem quemdam Simōnem Cȳrēnaeum rūre venientem,
Alexandrī Rūfī‧que patrem, cōgunt ejus ferre crucem.
\vs%{Mc-15-22}
Eum‧que ad locum \implication{Golgotha} dūcunt,
quod \explication{Calvāriae locum} significat.
\vs%{Mc-15-23}
Et eī myrrhinum vīnum bibere dant, id quod ille nōn adcēpit.

\vs%{Mc-15-24}
Eō autem crucifīxō, \bibleallusion{partītī sunt} ejus
\bibleallusion{vestīmenta, jactā super iīs sorte}, quid quisque auferret.
\vs%{Mc-15-25}
Erat hōra tertia quum eum crucifīxērunt,
\vs%{Mc-15-26}
īnscrīptō hōc ejus crīminis titulō: \tit{Rēx Jūdaeōrum}.
\vs%{Mc-15-27}
Et cum eō latrōnēs duōs,
alterum ad dexteram, alterum ad sinistram, crucifīxērunt.
\vs%{Mc-15-28}
\obsolete{Ita comprobātum est scrīptum illud, quod sīc habet:
\biblequote{Et inter scelerātōs habitus est}.}

\vs%{Mc-15-29}
Praetereuntēs autem eī maledīcēbant, \bibleallusion{capita quassantēs},
et dīcentēs:
\quote{%
  Vah, quī templum dissolvis, et trīduō īnstaurās;
\vs%{Mc-15-30}
  servā tē ipsum et dē cruce dēscende!}.
\vs%{Mc-15-31}
Similiter et pontificēs inter sēsē inlūdentēs cum scrībīs dīcēbant:
\quote{%
  Aliōs servāvit, sē ipsum servāre nōn potest.
\vs%{Mc-15-32}
  Chrīstus ille Isrāēlītārum rēx dēscendat nunc dē cruce, ut eō vīsō crēdāmus}.
Item quī cum eō erant crucifīxī, convīcium eī faciēbant.

\vs%{Mc-15-33}
Hōrā sextā autem exstitērunt per tōtam terram tenebrae, ū̆sque ad hōram nōnam.
\vs%{Mc-15-34}
Ac hōrā nōnā exclāmāvit Jēsus magna vōce dīcēns:
\christquote{%
  \bibleallusion{Eli, Eli, lama sabachthani?}},
quōrum verbōrum haec nōtiō est:
\christquote{%
  \bibleallusion{Mī \DEE, mī \DEE, cūr mē dērelīquistī?}}.
\vs%{Mc-15-35}
Quō audītō, quīdam eōrum quī aderant, dīcēbant \indirectquote{eum Ēliān vocāre}.
\vs%{Mc-15-36}
Quīdam autem adcurrit,
et, inplētā \bibleallusion{acetō} spongiā arundinī‧que adfīxā,
eī \bibleallusion{bibere dat} dīcēns:
\quote{Sinite, videāmus an eum ēreptum veniat Ēliās}.
\vs%{Mc-15-37}
Et Jēsus, ēmissā magnā vōce, exspīrāvit.

\vs%{Mc-15-38}
Et templī vēlum ā summō ad īmum in duās part|īs fissum est.

\vs%{Mc-15-39}
Ac vidēns centuriō, quī eī ex adversō astābat, eum ita clāmantem exspīrāsse,
dīxit: \quote{Profectō hic homō \FĪLIUS erat \DEĪ}.

\vs%{Mc-15-40}
Erant etiam mulierēs procul spectantēs, in quibus erat Marīa Magdalēna
et Marīa Jācōbī minōris Jōsētos‧que māter et Salōmē.
\vs%{Mc-15-41}
Quae, etiam dum ille erat in Galilaeā, eum sequī eī‧que ministrāre cōnsuēverant;
item‧que aliae multae, quae cum eō Hierosolyma ascenderant.

\vs%{Mc-15-42}
Postquam jam advesperāvit, quum esset Praeparātiō, hoc est, antesabbatum,
\vs%{Mc-15-43}
vēnit Jōsēphus ab Harimathaeā honestus senātor,
quī et ipse \DEĪ\ \RĒGNUM exspectābat;
et ad Pīlātum ausus ingredī, petiit Jēsū corpus.
\vs%{Mc-15-44}
Pīlātus mīrātus sī jam mortuus esset, arcessītō centuriōne,
scīscitātus est ex eō, quam dūdum esset mortuus;
\vs%{Mc-15-45}
quō ā centuriōne cognitō, dōnāvit Jōsēphō corpus.
\vs%{Mc-15-46}
Et ille ēmptō linteō eum dēmīsit et involūtum linteō posuit in monumentō,
quod ex petrā excīsum erat et monumentī ōstium ingēns saxum advolvit,
\vs%{Mc-15-47}
spectantibus Marīa Magdalēna et Marīa Jōsētos mātre, ubinam pōnerētur.
\stopbookchapter

\startbookchapter %Chapter 16
\startabstract
  Mulierum unguenta. Redivīvus Jēsus in Galilaeam. Īdem ter vīsus.
  Discipulōrum incrēdulitās. Missiō \ĒVANGELIĪ, et baptismī mandāta.
  Notae crēdentium. In caelum Jēsus. Apostolōrum obficium.
\stopabstract

\vs%{Mc-16-1}
Exāctō Sabbatō, Marīa Magdalēna et Marīa Jācōbī māter et Salōmē,
ēmptīs arōmatis, ut eum ūnctum īrent,
\vs%{Mc-16-2}
valdē māne prīmā post Sabbatum diē, veniunt ad monumentum, ortō jam sōle.
\vs%{Mc-16-3}
Ac dum inter sēsē quaerunt,
\indirectquote{quis sibi saxum avolveret ab ōstiō monumentī},
\vs%{Mc-16-4}
intuitae vident avolūtum esse saxum, quod erat permagnum.
\vs%{Mc-16-5}
Ac monumentum ingressae vīdērunt adolēscentem ad dexteram sedentem,
albā stolā vestītum, et expāvērunt.
\vs%{Mc-16-6}
Et ille:
\quote{%
  Nē expavēscite! ― inquit iīs ― Jēsum quaeritis Nāzarēnum crucifīxum.
  Subrēxit, nōn est hīc; ēn locus ubī̆ positus fuit.
\vs%{Mc-16-7}
  Sed īte nūntiātum ejus discipulīs et Petrō
  \indirectsubquote{eum praeīre iīs in Galilaeam};
  illīc eum esse vīsūrōs, ut iīs praedīxit}.
\vs%{Mc-16-8}
Igitur illae ēgressae celeriter ex monumentō, fūgērunt, id‧que tremōre conreptae
atque adtonitae. Neque cuiquam quidquam dīxērunt, utpote quae metuērunt.

\vs%{Mc-16-9}
Resubrēxit autem māne, prīmō post Sabbatum diē,
et adpāruit prīmum Marīae Magdalēnae, ex quā septem ējēcerat daemonia;
\vs%{Mc-16-10}
id quod illa nūntiātum iit iīs, quī cum eō fuerant,
lūgentibus atque plōrantibus;
\vs%{Mc-16-11}
at illī audientēs eum vīvere, et ab eā vīsum esse, nōn crēdidērunt.
\vs%{Mc-16-12}
Posteā duōbus eōrum ambulantibus adpāruit aliā formā, rūs euntibus;
\vs%{Mc-16-13}
id quod illī cēterīs nūntiātum iērunt, sed nē illīs quidem crēdidērunt.

\vs%{Mc-16-14}
Tandem discumbentibus illīs \ŪNDECIM adpāruit,
et iīs incrēdulitātem ac pervicāciam exprobrāvit, quod iīs,
quī sē resubrēxisse vīderant, nōn crēdidissent.
\vs%{Mc-16-15}
Iīs‧que dīxit:
\christquote{%
  Īte in tōtum orbem, praedicāte \ĒVANGELIUM omnī creātūrae.
\vs%{Mc-16-16}
  Quī crēdiderit, et baptīzātus fuerit, servābitur;
  quī vērō nōn crēdiderit, condemnābitur.
\vs%{Mc-16-17}
  Signa autem eōs quī crēdiderint, haec sequentur:
  in nōmine meō daemonia ējicient, linguīs loquentur novīs;
\vs%{Mc-16-18}
  serpent|īs tollent, et sī quid mortiferum biberint, nōn iīs nocēbit,
  quum aegrōtīs manūs inpōnent, illī bene habēbunt}.

\vs%{Mc-16-19}
Igitur \DOMINUS postquam eōs adlocūtus est, sublātus est in caelum,
et ad \DEĪ dextram sēdit.

\vs%{Mc-16-20}
Illī vērō dīgressī, ubīque praedicārunt, \DOMINŌ adjuvante
et ōrātiōnem sequentibus signīs cōnfirmante.
\stopbookchapter
\stopbook
\stopcomponent

%%% Local Variables:
%%% coding: utf-8
%%% mode: context
%%% TeX-master: t
%%% End:
