% Copyright (C) 2024 A. G. Th. Vidovicus.

% This program is free software: you can redistribute it and/or modify it
% under the terms of the GNU General Public License as published
% by the Free Software Foundation, either version 3 of the License,
% or (at your option) any later version.

% This program is distributed in the hope that it will be useful,
% but WITHOUT ANY WARRANTY; without even the implied warranty
% of MERCHANTABILITY or FITNESS FOR A PARTICULAR PURPOSE.
% See the GNU General Public License for more details.

% You should have received a copy of the GNU General Public License along
% with this program. If not, see <https://www.gnu.org/licenses/>.

% ┌──────────┐
% │01.07.2024│
% └──────────┘

\startcomponent 1_john
\startbook[
  bookmark={Epistola prima Iohannis},
   marking={Epistola prima Iohannis},
     title={Jōhannis Apostolī Epistola Prīma},
]
    
\startintroduction[title={Argūmentum Athanāsiī in Prīmam Jōhannis Epistolam}]
  Hōc nōmine vocātur et ista epistola, proptereā quod ā Jōhanne Ēvangelistā
  scrīpta est, ut conmonefaceret eōs, quī jam \DOMINUM crēdiderant.
  Et prīmum quidem perinde, atque in \ĒVANGELIŌ, ita et in hāc epistolā,
  dē \VERBĪ dīvīnitāte disserit,
  ostendēns illud in \DEŌ semper esse ac docēns \PATREM esse lūcem,
  ut \VERBUM ita cognōscāmus ex \PATRE esse,
  tamquam fulgōrem ē lūce.
  Disserēns autem dē dīvīnitāte \VERBĪ expōnit
  fideī nostrae mystērium nōn esse rem novīciam,
  sed esse sempiternum ab initiō:
  nunc vērō manifestātum in \DOMINŌ, quia vīta est sempiterna et vērus \DEUS.
  Causam etiam, ob quam \VERBUM ad nōs vēnerit et adpāruerit, pōnit hanc:
  vidēlicet,
  ut opera \DIABOLĪ dissolveret ac nōs ā morte līberāret ecficeret‧que,
  utī \PATREM agnōscerēmus et \FĪLIUM, \DOMINUM nostrum Jēsum Chrīstum.

  Scrībit igitur ad quamvīs aetātem:
  ad puerōs, ad adolēscent|īs, ad senēs, quod \DEUS innōtuerit;
  \DIABOLICA vērō operātiō deinceps, dēlētā morte, dēvicta sit.
  Quod reliquum est per tōtam epistolam, dīlēctiōnem docet volēns,
  ut nōs invicem alius alium dīligāmus, proptereā quod et Chrīstus dīlēxit nōs.
  Disserit itaque dē differentiā spīrituum discernitque quisnam spīritus ex \DEŌ,
  quis vērō sēductiōnis sit,
  et quandō cognōscāmur fīliī \DEĪ, quandō vērō \DIABOLĪ.
  Item cujus peccātī grātiā ōrandum sit prō hīs quī dēlīquērunt,
  et prō quō nōn sit et quod vocātiōne indignus sit,
  nec Chrīstī esse dīcī possit, quī proximum nōn dīligit.

  Ūnitātem etiam \FĪLIĪ cum \PATRE ostendit,
  et quod quī \FĪLIUM negat, nec \PATREM habeat;
  dēcernit quoque in hāc epistolā, quod sit proprium antichristī:
  nempe hoc, sī dīcat Jēsum nōn esse Chrīstum, ita ut quasi ille nōn sit,
  sē ipsum mentiendō dīcat esse Chrīstum.

  Hortātur autem per tōtam epistolam Chrīstō crēdent|īs, nē animō sint abjectō,
  sī odiō habeantur in hōc mundō, sed magis gaudeant:
  ob id quod odium mundī hujus dēclāret
  crēdent|īs migrāsse ex hōc mundō et posthāc conversārī in caelīs;
  et in calce epistolae iterum admonet \FĪLIUM\ \DEĪ esse vītam aeternam
  ac vērum \DEUM, et ut illī serviāmus, nōs‧que ipsōs ā simulācrīs cūstōdiāmus.
\stopintroduction

\startbookchapter % Chapter 1
\startabstract
  Sermōnem illum aeternum, in quō est vīta et lūx,
  adnūntiāre sē testātur et propitium fidēlibus \DEUM fore,
  sī gement|īs sub vitiōrum onere cōnfugere discant ad ipsī̆us misericordiam.
\stopabstract

\vs%{1Iō-1-1}
Quod fuit ā prīncipiō, quod audīvimus, quod vīdimus oculīs nostrīs,
quod et spectāvimus et nostrīs ipsī manibus tetigimus, ---
\vs%{1Iō-1-2}
dē sermōne vītae loquor, quae vīta patefacta est ---
id, inquam, et vīdimus et testāmur et vōbīs vītam nūntiāmus aeternam,
quae apud \PATREM erat, et nōbīs patefacta est.
\vs%{1Iō-1-3}
Quod et vīdimus et audīvimus, id vōbīs adnūntiāmus,
ut et vōs nōbīscum conmūnitātem habeātis.
Est autem nōbīs conmūnitās cum \PATRE cum‧que ejus \FĪLIŌ Jēsū Chrīstō.
\vs%{1Iō-1-4}
Atque haec vōbīs scrībimus, utī plēnā adficiāminī laetitiā. % or adficiāmur
\vs%{1Iō-1-5}
Est autem hoc nūntium, quod et ab illō audīvimus et vōbīs nūntiāmus:
\DEUM esse lūmen, nec ūllās in eō tenebrās esse.

\vs%{1Iō-1-6}
Sī nōbīs cum \DEŌ conmūniōnem esse dīcimus, et tamen in tenebrīs dēgimus,
mentīmur nec vērum facimus.
\vs%{1Iō-1-7}
Sī in lūce versāmur, ut ipse in lūce est, nōbīs est inter nōs conmūnitās:
nōs‧que Jēsū Chrīstī sanguis, ejus \FĪLIĪ, expūrgat ab omnī peccātō.

\vs%{1Iō-1-8}
Sī nōs peccātī expert|īs esse dīcimus,
fallimus nōs ipsōs nec est in nōbīs vēritās.
\vs%{1Iō-1-9}
Sī cōnfitēmur peccāta nostra, ille ita fidēlis et jūstus est,
utī nōbīs peccātōrum veniam det nōs‧que ab omnī culpā expiet.
\vs%{1Iō-1-10}
Sī nōs peccāsse negāmus, mendācem facimus eum, nec est in nōbīs ejus sermō.
\stopbookchapter

\startbookchapter % Chapter 2
\startabstract
  Prōpositō Chrīstō mediātōre et advocātō,
  docet in vītae sānctimōnia sitam esse \DEĪ cognitiōnem,
  quam cujusvīs aetātis et conditiōnis hominibus conmūnem facit,
  modo ūnī Chrīstō adhaereant.
  Tandem ubi ad contemptum mundī hortātus est,
  suādet antichristōs fugiendōs esse,
  inhaerendum autem agnitae vēritātī.
\stopabstract

\vs%{1Iō-2-1}
Fīliolī meī, haec vōbīs scrībō, utī nōn peccētis;
quodsī quis peccāverit, patrōnum habēmus ad \PATREM, Jēsum Chrīstum jūstum.
\vs%{1Iō-2-2}
Quī idem piāculum est peccātōrum nostrōrum:
nec sōlum nostrōrum, sed etiam tōtī̆us mundī.

\vs%{1Iō-2-3}
Ac in eō nōbīs eum esse nōtum intellegimus, sī ejus praeceptīs pārēmus.
\vs%{1Iō-2-4}
Quī \indirectquote{eum sibi nōtum esse} dīcit,
et tamen ejus praeceptīs nōn pāret,
mendāx est, nec est in eō vēritās;
\vs%{1Iō-2-5}
quī vērō ejus dictīs pāret, is vērē \DEĪ amōre perfectō praeditus est.
Hāc rē nōs in eō esse cognōscimus.
\vs%{1Iō-2-6}
Quod quī sē in eō manēre dīcit, dēbet, ut ille sē gessit, ita ipse sē gerere.

\vs%{1Iō-2-7} % orig. Frātrēs
Cārissimī, nōn novam praeceptiōnem scrībō vōbīs sed veterem praeceptiōnem,
quam ā prīncipiō habuistis:
praeceptiō vetus est sermō, quem audīvistis prīncipiō.
\vs%{1Iō-2-8}
Rūrsus novam praeceptiōnem vōbīs scrībō, quae rēs et in sē et in vōbīs vēra est:
vidēlicet tenebrās praeterīre jam‧que vērum lūmen adpārēre.
\vs%{1Iō-2-9}
Quī sē in lūmine esse dīcit, et tamen frātrem suum ōdit,
is adhūc in tenebrīs est.
\vs%{1Iō-2-10}
Quī frātrem suum amat, is in lūce manet, et is est, quī nōn inpingat;
\vs%{1Iō-2-11}
quī vērō frātrem suum ōdit, is in tenebrīs est et in tenebrīs graditur
neque scit quō eat, quippe quum ejus oculōs obcaecāverint tenebrae.

% The next three verses sound rather unclassical,
% as the subordinate clause is introduced by quod or quia
% instead of accusative with infinitive construction
\vs%{1Iō-2-12}
Scrībō vōbīs, fīliolī, quod vōbīs ignōscuntur peccāta per ejus nōmen.
\vs%{1Iō-2-13}
Scrībō vōbīs, patrēs, quod eum, quī ā prīncipiō est, nōvistis.
Scrībō vōbīs, juvenēs, quod \MALUM vīcistis.
\vs%{1Iō-2-14}
Scrībō vōbīs, gnātī, quia \PATREM nōstis.
Scrīpsī vōbīs, patrēs, quia eum, quī ā prīncipiō est, nōstis.
Scrīpsī vōbīs, adolēscentēs, quoniam fortēs estis,
in vōbīs‧que manet \DEĪ sermō, et \MALUM vīcistis.

\vs%{1Iō-2-15}
Nē amāte mundum neque mundāna.
Sī quis mundum amat, is nōn est \PATRIS cāritāte praeditus:
\vs%{1Iō-2-16}
nam quidquid in mundō est,
vidēlicet carnis et oculōrum cupiditās et vītae adrogantia,
nōn ā \PATRE, sed ā mundō est.
\vs%{1Iō-2-17}
Et mundus abit et ejus cupiditātēs;
sed quī \DEĪ voluntātī pāret, is manet in sempiternum.

\vs%{1Iō-2-18}
Gnātī, ultimum tempus est;
et quemadmodum ventūrum audīvistis antichristum,
nunc quoque multī sunt antichristī:
unde ultimum tempus esse intellegimus.
\vs%{1Iō-2-19}
Ā nōbīs profectī sunt, sed nōn erant ex nōbīs:
sī enim ex nōbīs fuissent, nōbīscum mānsissent;
sed dēclārandum fuit, nōn omn|īs ex nōbīs esse.
\vs%{1Iō-2-20}
Et vōs ā \SĀNCTŌ ūnctī nōvistis omnia.
\vs%{1Iō-2-21}
Nōn ideō vōbīs scrīpsī, quod vērum nesciātis,
sed quod vērum scītis et quod nihil falsum ex vērō est.
\vs%{1Iō-2-22}
Quī est mendāx, nisi quī negat, Jēsum esse Chrīstum?
Is antichristus est, quī et \PATREM negat et \FĪLIUM.
\vs%{1Iō-2-23}
Quisquis \FĪLIUM negat, nē \PATREM quidem habet.

\vs%{1Iō-2-24}
Quod ergō vōs ab initiō audīvistis, id in vōbīs maneat;
sī in vōbīs mānserit, quod initiō audīvistis,
et vōs in \FĪLIŌ ac \PATRE manēbitis.
\vs%{1Iō-2-25}
Ac quod nōbīs prōmīsit, id vīta aeterna est.

\vs%{1Iō-2-26}
Haec vōbīs dē vestrīs dēceptōribus scrīpsī.
\vs%{1Iō-2-27}
Et tamen iī estis, in quibus ūnctiō, quam ab eō accēpistis, maneat;
neque vōbīs opus est,
utī quisquam vōs doceat, quippe quum eadem ūnctiō vōs dē cūnctīs doceat,
vēra‧que sit et falsitātis expers;
ac quemadmodum ille vōs docuit, sīc maneātis in eō.

\vs%{1Iō-2-28}
Et nunc, fīliolī, manēte in eō, utī,
quum exsistet, cōnfīdentiam habeāmus nec eum in ejus adventū ērubēscāmus.
\vs%{1Iō-2-29}
Sī scītis eum esse jūstum, cognōscite, quīcumque jūsta faciat,
ex eō esse nātum.
\stopbookchapter

\startbookchapter % Chapter 3
\startabstract
  Inaestimābilem dīvīnae adoptiōnis honōrem conmendāns,
  testandam esse bonīs operibus vītae novitātem docet,
  cujus certum symbolum est cāritās.
  Deinde fīdūciam et \DEĪ invocātiōnem subnectit.
\stopabstract

\vs%{1Iō-3-1}
Vidēte quantō nōs prōsecūtus sit amōre \PATER, utī \DEĪ fīliī vocēmur,
\added{et sumus!}
Ideō nōs mundus nōn cognōscit, quod eum nōn cognōscit.
\vs%{1Iō-3-2}
Cārissimī, nōs nunc \DEĪ fīliī sumus, tametsī nōndum patet, quid sīmus futūrī;
scīmus autem, quum id patefactum fuerit, nōs ejus fore simil|īs:
utpote quem cernēmus, ut est.

\vs%{1Iō-3-3}
Ac quisquis hanc in eō spem habet, is sēipsum expūrgat,
quemadmodum ille pūrus est.
\vs%{1Iō-3-4} % crīmen or noxa
Quisquis peccātum conmittit, is crīmen conmittit, et peccātum crīmen est.
\vs%{1Iō-3-5}
Et scītis, illum ideō advēnisse, utī peccāta nostra auferret,
nec ūllum in eō esse peccātum.
\vs%{1Iō-3-6}
Quisquis in eō manet, nōn peccat; quisquis peccat, eum nōn didicit neque nōvit.

\vs%{1Iō-3-7}
Fīliolī, nēmō vōs dēcipiat.
Quī jūsta facit, jūstus est, quemadmodum ille jūstus est.
\vs%{1Iō-3-8}
Quī peccātum conmittit, ex \DIABOLŌ est: sī̆quidem ā prīncipiō \DIABOLUS peccat.
Ideō patefactus est \DEĪ\ \FĪLIUS, utī \DIABOLĪ opera abolēret.
\vs%{1Iō-3-9}
Quisquis ex \DEŌ nātus est, peccātum nōn conmittit, quod \DEĪ sēmen in eō manet;
ideō‧que peccāre nequit, quod ex \DEŌ nātus est.

\vs%{1Iō-3-10}
In eō manifēstī sunt \DEĪ fīliī et \DIABOLĪ fīliī,
quod quisquis jūsta nōn facit, is \DEĪ nōn est;
item‧que quī frātrem suum nōn amat.

\vs%{1Iō-3-11}
Nam haec est, quam prīncipiō audīvistis, dēnūntiātiō, ut amēmus inter nōs:
\vs%{1Iō-3-12}
nōn utī Caīnus, quī, quum ā \MALŌ esset, frātrem suum interēmit.
Cujus autem grātiā eum interemit?
Quoniam ejus opera erant mala, quum frātris essent jūsta.

\vs%{1Iō-3-13}
Nōlīte mīrārī, frātrēs meī, sī vōs ōdit mundus.
\vs%{1Iō-3-14}
Nōs scīmus nōs ā morte ad vītam migrāsse, quoniam frātrēs amāmus;
quī frātrem nōn amat, manet in morte:
\vs%{1Iō-3-15}
Quisquis frātrem suum ōdit, homicīda est;
scītis autem, nēminem esse homicīdam, quī aeternam vītam habeat in sē manentem.

\vs%{1Iō-3-16}
Inde amōrem cognōscimus, quod et ille prō nōbīs animam suam posuit;
et nōs dēbēmus prō frātribus animās pōnere.
\vs%{1Iō-3-17}
Jam quī, quum hujus vītae facultātēs habeat, et frātrem suum egēre videat,
tamen nōn ejus vicem miserātur,
quī fierī potest, utī sit \DEĪ cāritāte praeditus?
\vs%{1Iō-3-18}
Fīliolī meī, nōn verbīs aut linguā, sed rē ac vēritāte amēmus.
\vs%{1Iō-3-19}
Ita fīet, utī nōs ā vēritāte esse intellegāmus,
et id in ejus cōnspectū animīs persuāsum habeāmus;
\vs%{1Iō-3-20}
quodsī nōs damnat animus noster,
major est \DEUS quam noster animus et cūncta nōvit.
\vs%{1Iō-3-21}
Cārissimī, sī noster animus nōs nōn damnat, cōnfīdentiam habēmus ad \DEUM;
\vs%{1Iō-3-22}
ac quidquid postulāmus, id ab eō impetrāmus,
quoniam ejus praeceptīs obtemperāmus eī‧que mōrem gerimus.

\vs%{1Iō-3-23}
Atque haec est ejus praeceptiō, ut ejus \FĪLIĪ Jēsū Chrīstī nōminī crēdāmus
et inter nōs amēmus, quemadmodum praeceptum dēdit.
\vs%{1Iō-3-24}
Ac quī ejus praeceptīs oboedit, et ipse in illō, et ille in ipsō manet;
atque hinc eum in nōbīs manēre cognōscimus,
vidēlicet ex \SPĪRITŪ, quō nōs dōnāvit.
\stopbookchapter

\startbookchapter % Chapter 4
\startabstract
  Ubī̆ quaedam īnseruit dē probandīs spīritibus,
  quia aliī ex mundō aliī ex \DEŌ loquuntur;
  ad cāritātem redit et \DEĪ exemplō ad frāternum amōrem hortātur.
\stopabstract

\vs%{1Iō-4-1}
Cārissimī, nē cuivīs spīrituī crēdite, sed spīritūs explōrāte an ex \DEŌ sint:
nam multī falsī vātēs in mundum prōdiērunt.
\vs%{1Iō-4-2}
Hinc cognōscite \DEĪ spīritum:
quisquis spīritus Jēsum Chrīstum in carne vēnisse cōnfitētur, is ā \DEŌ est.
\vs%{1Iō-4-3}
Quisquis autem spīritus Jēsum Chrīstum in carne vēnisse nōn cōnfitētur,
is ā \DEŌ nōn est;
atque hoc illud est dē Antichristō, quod audīvistis ventūrum esse,
quī jam nunc in mundō est.
\vs%{1Iō-4-4}
Vōs ā \DEŌ estis, fīliolī, et eōs vīcistis,
quoniam major est quī in vōbīs quam quī in mundō est.
\vs%{1Iō-4-5}
Illī ā mundō sunt: itaque ā mundō loquuntur eōs‧que mundus audit.
\vs%{1Iō-4-6}
Nōs ā \DEŌ sumus: \DEUM quī nōvit, nōs audit; quī ā \DEŌ nōn est, nōn audit nōs.
Inde quis vēritātis, quis sit errōris spīritus, dignōscimus.

\vs%{1Iō-4-7} % cāritās or amor
Cārissimī, amēmus nōs inter nōs: nam cāritās ā \DEŌ est;
ac quisquis amat, ex \DEŌ nātus est {\DEUM}‧que nōvit.
\vs%{1Iō-4-8}
Quī nōn amat, nōn nōvit \DEUM: nam \DEUS amor est.
\vs%{1Iō-4-9}
In eō patuit \DEĪ ergā nōs amor,
quod \FĪLIUM suum ūnicum mīsit \DEUS in mundum, utī per eum vīvāmus.
\vs%{1Iō-4-10}
In eō amor est, quod \DEUS, nōn quod nōs eum amāverīmus,
sed quod ipse nōs amāverit, mīsit \FĪLIUM suum, prō peccātīs nostrīs piāculum.

\vs%{1Iō-4-11}
Cārissimī, sī sīc nōs amāvit \DEUS, nōs vicissim dēbēmus amāre nōs mūtuō.
\vs%{1Iō-4-12} or perfectō amōre
\DEUM nēmō umquam vīdit; sī alius alium amāmus,
manet in nōbīs \DEUS sumus‧que ejus perfectā cāritāte praeditī.
\vs%{1Iō-4-13}
Hinc et nōs in eō et in nōbīs eum manēre intellegimus,
quod nōbīs suum \SPĪRITUM impertīvit;
\vs%{1Iō-4-14}
et nōs vīdimus et testāmur ā \PATRE lēgātum esse \FĪLIUM, servātōrem mundī.
\vs%{1Iō-4-15}
Quisquis Jēsum \DEĪ esse \FĪLIUM cōnfessus fuerit,
et in eō \DEUS, et is in \DEŌ manet.
\vs%{1Iō-4-16}
Et nōs amōrem, quō nōs \DEUS prōsequitur, nōvimus ac crēdimus.
\DEUS amor est; et quī in amōre manet, in \DEŌ manet, et \DEUS in eō.

\vs%{1Iō-4-17}
In eō perfectā amōre praeditī sumus, sī dē jūdiciī diē cōnfīdimus:
quod quālis ille est, tālēs nōs sīmus in hōc mundō.
\vs%{1Iō-4-18}
In cāritāte metus nōn est,
quīn perfecta cāritās forās quatit metum:
quoniam metus cruciātum habet,
quīque metuit, perfectā cāritāte nōn est.

\vs%{1Iō-4-19}
Nōs eum amāmus, quoniam prior ipse nōs amāvit.
\vs%{1Iō-4-20}
Sī quis ā sē \DEUM amārī dīcit, et tamen frātrem suum ōdit, mendāx est:
quī enim frātrem suum, quem vīdit, nōn amat,
\DEUM, quem nōn vīdit, quō pactō potest amāre?
\vs%{1Iō-4-21}
Atque hoc ab eō praeceptum habēmus,
utī \DEUM quī amat, suum quoque frātrem amet.
\stopbookchapter

\startbookchapter % Chapter 5
\startabstract
  Ostendit frāternum amōrem et fidem esse rēs conjūnctās:
  nūllam autem habērī posse fidem \DEŌ, nisi crēdendō in Chrīstum.
  Hinc certa est invocātiō, utī prō frātribus etiam valeant precēs nostrae.
\stopabstract

\vs%{1Iō-5-1}
Quisquis Jēsum crēdit esse Chrīstum, ex \DEŌ nātus est;
et quisquis genitōrem amat, etiam ex eō nātum amat.
\vs%{1Iō-5-2}
Hinc intellegimus amārī ā nōbīs \DEĪ fīliōs,
sī \DEUM amāmus ejus‧que praeceptīs obtemperāmus.
\vs%{1Iō-5-3}
Is enim \DEĪ amor est, sī ejus praeceptīs obtemperāmus:
quae sānē ejus praecepta gravia nōn sunt.
\vs%{1Iō-5-4}
Nam quidquid ex \DEŌ nātum est, vincit mundum,
et haec victōria est, quae vīcit mundum: fidēs nostra.

\vs%{1Iō-5-5}
Quis est, quī vincit mundum, nisi quī crēdit Jēsum esse \DEĪ fīlium?
\vs%{1Iō-5-6}
Hic est, quī vēnit per aquam et sanguinem, Jēsus Chrīstus;
nōn in aquā sōlum, sed in aquā et sanguine.
Et \SPĪRITUS est, quī testātur, sī̆quidem \SPĪRITUS est vēritās.
\vs%{1Iō-5-7}
% The Johannine Comma
Quoniam trēs sunt, quī testantur \obsolete{in caelō:
\PATER, sermō, et \SPĪRITUS\ \SĀNCTUS;
et hī trēs ūnum sunt.
\vs%{1Iō-5-8}
Item trēs sunt, quī testantur} in terrā:
\SPĪRITUS et aqua et sanguis; quī trēs ūnum sunt.
\vs%{1Iō-5-9}
Sī hominum testimōnium admittimus, \DEĪ testimōnium majus est.

Atque hoc est, quod \DEUS dē \FĪLIŌ suō testimōnium dīxit:
\vs%{1Iō-5-10}
quī \DEĪ\ \FĪLIŌ fidem habet, is in sē \DEĪ testimōnium habet;
quī \DEŌ nōn crēdit, mendācem facit eum,
quod testimōniō fidem nōn habeat, quod dīxit \DEUS dē \FĪLIŌ suō.
\vs%{1Iō-5-11}
Est autem hoc testimōnium, quod aeternā vītā dōnāvit nōs \DEUS,
quae vīta in ejus \FĪLIŌ est:
\vs%{1Iō-5-12}
quī \DEĪ\ \FĪLIUM habet, vītam habet;
quī \DEĪ\ \FĪLIUM nōn habet, vītam nōn habet.

\vs%{1Iō-5-13}
Haec vōbīs, nōminī \FĪLIĪ\ \DEĪ fidem habentibus, ideō scrīpsī,
utī vōs aeternam habēre vītam sciātis
utī‧que nōminī \FĪLIĪ\ \DEĪ fidem adjungātis.

\vs%{1Iō-5-14}
Atque ea est, quam ad eum habēmus, cōnfīdentia,
quod sī quid ex ejus voluntāte postulāmus, audit nōs.
\vs%{1Iō-5-15}
Quodsī nōs ab eō audīrī scīmus, quidquid poscāmus, scīmus,
ea obtinuisse nōs, quae ab eō poposcimus.

\vs%{1Iō-5-16}
Sī quis frātrem suum peccātum nōn lētāle conmittere vīderit,
sī is \DEUM poscat, \DEUS illī vītam dabit, vidēlicet peccantibus nōn lētāliter.
Est lētāle peccātum prō quō ōrandum esse nōn dīcō.
\vs%{1Iō-5-17}
Omnis injūstitia est peccātum; est et peccātum nōn lētāle.

\vs%{1Iō-5-18}
Scīmus, quisquis ex \DEŌ nātus est, eum nōn peccāre:
nam ex \DEŌ nātus sēipsum servat, nec eum \MALUS attingit.
\vs%{1Iō-5-19}
Scīmus nōs ā \DEŌ esse, tōtum‧que mundum in \MALŌ jacēre.
\vs%{1Iō-5-20}
Scīmus item \DEĪ\ \FĪLIUM vēnisse et nōs eō ingeniō dōnāsse,
utī vērum cognōscāmus \DEUM;
sumus‧que in vērō, in ejus \FĪLIŌ Jēsū Chrīstō.
Is vērus est \DEUS aeterna‧que vīta.
\vs%{1Iō-5-21}
Fīliolī, cavēte \unclassical{deastrōs}! \obsolete{Āmēn.}
\stopbookchapter
\stopbook
\stopcomponent

%%% Local Variables:
%%% coding: utf-8
%%% mode: context
%%% TeX-master: t
%%% End:
